\documentclass[11pt, ngerman, fleqn, DIV=15, headinclude, BCOR=1cm]{scrartcl}

\usepackage[bibatend]{../header}

\usepackage{booktabs}

\usepackage{tikz}

\usepackage[tikz]{mdframed}
\newmdtheoremenv[%
    backgroundcolor=black!5,
    innertopmargin=\topskip,
    splittopskip=\topskip,
]{theorem}{Theorem}[section]

\newmdenv[%
    backgroundcolor=black!5,
    frametitlebackgroundcolor=black!10,
    roundcorner=5pt,
    innertopmargin=\topskip,
    splittopskip=\topskip,
    frametitle={Problem Statement},
    frametitlerule=true,
]{problem}

\hypersetup{
    pdftitle=
}

\newcounter{totalpoints}
\newcommand\punkte[1]{#1\addtocounter{totalpoints}{#1}}

\newcounter{problemset}
\setcounter{problemset}{1}

\subject{physics755 -- Quantum Field Theory}
\ihead{physics755 -- Problem Set \arabic{problemset}}

\title{Problem Set \arabic{problemset}}

\publishers{Group Friday}
\ofoot{Group Friday}



\author{
    Martin Ueding \\ \small{\href{mailto:mu@martin-ueding.de}{mu@martin-ueding.de}}
}
\ifoot{Martin Ueding}

\ohead{\rightmark}

\begin{document}

\maketitle

\vspace{3ex}

\begin{center}
    \begin{tabular}{rrr}
        problem number & achieved points & possible points \\
        \midrule
        H 1.1 & & \punkte{10} \\
        \midrule
        total & & \arabic{totalpoints}
    \end{tabular}
\end{center}

\vspace{5ex}

I, Martin Ueding, would like to scan and upload the problem sets with your
corrections to my website \href{http://martin-ueding.de}{martin-ueding.de}.
There, the original problem set as well as the reviewed one will be licensed
under the “\href{http://creativecommons.org/licenses/by-sa/4.0/}{Creative
Commons Attribution-ShareAlike 4.0 International License}”. Is that okay with
you?

Yes $\Box$ \hspace{2cm} No $\Box$

\newpage

\section{The complex scalar field}

\subsection{Hamiltonian}

We have given:
\[
    S = \int \dif^4 x \, \sbr{
        \partial_\mu \phi^* \partial^\mu \phi - m^2 \phi^* \phi
    }.
\]
The scoping of the $\partial$ is not perfectly clear, we assume that it will
only act on the symbol directly after it. So that the first term in the action
density is the four-gradient squared.

Now we use
\[
    S = \int \dif^4 x \, \mathscr L
    \eqnsep
    L = \int \dif^3 x \, \mathscr L
\]
where $\mathscr L$ is the Lagrange density and obtain the Lagrange function
$L$:
\[
    L = \int \dif^3 x \, \sbr{
        \partial_\mu \phi^* \partial^\mu \phi - m^2 \phi^* \phi
    }.
\]

The conjugate momenta are then:
\[
    \pi_i = \pd{L}{[\partial_i \phi]} = \partial^i \phi^*.
\]

% TODO Commutation relations.

The Hamiltonian can then be obtained using a Legendre transformation:
\begin{align*}
    H
    &= \int \dif^3 x \sbr{\pi_i \partial^i \phi - L(\phi, \partial \phi(\phi,
\pi))} \\
    &= \int \dif^3 x \sbr{\pi^{*i} \pi_i - \partial_i \phi^* \partial^i \phi +
m^2 \phi^* \phi} \\
    \intertext{%
        Using the metric tensor $\eta^{ii} = -1$ one can write this as
        three-vectors:
    }
    &= \int \dif^3 x \sbr{\vec\pi \cdot \vec\pi - \vnabla \phi^* \cdot
\vnabla \phi + m^2 \phi^* \phi} \\
\end{align*}

\end{document}

% vim: spell spelllang=en tw=79
