\documentclass[11pt, english, fleqn, DIV=15, headinclude, BCOR=1cm]{scrartcl}

\usepackage[bibatend, color]{../header}

\usepackage{booktabs}

\usepackage{tikz}

\usepackage[tikz]{mdframed}
\newmdtheoremenv[%
    backgroundcolor=black!5,
    innertopmargin=\topskip,
    splittopskip=\topskip,
]{theorem}{Theorem}[section]

\newmdenv[%
    backgroundcolor=black!5,
    frametitlebackgroundcolor=black!10,
    roundcorner=5pt,
    innertopmargin=\topskip,
    splittopskip=\topskip,
    frametitle={Problem Statement},
    frametitlerule=true,
]{problem}

\hypersetup{
    pdftitle=
}

\newcounter{totalpoints}
\newcommand\punkte[1]{#1\addtocounter{totalpoints}{#1}}

\newcounter{problemset}
\setcounter{problemset}{1}

\subject{physics755 -- Quantum Field Theory}
\ihead{physics755 -- Problem Set \arabic{problemset}}

\title{Problem Set \arabic{problemset}}

\publishers{Group Friday -- Robert Lauktien}
\ofoot{Group Friday -- Robert Lauktien}



\author{
    Martin Ueding \\ \small{\href{mailto:mu@martin-ueding.de}{mu@martin-ueding.de}}
}
\ifoot{Martin Ueding}

\ohead{\rightmark}

\begin{document}

\maketitle

\vspace{3ex}

\begin{center}
    \begin{tabular}{rrr}
        problem number & achieved points & possible points \\
        \midrule
        H 1.1 & & \punkte{10} \\
        \midrule
        total & & \arabic{totalpoints}
    \end{tabular}
\end{center}

\vspace{5ex}

I, Martin Ueding, would like to scan and upload the problem sets with your
corrections to my website \href{http://martin-ueding.de}{martin-ueding.de}.
There, the original problem set as well as the reviewed one will be licensed
under the “\href{http://creativecommons.org/licenses/by-sa/4.0/}{Creative
Commons Attribution-ShareAlike 4.0 International License}”. Is that okay with
you?

Yes $\Box$ \hspace{2cm} No $\Box$

\newpage

\section{The complex scalar field}

\subsection{Hamiltonian}

\paragraph{Conjugate momenta}

I have given the following action:
\[
    S = \int \dif^4 x \, \sbr{
        \partial_\mu \phi^* \partial^\mu \phi - m^2 \phi^* \phi
    }.
\]
The scoping of the $\partial$ is not perfectly clear, I assume that it will
only act on the symbol directly after it. So that the first term in the action
density is the four-gradient squared. I will use GR notation for that:
\[
    S = \int \dif^4 x \, \sbr{
        \phi^*_{,\mu} \phi^{,\mu} - m^2 \phi^* \phi
    }.
\]

Now I use
\[
    S = \int \dif t \, L
    \eqnsep
    L = \int \dif^3 x \, \mathscr L
\]
where $\mathscr L$ is the Lagrange density and obtain the Lagrange function
$L$:
\[
    L = \int \dif^3 x \, \sbr{
        \phi^*_{,\mu} \phi^{,\mu} - m^2 \phi^* \phi
    }.
\]

The conjugate momentum densities are then:
\[
    \pi = \pd{L}{\dot \phi} = \dot \phi^{*}
    \eqnsep
    \pi^* = \pd{L}{\dot \phi^*} = \dot \phi.
\]

\paragraph{Commutation relations}

The commutation relations are chosen in a certain way to quantize the theory. I
do not think there is anything one can do to derive them right here. The
relations are, with $a$ and $b$ labelling multiple fields
\parencite[(12.3.1)]{Schwabl/Quantenmechanik_fuer_Fortgeschrittene/2008}:
\begin{gather*}
    \sbr{\phi_a(x, t), \pi_b(y, t)} = \iup \delta_{ab} \delta(x - y) \\
    \sbr{\phi_a(x, t), \phi_b(y, t)} = 0 \\
    \sbr{\pi_a(x, t), \pi_b(y, t)} = 0
\end{gather*}

Now one can insert the momenta and yield
\begin{gather*}
    \sbr{\phi(x, t), \dot\phi^*(y, t)} = \iup \delta_{ab} \delta(x - y) \\
    \sbr{\phi^*(x, t), \dot\phi(y, t)} = \iup \delta_{ab} \delta(x - y)
\end{gather*}
by also setting $\phi_a := \phi$ and $\phi_b := \phi^*$
\parencite[(13.2.4)]{Schwabl/Quantenmechanik_fuer_Fortgeschrittene/2008}.

\paragraph{Hamiltonian}

The Hamiltonian can then be obtained using a Legendre transformation. I have
not an explicit recipe for the complex valued case since
\textcite[Chapter~2]{Peskin/QFT/1995} only cover the real Klein-Gordon field
and \textcite[Chapter~13.2]{Schwabl/Quantenmechanik_fuer_Fortgeschrittene/2008}
does not give the Hamiltonian of the complex Klein-Gordon field. I figured out
a way that gives the correct result. Please tell me whether it is the correct
reasoning.

The Legendre transformation in classical mechanics is
\[
    H = \pi_i \dot q^i - L.
\]
The $i$ denote the finite number of degrees of freedom. In a field theory, the
degrees of freedoms are infinite, and therefore one has $q_i \to q(x)$ and $x$
labels the degrees of freedom. Now $q(x)$ is the \emph{one} field in
consideration. There is a momentum $\pi(x)$ which also has the same degrees of
freedom. There still has to be a sum over all the degrees of freedom, which is
now done by integration over $x$.
\[
    H = \int \dif x \, \pi(x) \dot q(x) - L.
\]

Here we have two fields, $\phi$ and $\phi^*$. Therefore, there are multiple
terms in the Hamiltonian. The equation looks really similar:
\[
    H = \int \dif x \, \pi_a(x) \dot \phi^a(x) - L.
\]
Here, the index $a$ numbers the different fields ($a = 1, 2$) like you (R.\,L.)
did in the exercise class on 2015-04-17. So the index summation is over the
different fields. The summing over $i$ is now an integral over $x$.

With this, I can find the Hamiltonian:
\begin{align*}
    H
    &= \int \dif^3 x \sbr{\pi \dot \phi + \pi^* \dot\phi^* - \mathscr L(\phi, \dif \phi(\phi,
    \pi))} \\
    \intertext{%
        Before I insert $\mathscr L$, I will replace the time derivative of the
        fields with the canonical momentum density. This will also make the
        first two summands equal, since the momenta commute.
    }
    &= \int \dif^3 x \sbr{2 \pi^* \pi - \mathscr L} \\
    \intertext{%
        Now I expand $\mathscr L$.
    }
    &= \int \dif^3 x \sbr{2 \pi^* \pi - \phi^*_{,\mu} \phi^{,\mu} +
    m^2 \phi^* \phi} \\
    \intertext{%
        Using the metric tensor $\eta^{ii} = -1$ one can separate time and
        spatial components:
    }
    &= \int \dif^3 x \sbr{2 \pi^* \pi - \sbr{\dot \phi^* \dot \phi - \vnabla \phi^* \cdot
    \vnabla \phi} + m^2 \phi^* \phi} \\
    \intertext{%
        The first term in the inner bracket is just $\pi^* \pi$ and cancels out
        part of the first summand. I am left with
    }
    &= \int \dif^3 x \sbr{ \pi^* \pi + \vnabla \phi^* \cdot \vnabla \phi + m^2 \phi^* \phi}.
\end{align*}
This is exactly the Hamiltonian given in Equation~(7) on the problem set.

\paragraph{Heisenberg equations of motion}

Now that the Hamiltonian is derived, one can use the time evolution in the
Heisenberg picture to get the equations of motion for the two fields. This is
plugging in into
\(
    \iup \dot O = [H, O]
\)
for given operator $O$ \parencite[(2.44)]{Peskin/QFT/1995}. I can insert $O =
\phi$ and $O = \pi$ to get the Hamiltonian equations of motions. I will start
with the $\phi(\vec x, t)$.
\begin{align*}
    \iup \dot\phi(\vec x, t)
    &= \sbr{\phi(\vec x, t), H} \\
    \intertext{%
        I insert the Hamiltonian. The integration has to be done over a
        different variable, I use $x'$. All the functions in the Hamiltonian
        have arguments $(\vec x', t)$ which I omitted to fit this onto a single
        line.
    }
    &= \sbr{\phi(\vec x, t),
        \int \dif^3 x' \sbr{ \pi^* \pi + \vnabla \phi^* \cdot \vnabla \phi + m^2 \phi^* \phi}
    } \\
    \intertext{%
        The integration of $\vec x'$ does not involve $\vec x$ at all, so I can
        use the linearity of the integral and move it out of the commutator. I
        now have three individual ones I can work with independently.
    }
    &= \int \dif^3 x' \sbr{
        \sbr{\phi(\vec x, t), \pi^* \pi}
        + \sbr{\phi(\vec x, t), \vnabla \phi^* \cdot \vnabla \phi}
        + m^2 \sbr{\phi(\vec x, t), \phi^* \phi}
    }
    \intertext{%
        (1) In the first term, $\phi$ commutes with $\pi^*$, so I can move this
        out of the commutator. (2) The $\phi$ commute with each other freely,
        so this commutator is just zero. (3) $\phi$ commutes with itself at any
        time, so this is also zero. What remains is
    }
    &= \int \dif^3 x' \, \pi^* \sbr{\phi(\vec x, t), \pi(\vec x', t)}.
    \intertext{%
        Using the commutation relations that I copied earlier, this can be
        written as
    }
    &= \int \dif^3 x' \, \pi^*(\vec x', t) \, \iup \, \deltaup(\vec x - \vec x')
    \intertext{%
        Now I perform the integration and obtain
    }
    &= \iup \pi^*(\vec x, t).
\end{align*}
This looks similar to the treatment of the real Klein-Gordon field by
\textcite[25]{Peskin/QFT/1995}. The time evolution of the complex conjugate
$\phi^*$ is given by $\pi(\vec x, t)$ as this whole derivation can be complex
conjugated ($H^* = H$).

Now I need to perform an analogue calculation for $\dot\pi$:
\begin{align*}
    \iup \dot\pi(\vec x, t)
    &= \sbr{\pi(\vec x, t), } \\
    &= \sbr{\pi(\vec x, t),
        \int \dif^3 x' \sbr{ \pi^* \pi + \vnabla \phi^* \cdot \vnabla \phi + m^2 \phi^* \phi}
    } \\
    \intertext{%
        Again I move the integration out of the commutator and yield three
        independent ones:
    }
    &= \int \dif^3 x' \sbr{
        \sbr{\pi(\vec x, t), \pi^* \pi}
        + \sbr{\pi(\vec x, t), \vnabla \phi^* \cdot \vnabla \phi}
        + m^2 \sbr{\pi(\vec x, t), \phi^* \phi}
    }
    \intertext{%
        The commutators are now different from the previous case with $\phi$.
        (1) This commutator is zero, since the momentum density commutes with
        itself at any point in time. (2) This will need partial integration to
        isolate $\phi$ which does not commute with $\pi$. (3) The last term is
        like the first term of the previous derivation.
    }
    &= \eval{\vnabla \phi^* \sbr{\pi(\vec x, t), \phi}}_{\vec x' = \partial
\R^3} + \int \dif^3 x' \sbr{
        - \laplace \phi^* \sbr{\pi(\vec x, t), \phi}
        + m^2 \phi^* \sbr{\pi(\vec x, t), \phi(\vec x', t)}
    }
    \intertext{%
        (1) The surface term in front will vanish on the boundary at infinity
        because everything is assumed to be normalizable and must therefore
        decay faster than $1/\sqrt r$. I will drop that term. (2) The
        commutator is the negative of the commutation relation given before.
        (3) That term is now the same as the second term and I will factor that
        out.
    }
    &= \int \dif^3 x' \sbr{\laplace \phi^* - m^2 \phi^*}
    \sbr{\phi(\vec x, t), \pi(\vec x, t)}
    \intertext{%
        That last remaining commutator gives a $\deltaup$-distribution again. I
        integrate over it to remove the $\vec x'$ and I obtain
    }
    &= [\laplace - m^2]\phi^*(\vec x, t)
\end{align*}
This also looks similar to the results of \textcite[25]{Peskin/QFT/1995}.

\paragraph{Klein-Gordon equation}

Both of those results have to be combined to give the equation of motion for
the field $\phi$. Taking the time derivative of the first equation and
inserting the second yields
\[
    \ddot \phi(\vec x, t) = [\laplace - m^2] \phi(\vec x, t).
\]
as well as the complex conjugate of this equation. This indeed is the
classical Klein-Gordon equation \parencite[(2.7)]{Peskin/QFT/1995} and the
quantum mechanical one \parencite[(2.45)]{Peskin/QFT/1995}.

\subsection{Fourier modes}

\subsection{Symmetry}

In all terms only $X^*X = |X|^2$ comes up. A constant phase factor would cancel
out in those and therefore not change the action, Lagrangian or the
Hamiltonian. The field and its derivatives and therefore the coordinates would
change. But the equations of motions would retain their form.

\subsection{Conserved charge}

\subsection{Two fields}

\end{document}

% vim: spell spelllang=en tw=79
