\documentclass[11pt, english, fleqn, DIV=15, headinclude, BCOR=1cm]{scrartcl}

\usepackage[bibatend, color]{../header}
\usepackage{../my-boxes}

\usepackage{booktabs}

\hypersetup{
    pdftitle=
}

\newcounter{totalpoints}
\newcommand\punkte[1]{#1\addtocounter{totalpoints}{#1}}

\newcounter{problemset}
\setcounter{problemset}{1}

\subject{physics755 -- Quantum Field Theory}
\ihead{physics755 -- Problem Set \arabic{problemset}}

\title{Problem Set \arabic{problemset}}

\newcommand\thegroup{Group Tuesday -- Ripunjay Acharya}

\publishers{\thegroup}
\ofoot{\thegroup}

\author{
    Martin Ueding \\ \small{\href{mailto:mu@martin-ueding.de}{mu@martin-ueding.de}}
}
\ifoot{Martin Ueding}

\ohead{\rightmark}

\begin{document}

\maketitle

\vspace{3ex}

\begin{center}
    \begin{tabular}{rrr}
        problem & achieved points & possible points \\
        \midrule
        \nameref{homework:1} & & \punkte{10} \\
        \midrule
        total & & \arabic{totalpoints}
    \end{tabular}
\end{center}

\vspace{5ex}

I, Martin Ueding, would like to scan and upload the problem sets with your
corrections to my website \href{http://martin-ueding.de}{martin-ueding.de}.
There, the original problem set as well as the reviewed one will be licensed
under the “\href{http://creativecommons.org/licenses/by-sa/4.0/}{Creative
Commons Attribution-ShareAlike 4.0 International License}”. Is that okay with
you?

Yes $\Box$ \hspace{2cm} No $\Box$

\newpage

\section{The complex scalar field}
\label{homework:1}

\subsection{Hamiltonian}

\paragraph{Conjugate momenta}

I have given the following action:
\begin{equation}
    S = \int \dif^4 x \, \sbr{
        \partial_\mu \phi^* \partial^\mu \phi - m^2 \phi^* \phi
    }.
\end{equation}
The scoping of the $\partial$ is not perfectly clear, I assume that it will
only act on the symbol directly after it. So that the first term in the action
density is the four-gradient squared. I will use GR notation for that:
\begin{equation}
    S = \int \dif^4 x \, \sbr{
        \phi^*_{,\mu} \phi^{,\mu} - m^2 \phi^* \phi
    }.
\end{equation}

Now I use
\begin{equation}
    S = \int \dif t \, L
    \eqnsep
    L = \int \dif^3 x \, \mathscr L
\end{equation}
where $\mathscr L$ is the Lagrange density and obtain the Lagrange function
$L$:
\begin{equation}
    L = \int \dif^3 x \, \sbr{
        \phi^*_{,\mu} \phi^{,\mu} - m^2 \phi^* \phi
    }.
\end{equation}

The conjugate momentum densities are then:
\begin{equation}
    \pi = \pd{L}{\dot \phi} = \dot \phi^{*}
    \eqnsep
    \pi^* = \pd{L}{\dot \phi^*} = \dot \phi.
\end{equation}

\paragraph{Commutation relations}

The commutation relations are chosen in a certain way to quantize the theory. I
do not think there is anything one can do to derive them right here. The
relations are, with $a$ and $b$ labelling multiple fields
\parencite[(12.3.1)]{Schwabl/Quantenmechanik_fuer_Fortgeschrittene/2008}:
\begin{gather*}
    \sbr{\phi_a(x, t), \pi_b(y, t)} = \iup \delta_{ab} \delta(x - y) \\
    \sbr{\phi_a(x, t), \phi_b(y, t)} = 0 \\
    \sbr{\pi_a(x, t), \pi_b(y, t)} = 0
\end{gather*}

Now one can insert the momenta and yield
\begin{gather*}
    \sbr{\phi(x, t), \dot\phi^*(y, t)} = \iup \delta_{ab} \delta(x - y) \\
    \sbr{\phi^*(x, t), \dot\phi(y, t)} = \iup \delta_{ab} \delta(x - y)
\end{gather*}
by also setting $\phi_a := \phi$ and $\phi_b := \phi^*$
\parencite[(13.2.4)]{Schwabl/Quantenmechanik_fuer_Fortgeschrittene/2008}.

\paragraph{Hamiltonian}

The Hamiltonian can then be obtained using a Legendre transformation. I have
not an explicit recipe for the complex valued case since
\textcite[Chapter~2]{Peskin/QFT/1995} only cover the real Klein-Gordon field
and \textcite[Chapter~13.2]{Schwabl/Quantenmechanik_fuer_Fortgeschrittene/2008}
does not give the Hamiltonian of the complex Klein-Gordon field. I figured out
a way that gives the correct result. Please tell me whether it is the correct
reasoning.

The Legendre transformation in classical mechanics is
\begin{equation}
    H = \pi_i \dot q^i - L.
\end{equation}
The $i$ denote the finite number of degrees of freedom. In a field theory, the
degrees of freedoms are infinite, and therefore one has $q_i \to q(x)$ and $x$
labels the degrees of freedom. Now $q(x)$ is the \emph{one} field in
consideration. There is a momentum $\pi(x)$ which also has the same degrees of
freedom. There still has to be a sum over all the degrees of freedom, which is
now done by integration over $x$.
\begin{equation}
    H = \int \dif x \, \pi(x) \dot q(x) - L.
\end{equation}

Here we have two fields, $\phi$ and $\phi^*$. Therefore, there are multiple
terms in the Hamiltonian. The equation looks really similar:
\begin{equation}
    H = \int \dif x \, \pi_a(x) \dot \phi^a(x) - L.
\end{equation}
Here, the index $a$ numbers the different fields ($a = 1, 2$). So the index
summation is over the different fields. The summing over $i$ is now an integral
over $x$.

With this, I can find the Hamiltonian:
\begin{align}
    H
    &= \int \dif^3 x \sbr{\pi \dot \phi + \pi^* \dot\phi^* - \mathscr L(\phi, \dif \phi(\phi,
    \pi))} \\
    \intertext{%
        Before I insert $\mathscr L$, I will replace the time derivative of the
        fields with the canonical momentum density. This will also make the
        first two summands equal, since the momenta commute.
    }
    &= \int \dif^3 x \sbr{2 \pi^* \pi - \mathscr L} \\
    \intertext{%
        Now I expand $\mathscr L$.
    }
    &= \int \dif^3 x \sbr{2 \pi^* \pi - \phi^*_{,\mu} \phi^{,\mu} +
    m^2 \phi^* \phi} \\
    \intertext{%
        Using the metric tensor $\eta^{ii} = -1$ one can separate time and
        spatial components:
    }
    &= \int \dif^3 x \sbr{2 \pi^* \pi - \sbr{\dot \phi^* \dot \phi - \vnabla \phi^* \cdot
    \vnabla \phi} + m^2 \phi^* \phi} \\
    \intertext{%
        The first term in the inner bracket is just $\pi^* \pi$ and cancels out
        part of the first summand. I am left with
    }
    &= \int \dif^3 x \sbr{ \pi^* \pi + \vnabla \phi^* \cdot \vnabla \phi + m^2 \phi^* \phi}.
\end{align}
This is exactly the Hamiltonian given in Equation~(7) on the problem set.

\paragraph{Heisenberg equations of motion}

Now that the Hamiltonian is derived, one can use the time evolution in the
Heisenberg picture to get the equations of motion for the two fields. This is
plugging in into
\(
    \iup \dot O = [H, O]
\)
for given operator $O$ \parencite[(2.44)]{Peskin/QFT/1995}. I can insert $O =
\phi$ and $O = \pi$ to get the Hamiltonian equations of motions. I will start
with the $\phi(\vec x, t)$.
\begin{align}
    \iup \dot\phi(\vec x, t)
    &= \sbr{\phi(\vec x, t), H} \\
    \intertext{%
        I insert the Hamiltonian. The integration has to be done over a
        different variable, I use $x'$. All the functions in the Hamiltonian
        have arguments $(\vec x', t)$ which I omitted to fit this onto a single
        line.
    }
    &= \sbr{\phi(\vec x, t),
        \int \dif^3 x' \sbr{ \pi^* \pi + \vnabla \phi^* \cdot \vnabla \phi + m^2 \phi^* \phi}
    } \\
    \intertext{%
        The integration of $\vec x'$ does not involve $\vec x$ at all, so I can
        use the linearity of the integral and move it out of the commutator. I
        now have three individual ones I can work with independently.
    }
    &= \int \dif^3 x' \sbr{
        \sbr{\phi(\vec x, t), \pi^* \pi}
        + \sbr{\phi(\vec x, t), \vnabla \phi^* \cdot \vnabla \phi}
        + m^2 \sbr{\phi(\vec x, t), \phi^* \phi}
    }
    \intertext{%
        (1) In the first term, $\phi$ commutes with $\pi^*$, so I can move this
        out of the commutator. (2) The $\phi$ commute with each other freely,
        so this commutator is just zero. (3) $\phi$ commutes with itself at any
        time, so this is also zero. What remains is
    }
    &= \int \dif^3 x' \, \pi^* \sbr{\phi(\vec x, t), \pi(\vec x', t)}.
    \intertext{%
        Using the commutation relations that I copied earlier, this can be
        written as
    }
    &= \int \dif^3 x' \, \pi^*(\vec x', t) \, \iup \, \deltaup(\vec x - \vec x')
    \intertext{%
        Now I perform the integration and obtain
    }
    &= \iup \pi^*(\vec x, t).
\end{align}
This looks similar to the treatment of the real Klein-Gordon field by
\textcite[25]{Peskin/QFT/1995}. The time evolution of the complex conjugate
$\phi^*$ is given by $\pi(\vec x, t)$ as this whole derivation can be complex
conjugated ($H^* = H$).

Now I need to perform an analogue calculation for $\dot\pi$:
\begin{align}
    \iup \dot\pi(\vec x, t)
    &= \sbr{\pi(\vec x, t), } \\
    &= \sbr{\pi(\vec x, t),
        \int \dif^3 x' \sbr{ \pi^* \pi + \vnabla \phi^* \cdot \vnabla \phi + m^2 \phi^* \phi}
    } \\
    \intertext{%
        Again I move the integration out of the commutator and yield three
        independent ones:
    }
    &= \int \dif^3 x' \sbr{
        \sbr{\pi(\vec x, t), \pi^* \pi}
        + \sbr{\pi(\vec x, t), \vnabla \phi^* \cdot \vnabla \phi}
        + m^2 \sbr{\pi(\vec x, t), \phi^* \phi}
    }
    \intertext{%
        The commutators are now different from the previous case with $\phi$.
        (1) This commutator is zero, since the momentum density commutes with
        itself at any point in time. (2) This will need partial integration to
        isolate $\phi$ which does not commute with $\pi$. (3) The last term is
        like the first term of the previous derivation.
    }
    &= \eval{\vnabla \phi^* \sbr{\pi(\vec x, t), \phi}}_{\vec x' = \partial
\R^3} + \int \dif^3 x' \sbr{
        - \laplace \phi^* \sbr{\pi(\vec x, t), \phi}
        + m^2 \phi^* \sbr{\pi(\vec x, t), \phi(\vec x', t)}
    }
    \intertext{%
        (1) The surface term in front will vanish on the boundary at infinity
        because everything is assumed to be normalizable and must therefore
        decay faster than $1/\sqrt r$. I will drop that term. (2) The
        commutator is the negative of the commutation relation given before.
        (3) That term is now the same as the second term and I will factor that
        out.
    }
    &= \int \dif^3 x' \sbr{\laplace \phi^* - m^2 \phi^*}
    \sbr{\phi(\vec x, t), \pi(\vec x, t)}
    \intertext{%
        That last remaining commutator gives a $\deltaup$-distribution again. I
        integrate over it to remove the $\vec x'$ and I obtain
    }
    &= [\laplace - m^2]\phi^*(\vec x, t).
\end{align}
This also looks similar to the results of \textcite[25]{Peskin/QFT/1995}.

\paragraph{Klein-Gordon equation}

Both of those results have to be combined to give the equation of motion for
the field $\phi$. Taking the time derivative of the first equation and
inserting the second yields
\begin{equation}
    \ddot \phi(\vec x, t) = [\laplace - m^2] \phi(\vec x, t).
\end{equation}
as well as the complex conjugate of this equation. This indeed is the
classical Klein-Gordon equation \parencite[(2.7)]{Peskin/QFT/1995} and the
quantum mechanical one \parencite[(2.45)]{Peskin/QFT/1995}.

\subsection{Fourier modes}

\paragraph{Expand into Fourier modes}

I will go along the lines of \textcite[§\,2.3]{Peskin/QFT/1995} here. The
expansion of $\phi$ and $\pi$ into Fourier modes is formally given by:
\begin{equation}
    \phi(\vec x, t) = \int \frac{\dif^3p}{[2\piup]^3} \exp(\iup \vec p \vec x)
    \phi(\vec p, t).
\end{equation}
This is just the back-transform from the momentum space into spatial space (is
that even a word?). The real task is to find a representation for $\phi(\vec p,
t)$. In momentum space, the Klein-Gordon equation looks just like the equation
of the harmonic oscillator \parencite[(2.21)]{Peskin/QFT/1995}. The field and
the momentum can then be written with ladder operators in analogy and the
harmonic oscillator
\parencite[(2.23)]{Peskin/QFT/1995}:
\begin{equation}
    \phi = \frac{1}{2 \omega} [a + a^\dagger]
    \eqnsep
    p = - \iup \sqrt{\frac{\omega}{2}} [a - a^\dagger]
    \eqnsep
    \omega := \sqrt{|\vec p|^2 + m^2}.
\end{equation}

\begin{question}
    Why can I do this? Is it because the Hamiltonian looks like the one from
    the harmonic oscillator and one now thinks about the Klein-Gordan field as
    a field made up of an infinite number of harmonic oscillators? Is that why
    \textcite[(2.25)]{Peskin/QFT/1995} then use $a_{\vec p}$ next, because that
    creates another harmonic oscillator with momentum $\vec p$?
\end{question}

I repeat the expansion of the field into Fourier modes:
\begin{align}
    \phi(\vec x, t)
    &= \int \frac{\dif^3p}{[2\piup]^3} \exp(\iup \vec p \vec x) \phi(\vec p, t).
    \intertext{%
        Now I insert the ladder operators. I have not understood why the last
        exponential has a minus sign in it. It looks like the second summand is
        the Hermitean conjugate of the first one. This makes $\phi$
        self-adjoint and therefore real. \textcite[§\,2.3]{Peskin/QFT/1995}
        cover the real Klein-Gordon field, so I am not sure whether this is
        correct here as well. According to
        \textcite[§\,13.2]{Schwabl/Quantenmechanik_fuer_Fortgeschrittene/2008}
        this is not correct here and I have to use independent operators for
        creation and annihilation. The complex conjugate field will then have
        the order reversed. I will use the notation and call the second set of
        ladder operators $b$. \textcite{Peskin/QFT/1995} in contrast to
        \textcite{Schwabl/Quantenmechanik_fuer_Fortgeschrittene/2008} have the
        minus sign at different positions as well. I will stick with the latter
        one since that really is for the relativistic Klein-Gordan field.
    }
    \phi(\vec x, t)
    &= \int \frac{\dif^3p}{[2\piup]^3}
    \frac{1}{\sqrt{2 \omega_{\vec p}}} \sbr{
        a_{\vec p} \exp(- \iup \vec p \vec x)
        + b_{\vec p}^\dagger \exp(\iup \vec p \vec x)
    }
    \intertext{%
        Since the integral is symmetric in $\vec p$, one can factor out the
        exponential by using $-\vec p$ in those parts.
    }
    \label{eq:fourier-phi}
    \phi(\vec x, t)
    &= \int \frac{\dif^3p}{[2\piup]^3}
    \frac{1}{\sqrt{2 \omega_{\vec p}}} \sbr{
        a_{-\vec p}
        + b_{\vec p}^\dagger
    } \exp(\iup \vec p \vec x)
    \intertext{%
        The complex conjugate, or rather the Hermitian conjugate of this
        expression can now be computed to give the other part.
    }
    \label{eq:fourier-phi-cc}
    \phi^*(\vec x, t)
    &= \int \frac{\dif^3p}{[2\piup]^3}
    \frac{1}{\sqrt{2 \omega_{\vec p}}} \sbr{
        b_{-\vec p}
        + a_{\vec p}^\dagger
    } \exp(\iup \vec p \vec x)
\end{align}

The same thing can then be done for the momentum densities as well:
\begin{align}
    \label{eq:fourier-pi}
    \pi(\vec x, t)
    &= - \iup \int \frac{\dif^3p}{[2\piup]^3} \sqrt{\frac{\omega_{\vec p}}{2}}
    \sbr{a_{- \vec p} - b_{\vec p}^\dagger} \exp(\iup \vec p \vec x) \\
    \label{eq:fourier-pi-cc}
    \pi^*(\vec x, t)
    &= - \iup \int \frac{\dif^3p}{[2\piup]^3} \sqrt{\frac{\omega_{\vec p}}{2}}
    \sbr{b_{- \vec p} - a_{\vec p}^\dagger} \exp(\iup \vec p \vec x)
\end{align}

\begin{small}
    The real number constant $\piup$ and the momentum density $\pi$ can be told
    apart since the former is in upright font whereas the latter is in italic
    font. This is done according to ISO standard 80000-2
    \parencite{iso_80000-2:2009}.
\end{small}

\paragraph{Hamilton operator in ladder operators}

\newcommand\Lad{a_{\vec p}^\dagger}
\newcommand\Lamd{a_{-\vec p}^\dagger}
\newcommand\Lam{a_{-\vec p}}
\newcommand\La{a_{\vec p}}
\newcommand\Lbd{b_{\vec p}^\dagger}
\newcommand\Lbmd{b_{-\vec p}^\dagger}
\newcommand\Lbm{b_{- \vec p}}
\newcommand\Lb{b_{\vec p}}

The next step is to take those four expressions derived here and build up the
Hamilton operator with them. The last version that I had for the Hamiltonian
was the following:
\begin{align}
    H
    &= \int \dif^3 x \sbr{\pi^* \pi + \vnabla \phi^* \cdot \vnabla \phi + m^2 \phi^* \phi}.
    \intertext{%
        Now I insert all the four expressions into there. It is crucial to
        choose different integration variables in the products since those are
        independent of each other. In the following I just replaced everything
        with the Fourier expansions, nothing is reordered yet.
    }
    \nonumber
    &= \int \dif^3 x \, \bigg[
    \\&\qquad
    \nonumber
    \iup \int \frac{\dif^3p}{[2\piup]^3} \sqrt{\frac{\omega_{\vec p}}{2}}
    \sbr{b_{- \vec p} - a_{\vec p}^\dagger} \exp(\iup \vec p \vec x)
    \iup \int \frac{\dif^3p'}{[2\piup]^3} \sqrt{\frac{\omega_{\vec p'}}{2}}
    \sbr{a_{- \vec p'} - b_{\vec p'}^\dagger} \exp(\iup \vec p' \vec x)
    \\&\qquad
    \nonumber
        + \vnabla
\int \frac{\dif^3p}{[2\piup]^3}
    \frac{1}{\sqrt{2 \omega_{\vec p}}} \sbr{
        b_{-\vec p}
        + a_{\vec p}^\dagger
    } \exp(\iup \vec p \vec x)
        \cdot
        \vnabla
    \int \frac{\dif^3p'}{[2\piup]^3}
    \frac{1}{\sqrt{2 \omega_{\vec p'}}} \sbr{
        a_{-\vec p'}
        + b_{\vec p'}^\dagger
    } \exp(\iup \vec p' \vec x)
    \\&\qquad
    \nonumber
        + m^2
    \int \frac{\dif^3p}{[2\piup]^3}
    \frac{1}{\sqrt{2 \omega_{\vec p}}} \sbr{
        b_{-\vec p}
        + a_{\vec p}^\dagger
    } \exp(\iup \vec p \vec x)
    \int \frac{\dif^3p'}{[2\piup]^3}
    \frac{1}{\sqrt{2 \omega_{\vec p'}}} \sbr{
        a_{-\vec p'}
        + b_{\vec p'}^\dagger
    } \exp(\iup \vec p' \vec x)
    \\&\quad
    \bigg]
    \intertext{%
        Now I can bring terms together and make the whole thing more compact.
    }
    \nonumber
    &= \int \dif^3 x \, \bigg[
    \\\nonumber&\qquad
    - \int \frac{\dif^3p \,\dif^3p'}{[2\piup]^6}
    \frac{\sqrt{\omega_{\vec p}\omega_{\vec p'}}}2
    \exp\del{\iup [\vec p+\vec p'] \vec x}
    \sbr{b_{- \vec p} - a_{\vec p}^\dagger}
    \sbr{a_{- \vec p'} - b_{\vec p'}^\dagger}
    \\\nonumber&\qquad
    +
    \int \frac{\dif^3p \, \dif^3p'}{[2\piup]^6}
    \frac{1}{2 \sqrt{\omega_{\vec p} \omega_{\vec p'}}}
    \sbr{
        b_{-\vec p}
        + a_{\vec p}^\dagger
    }
    \sbr{
        a_{-\vec p'}
        + b_{\vec p'}^\dagger
    }
    \vnabla
    \exp(\iup \vec p \vec x)
        \cdot
    \vnabla
    \exp(\iup \vec p' \vec x)
    \\\nonumber&\qquad
        + m^2
    \int \frac{\dif^3p \, \dif^3p'}{[2\piup]^6}
    \frac{1}{2 \sqrt{\omega_{\vec p} \omega_{\vec p'}}}
    \sbr{
        b_{-\vec p}
        + a_{\vec p}^\dagger
    }
    \sbr{
        a_{-\vec p'}
        + b_{\vec p'}^\dagger
    } \exp\del{\iup [\vec p + \vec p'] \vec x}
    \\&\quad
    \bigg]
    \intertext{%
        I can now let the $\vnabla$ act on the exponential to give me a $\vec
        p$ and $\vec p'$ respectively. The integration over the momentum is the
        same in each summand, so I can pull this out front as well. After the
        differentiation, the exponentials are the same in every term as well,
        so I gather those in the first line as well.
    }
    \nonumber
    &= \int \dif^3 x \,
    \frac{\dif^3p \,\dif^3p'}{[2\piup]^6} \,
    \exp\del{\iup [\vec p+\vec p'] \vec x}
    \bigg[
    \\\nonumber&\qquad
    -
    \frac{\sqrt{\omega_{\vec p}\omega_{\vec p'}}}2
    \sbr{b_{- \vec p} - a_{\vec p}^\dagger}
    \sbr{a_{- \vec p'} - b_{\vec p'}^\dagger}
    \\\nonumber&\qquad
    +
    \frac{1}{2 \sqrt{\omega_{\vec p} \omega_{\vec p'}}}
    [\iup \vec p \cdot \iup \vec p']
    \sbr{ b_{-\vec p} + a_{\vec p}^\dagger }
    \sbr{ a_{-\vec p'} + b_{\vec p'}^\dagger }
    \\\nonumber&\qquad
    + m^2
    \frac{1}{2 \sqrt{\omega_{\vec p} \omega_{\vec p'}}}
    \sbr{ b_{-\vec p} + a_{\vec p}^\dagger }
    \sbr{ a_{-\vec p'} + b_{\vec p'}^\dagger }
    \\&\quad
    \bigg]
    \intertext{%
        Now the second and third summand share a lot and I can factor that out
        again.
    }
    \nonumber
    &= \int \dif^3 x \,
    \frac{\dif^3p \,\dif^3p'}{[2\piup]^6} \,
    \exp\del{\iup [\vec p+\vec p'] \vec x}
    \bigg[
    \\\nonumber&\qquad
    -
    \frac{\sqrt{\omega_{\vec p}\omega_{\vec p'}}}2
    \sbr{b_{- \vec p} - a_{\vec p}^\dagger}
    \sbr{a_{- \vec p'} - b_{\vec p'}^\dagger}
    \\\nonumber&\qquad
    +
    \frac{- \vec p \vec p' + m^2}{2 \sqrt{\omega_{\vec p} \omega_{\vec p'}}}
    \sbr{ b_{-\vec p} + a_{\vec p}^\dagger }
    \sbr{ a_{-\vec p'} + b_{\vec p'}^\dagger }
    \\&\quad
    \bigg]
    \intertext{%
        Now that looks a lot like the formula by
        \textcite[(2.31)]{Peskin/QFT/1995}, just that I have a factor 2 more
        then they do. I am not sure how I or they got that, but in case
        something is off by a factor of two later on, this is where it started.
        Maybe it is the factor $1/2$ they have in their Lagrange density of the
        real Klein-Gordon field. There is only one function that depends on
        $\vec x$, and that is the exponential. That integration will give me a
        $\deltaup(\vec p + \vec p')$. I can then carry out the integration over
        $\vec p'$ to eliminate it by applying $\vec p' \to - \vec p$.
    }
    &= \int \frac{\dif^3p}{[2\piup]^3} \,
    \sbr{
        -
        \frac{\omega_{\vec p}}2
        \sbr{b_{- \vec p} - a_{\vec p}^\dagger}
        \sbr{a_{\vec p} - b_{-\vec p}^\dagger}
        +
        \frac{p^2 + m^2}{2 \omega_{\vec p}}
        \sbr{ b_{-\vec p} + a_{\vec p}^\dagger }
        \sbr{ a_{\vec p} + b_{-\vec p}^\dagger }
    }
    \intertext{%
        The definition was $\omega_{\vec p}^2 = p^2 + m^2$. The fraction will
        then simplify and both of the fractions can be factored out. I change
        the order of the two summands to get rid of that additional minus
        sign.
    }
    &= \int \frac{\dif^3p}{[2\piup]^3} \,
    \frac{\omega_{\vec p}}2
    \sbr{
        \sbr{\Lbm + \Lad}
        \sbr{\La + \Lbmd}
        -
        \sbr{\Lbm - \Lad}
        \sbr{\La - \Lbmd}
    }
    \intertext{%
        Since I do not see a shorter way, I expand all those products of the
        ladder operators. While doing that, I am careful not to commute
        anything.
    }
    &= \int \frac{\dif^3p}{[2\piup]^3} \,
    \frac{\omega_{\vec p}}2
    \sbr{
        \Lbm\La + \Lbm\Lbmd + \Lad\La + \Lad\Lbmd
        - \sbr{
            \Lbm\La - \Lbm \Lbmd - \Lad\La + \Lad\Lbmd
        }
    }
    \intertext{%
        I factor out the sign and see about the remaining ones.
    }
    &= \int \frac{\dif^3p}{[2\piup]^3} \, \omega_{\vec p}
    \sbr{ \Lbm\Lbmd + \Lad\La }
    \intertext{%
        I am not sure about this, but I think I can remove the minus sign
        taking the hermitian conjugate of the ladder operators $b$.
    }
    &= \int \frac{\dif^3p}{[2\piup]^3} \, \omega_{\vec p}
    \sbr{\Lad\La + \Lbd\Lb}
\end{align}
Now the whole thing is symmetric in $a$ and $b$ which is nice. There is no zero
point energy like the real Klein-Gordan field seems to have
\parencite[(2.31)]{Peskin/QFT/1995}.

\paragraph{Two sets of particles}

The Hamiltonian contains two sets of ladder operators, $a$ and $b$. Those
independently create particles with the same momentum $\vec p$, mass $m$ and
energy $\omega_{\vec p}$.

\subsection{Symmetry}

\paragraph{Global symmetry}

In all terms only $X^*X = |X|^2$ comes up, where $X = \phi, \dot\phi, \pi$. A
constant phase factor would cancel out in those and therefore not change the
action, Lagrangian or the Hamiltonian. The field and its derivatives and
therefore the coordinates would change. But the equations of motions would
retain their form.

This can also be shown explicitly using Noether's theorem. The infinitesimal
version of the transformation is given by
\begin{equation}
    \phi \to \tilde\phi = \phi + \iup \alpha \phi + \mathrm O(\alpha^2).
\end{equation}
The Lagrangian density then transforms like this:
\begin{align}
    \mathscr L
    &= \phi^*_{,\mu} \phi^{,\mu} - m^2 \phi^* \phi
    \intertext{%
        Now I replace $\phi$ with $\tilde\phi$.
    }
    &= [1+\iup\alpha] \tilde\phi^*_{,\mu} [1 - \iup\alpha] \tilde\phi^{,\mu} -
    m^2 [1 + \iup\alpha] \phi^* [1 - \iup\alpha] \phi + \mathrm O(\alpha^2) \\
    &= [1+\alpha^2] \tilde\phi^*_{,\mu}  \tilde\phi^{,\mu} -
    m^2 [1 + \alpha^2] \phi^* \phi + \mathrm O(\alpha^2)
    \intertext{%
        The $\alpha^2$ terms can be put into the Landau symbol like this:
    }
    &= \tilde\phi^*_{,\mu}  \tilde\phi^{,\mu} - m^2 \phi^* \phi + \mathrm O(\alpha^2)
\end{align}
The Lagrangian density has not changed in first order, just like argued in the
in the first paragraph.

\paragraph{Conserved quantity}

Using this, I can compute the conserved current. Since the Lagrangian density
is unchanged, $\mathscr J = 0$ (that is a script $J$). The conserved current
$j$ then is:
\begin{equation}
    j^\mu =
    \pd{\mathscr L}{\phi_{,\mu}} \iup \phi
    - \pd{\mathscr L}{\phi_{,\mu}^*} \iup \phi^*
    = \iup
    \sbr{
        \pd{\mathscr L}{\phi_{,\mu}} \phi
        - \pd{\mathscr L}{\phi_{,\mu}^*} \phi^*
    }
\end{equation}
There are two terms because I have to independent fields that I have to take
account. The minus comes from the complex conjugation of the $\iup$.

\needspace{5cm}
\begin{question}
    Is $j$ or $\mathcal J$ the Noether current?
\end{question}

\paragraph{Conserved charge}

The zeroth component of the conserved current density, $j^0$ is the charge
density. The spatial integral over this charge is the total charge:
\begin{equation}
    Q = \int \dif^3 x \, j^0.
\end{equation}
Here the charge density is given by:
\begin{equation}
    j^\mu
    = \iup
    \sbr{
        \pd{\mathscr L}{\dot\phi} \phi
        - \pd{\mathscr L}{\dot\phi^*} \phi^*
    }
    = \iup \sbr{ \pi \phi - \pi^* \phi^* }
\end{equation}
From this, the conserved charge is
\begin{equation}
    \label{eq:mein-Q}
    Q = \iup \int \dif^3 x \sbr{ \pi \phi - \pi^* \phi^* },
\end{equation}
which differs by a factor of $-2$ from Equation~(8) on the problem set. Also
the order of $\pi$ and $\phi$ is different. Those two do not commute, both
would give a $\iup\deltaup(x)$, except that the complex conjugates would give
a $-\iup\deltaup(x)$. Those do not cancel.

\needspace{5cm}
\begin{question}
    How do I derive the conserved charge given in Equation~(8) on the problem
    set?
\end{question}

\subsection{Conserved charge}

For some reason my conserved charge, Equation~\eqref{eq:mein-Q}, differs from
the one given on the problem set. I will continue to use the one from the
problem set. The conserved charge given is
\begin{align}
    Q
    &= \int \dif^3 x \, \frac\iup2 \sbr{\phi^* \pi^* - \pi\phi}
    \intertext{%
        Now I can insert the expansions of the field and momentum density in
        terms of ladder operators from Equations~\eqref{eq:fourier-phi},
        \eqref{eq:fourier-phi-cc}, \eqref{eq:fourier-pi} and
        \eqref{eq:fourier-pi-cc}.
    }
    \nonumber
    &= \frac12 \int \dif^3 x \,
    \int \frac{\dif^3p \, \dif^3 p'}{[2\piup]^6}
    \sqrt{\frac{\omega_{\vec p'}}{\omega_{\vec p}}}
    \exp\del{\iup [\vec p + \vec p'] \vec x} \\
    %
    &\qquad \times
    \sbr{
        \sbr{b_{-\vec p} + a_{\vec p}^\dagger}
        \sbr{b_{-\vec p'} - a_{\vec p'}^\dagger}
        -
        \sbr{a_{-\vec p'} - b_{\vec p'}^\dagger}
        \sbr{a_{-\vec p} + b_{\vec p}^\dagger}
    }
    \intertext{%
        The integration over $\vec x$ again yields a $\deltaup(\vec p + \vec
        p')$-distribution that sets $\vec p' := - \vec p$.
    }
    &= \frac12
    \int \frac{\dif^3p}{[2\piup]^3}
    \sbr{
        \sbr{b_{-\vec p} + a_{\vec p}^\dagger}
        \sbr{b_{\vec p} - a_{-\vec p}^\dagger}
        -
        \sbr{a_{\vec p} - b_{-\vec p}^\dagger}
        \sbr{a_{-\vec p} + b_{\vec p}^\dagger}
    }
    \intertext{%
        Now I have to compute the products of all that again, just like in the
        derivation of the Hamiltonian.
    }
    &= \frac12
    \int \frac{\dif^3p}{[2\piup]^3}
    \sbr{
        \Lbm \Lb - \Lbm \Lamd + \Lad \Lb - \Lad \Lamd
        -
        \sbr{
            \La \Lam + \La \Lbd - \Lbmd \Lam - \Lbmd \Lbd
        }
    } \\
    \intertext{%
        I factor out the minus sign.
    }
    &= \frac12
    \int \frac{\dif^3p}{[2\piup]^3}
    \sbr{
        \Lbm \Lb - \Lbm \Lamd + \Lad \Lb - \Lad \Lamd
        -
        \La \Lam - \La \Lbd + \Lbmd \Lam + \Lbmd \Lbd
    }
    \intertext{%
        I sort the terms and commute $a$ with $b$.
    }
    &= \frac12
    \int \frac{\dif^3p}{[2\piup]^3}
    \sbr{
        - \La \Lam
        - \La \Lbd
        - \Lad \Lamd
        + \Lad \Lb
        + \Lam \Lbmd
        - \Lamd \Lbm
        + \Lbmd \Lbd
        + \Lbm \Lb
    }
    \intertext{%
        The bounds of the integration is symmetric, so I can move the minus
        sign in the $\vec p$ to the last term or remove it completely.
    }
    &= \frac12
    \int \frac{\dif^3p}{[2\piup]^3}
    \sbr{
        - \La \Lam
        - \Lad \Lamd
        + \Lad \Lb
        - \Lad \Lb
        + \La \Lbd
        - \La \Lbd
        + \Lbd \Lbmd
        + \Lb \Lbm
    }
    \intertext{%
        The mixed term vanish now.
    }
    &= \frac12
    \int \frac{\dif^3p}{[2\piup]^3}
    \sbr{
        - \La \Lam
        - \Lad \Lamd
        + \Lbd \Lbmd
        + \Lb \Lbm
    }
    \intertext{%
        Now there is nothing I directly see to go on. However, if $a_{- \vec p}
        = a_{\vec p}^\dagger$, then I could further simplify it. It makes sense
        somewhat: Creating a particle with opposite wavenumber is the same as
        removing one particle with the actual wavenumber. So I can do this?
    }
    &= \frac12
    \int \frac{\dif^3p}{[2\piup]^3}
    \sbr{
        - \La \Lad
        - \Lad \La
        + \Lbd \Lb
        + \Lb \Lbd
    } \\
    \intertext{%
        I shifted the minus sign in the momentum from one operator to the other
        with the argument of the symmetric integration bounds. If I had shifted
        them in another way around before, I would now be done and have
    }
    &= \int \frac{\dif^3p}{[2\piup]^3}
    \sbr{ - n_{a \vec p} + n_{b \vec p} }.
\end{align}
$n$ is the occupation number operator $a^\dagger a$. That means that the
particles $a$ have charge $-1$ and the particles $b$ have charge $1$. I could
have chosen the signs differently and arrived at the charges the other way
around as well.

\subsection{Two fields}

\paragraph{Symmetries of two fields}

So far I have been dealing with a single (yet complex) field and a symmetry
operation which just had one generator. Both is going to change now. The
Lagrangian is now composed of parts for both fields $\phi_a$ with $a = 1, 2$:
\begin{equation}
    \mathscr L = \sum_a \mathscr L_a
    = \phi_{a,\mu}^* \phi^{a,\mu} - m^2 \phi_a^* \phi^a.
\end{equation}

%\needspace{5cm}
\begin{question}
    Now $\vec \phi$ seems to be a two-vector here since it consists of two
    fields. Is \emph{that} a spinor?
\end{question}

If I look at the fields individually, I only have to symmetry operation with
parameter $\alpha$. Once I thought about $\vec\phi$ as a vector, it came
natural to look at transformation matrices:
\begin{equation}
    \tilde{\vec\phi} = \tens A \vec\phi.
\end{equation}
The Lagrangian density now looks like this:
\begin{align}
    \mathscr L
    &= \sbr{\tens A\inv\tilde{\vec\phi}_{,\mu}}^\dagger \sbr{\tens A\inv
    \tilde{\vec\phi}^{,\mu}} - m^2 \sbr{\tens A\inv \tilde{\vec\phi}}^\dagger
    \sbr{\tens A\inv \tilde{\vec\phi}}.
    \intertext{%
        Now I can apply the hermitian conjugate to the square brackets. In the
        next step it will become clear that $\tens A$ has to be unitary. I will
        therefore say that the symmetry transformations have to be SU(2)
        matrices. Then the adjoint of the inverse is the matrix itself.
    }
    &= \tilde{\vec\phi}_{,\mu}^\dagger \tens A \tens A\inv
    \tilde{\vec\phi}^{,\mu} - m^2 \tilde{\vec\phi}^\dagger \tens A
    \tens A\inv \tilde{\vec\phi}
    \intertext{%
        The matrices cancel out and one is left with the unaltered Lagrangian
        density.
    }
    &= \tilde{\vec\phi}_{,\mu}^\dagger \tilde{\vec\phi}^{,\mu} - m^2
    \tilde{\vec\phi}^\dagger \tilde{\vec\phi}
\end{align}
Note that the $\vec\phi$ are field vectors (in bold) and that the contraction
is implied.

I have now found that any SU(2) matrix transforms the field vector in a way
that leaves the equations of motion invariant. There is also the U(1) symmetry
that is generated by $\exp(\iup \alpha) \tens 1_2$ where $\tens 1_2$ is the
unit matrix in two dimensions. From here I can go to Noether's theorem and
compute the conserved quantities.

There are four generators $T_\lambda$ here, one from U(1) and three from SU(3).
The generators are in the Physicist's convention of hermitian generators:
\begin{equation}
    T_0 = \tens 1_2
    \eqnsep
    T_i = \sigma_i
\end{equation}
According to \textcite{wikipedia/Pauli-Matrizen} it is customary to include the
unit matrix $\tens 1_2$ as a zeroth Pauli matrix. I do see why from this.
$\tens 1_2$ is the one and only generator of U(1), but expressed in two
dimensions. This is a reducible representation of U(1), but I want to operate
on two-dimensional quantities, so I need a representation in this number of
dimensions. And so I just take the simple generator 1 twice and have $\tens 1_2
= \tens 1_1 \oplus \tens 1_1$ as a generator in two dimensions.

\paragraph{Conserved quantities}

Using the generators I can now express the four basic unitary matrices that
give me symmetry transformations as
\begin{equation}
    \tens A = \exp\del{\iup \alpha^\lambda T_\lambda}
\end{equation}
where $\vec\alpha$ is now a parameter four-vector. I can then compute the
Noether charge densities from that:
\begin{align}
    j_\lambda^0
    &= \pd{\mathscr L}{\dot \phi^a} \iup \sbr{\tens \sigma_\lambda \cdot
    \vec\phi}^a + \hc,
    \intertext{%
        where $\hc$ stands for the hermitian conjugate of all the previous
        terms in the same scope. The “$\cdot$” stands for a single contraction
        of indices. I can now compute the derivative and replace it by the
        canonical momentum density.
    }
    &= \pi_a \iup \sbr{\tens \sigma_\lambda \cdot \vec\phi}^a + \hc.
    \intertext{%
        Then I can write this a whole vector-matrix-vector expression.
    }
    &= \iup \vec\pi \cdot \tens \sigma_\lambda \cdot \vec\phi + \hc
\end{align}
The conserved charges are now given by:
\begin{equation}
    Q_\lambda = 
    \iup \int \dif^3x \, \vec\pi \cdot \tens \sigma_\lambda \cdot \vec\phi + \hc
\end{equation}
This again differs by a factor $-2$ from the result given on the problem set.
The change in ordering of $\pi$ and $\phi$ is accounted by the hermitian
conjugation instead of the plain complex conjugation.

\paragraph{Commutation relations}

The commutation relations of the $Q_i$ follow now. I already use the linearity
of the integral to make it a bit shorter.
\begin{align}
    [Q^i, Q^j]
    &= \iup \int \dif^3x \,
    [\vec\pi \cdot \tens \sigma^i \cdot \vec\phi, \vec\pi \cdot \tens \sigma^j \cdot \vec\phi] + \hc
    \intertext{%
        I have to expand this with indices again to be able to commute some of
        the parts. All the field indices go to the bottom to make it a bit
        easier on the eye. Summation convention is still applicable.
    }
    &= \iup \int \dif^3x \,
    [\pi_a \sigma^i_{ab} \phi_b, \pi_c \sigma^j_{cd} \phi_d] + \hc
    \intertext{%
        Now the Pauli matrices are just numbers and I can commute them with
        $\pi$ and $\phi$. I cannot commute the latter with each other since the
        canonical commutation relations apply for those. But I do not need to
        commute those anyway.
    }
    &= \iup \int \dif^3x \,
    [\pi_a \phi_b \sigma^i_{ab}, \sigma^j_{cd} \pi_c \phi_d] + \hc
    \intertext{%
        The parts apart of the Pauli matrices can be pulled out the commutator
        since they were going to commute with those matrices anyway. This
        leaves me with the commutator of the Pauli matrices themselves.
    }
    &= \iup \int \dif^3x \,
    \pi_a \phi_b [\sigma^i_{ab}, \sigma^j_{cd}] \pi_c \phi_d + \hc
\end{align}
Therefore, the $Q_i$ commute just like the $\sigma_i$.

\end{document}

% vim: spell spelllang=en tw=79
