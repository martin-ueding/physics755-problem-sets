\documentclass[11pt, english, fleqn, DIV=15, headinclude, BCOR=1cm]{scrartcl}

\usepackage[bibatend]{../header}
\usepackage{../my-boxes}

\usepackage{booktabs}

\hypersetup{
    pdftitle=
}

\newcounter{totalpoints}
\newcommand\punkte[1]{#1\addtocounter{totalpoints}{#1}}

\newcounter{problemset}
\setcounter{problemset}{5}

\subject{physics755 -- Quantum Field Theory}
\ihead{physics755 -- Problem Set \arabic{problemset}}

\title{Problem Set \arabic{problemset}}

\newcommand\thegroup{Group Tuesday -- Ripunjay Acharya}

\publishers{\thegroup}
\ofoot{\thegroup}

\author{
    Martin Ueding \\ \small{\href{mailto:mu@martin-ueding.de}{mu@martin-ueding.de}}
    \and
    Oleg Hamm
}
\ifoot{Martin Ueding, Oleg Hamm}

\ohead{\rightmark}

\begin{document}

\maketitle

\vspace{3ex}

\begin{center}
    \begin{tabular}{rrr}
        problem & achieved points & possible points \\
        \midrule
        \nameref{homework:1} & & \punkte{7} \\
        \nameref{homework:2} & & \punkte{8} \\
        \midrule
        total & & \arabic{totalpoints}
    \end{tabular}
\end{center}

\section{The Dirac representation}
\label{homework:1}

\subsection{Transformation of Weyl spinors}

We had transformations with $\vec L$ and $\vec K$, ones with $\vec J_+$ and
$\vec J_-$ and lastly ones with $\vec \sigma$. So we look at those new
$\tens\sigma^{\mu\nu}$. First the time component:
\begin{align*}
    \sigma^{00} &= 0 \\
    \intertext{%
        The mixed components:
    }
    \sigma^{0i}
    &= \frac\iup4 \sbr{\sigma^0 \bar\sigma^i - \sigma^i \bar\sigma^0} \\
    &= \frac\iup4 \sbr{- \sigma^i - \sigma^i} \\
    &= - \frac\iup2 \sigma^i \\
    \bar\sigma^{0i} &= - \frac\iup2 \bar\sigma^i \\
    \intertext{%
        The spatial components:
    }
    \sigma^{ij}
    &= \frac\iup4 \sbr{\sigma^i \bar\sigma^j - \sigma^i \bar\sigma^j} \\
    &= - \frac\iup4 \sbr{\sigma^i \sigma^j + \sigma^i \sigma^j} \\
    &= - \frac\iup4 \sbr{\sigma^i, \sigma^j} \\
    &= \epsilonup_{ijk} \sigma^k \\
    \bar\sigma^{ij} &= \sigma^{ij}
\end{align*}

Using those identities, we can rewrite the transformation that we previously
had for left handed spinors.
\begin{align*}
    -\iup \vec\Theta \cdot \frac{\vec \sigma}{2} - \vec\beta \cdot \frac{\vec \sigma}{2}
    &= - \iup \Theta_i \frac{\sigma^i}{2} - \beta_i \frac{\sigma^i}{2} \\
    &= - \frac\iup4 \sbr{\Theta_i \epsilonup_{ijk} \sigma^{jk} + \beta_i
    \sigma^{0i}} \\
    \intertext{%
        Now one defined the antisymmetric angle tensor $\tens\omega$ such that
        $\omega_{ij} = \Theta_i \epsilon_{ijk}$ and $\omega_{0i} = \beta_i$.
        Then we can write
    }
    &= - \frac\iup4 \omega_{\mu\nu} \sigma^{\mu\nu},
\end{align*}
which is the desired result for the left handed spinors. For the left handed
ones, there was an additional minus in front of $\vec\beta$, such that we need
to use $\bar\sigma$ there.

\section{Classical solutions}
\label{homework:2}

\end{document}

% vim: spell spelllang=en tw=79
