\documentclass[11pt, english, fleqn, DIV=15, headinclude, BCOR=1cm]{scrartcl}

\usepackage[bibatend]{../header}
\usepackage{../my-boxes}

\usepackage{booktabs}
\usepackage{slashed}

\hypersetup{
    pdftitle=
}

\newcounter{totalpoints}
\newcommand\punkte[1]{#1\addtocounter{totalpoints}{#1}}

\newcounter{problemset}
\setcounter{problemset}{5}

\subject{physics755 -- Quantum Field Theory}
\ihead{physics755 -- Problem Set \arabic{problemset}}

\title{Problem Set \arabic{problemset}}

\newcommand\thegroup{Group Tuesday -- Ripunjay Acharya}

\publishers{\thegroup}
\ofoot{\thegroup}

\author{
    Martin Ueding \\ \small{\href{mailto:mu@martin-ueding.de}{mu@martin-ueding.de}}
    \and
    Oleg Hamm
}
\ifoot{Martin Ueding, Oleg Hamm}

\ohead{\rightmark}

\begin{document}

\maketitle

\vspace{3ex}

\begin{center}
    \begin{tabular}{rrr}
        problem & achieved points & possible points \\
        \midrule
        \nameref{homework:1} & & \punkte{7} \\
        \nameref{homework:2} & & \punkte{8} \\
        \midrule
        total & & \arabic{totalpoints}
    \end{tabular}
\end{center}

\section{The Dirac representation}
\label{homework:1}

\subsection{Transformation of Weyl spinors}

We had transformations with $\vec L$ and $\vec K$, ones with $\vec J_+$ and
$\vec J_-$ and lastly ones with $\vec \sigma$. So we look at those new
$\tens\sigma^{\mu\nu}$. First the time component:
\begin{align*}
    \sigma^{00} &= 0 \\
    \intertext{%
        The mixed components:
    }
    \sigma^{0i}
    &= \frac\iup4 \sbr{\sigma^0 \bar\sigma^i - \sigma^i \bar\sigma^0} \\
    &= \frac\iup4 \sbr{- \sigma^i - \sigma^i} \\
    &= - \frac\iup2 \sigma^i \\
    \bar\sigma^{0i} &= - \frac\iup2 \bar\sigma^i \\
    \intertext{%
        The spatial components:
    }
    \sigma^{ij}
    &= \frac\iup4 \sbr{\sigma^i \bar\sigma^j - \sigma^i \bar\sigma^j} \\
    &= - \frac\iup4 \sbr{\sigma^i \sigma^j + \sigma^i \sigma^j} \\
    &= - \frac\iup4 \sbr{\sigma^i, \sigma^j} \\
    &= \epsilonup_{ijk} \sigma^k \\
    \bar\sigma^{ij} &= \sigma^{ij}
\end{align*}
These seem correct since \textcite[(3.26)f]{Peskin/QFT/1995} have the same
expressions.

Using those identities, we can rewrite the transformation that we previously
had for left handed spinors.
\begin{align*}
    -\iup \vec\Theta \cdot \frac{\vec \sigma}{2} - \vec\beta \cdot \frac{\vec \sigma}{2}
    &= - \iup \Theta_i \frac{\sigma^i}{2} - \beta_i \frac{\sigma^i}{2} \\
    &= - \frac\iup4 \sbr{\Theta_i \epsilonup_{ijk} \sigma^{jk} + \beta_i
    \sigma^{0i}} \\
    \intertext{%
        Now one defined the antisymmetric angle tensor $\tens\omega$ such that
        $\omega_{ij} = \Theta_i \epsilon_{ijk}$ and $\omega_{0i} = \beta_i$.
        Then we can write
    }
    &= - \frac\iup4 \omega_{\mu\nu} \sigma^{\mu\nu},
\end{align*}
which is the desired result for the left handed spinors. For the left handed
ones, there was an additional minus in front of $\vec\beta$, such that we need
to use $\bar\sigma$ there.

\subsection{Dirac spinor transformation}

We compute the $\tens S$ explicitly, just as done on the lecture on Friday.
\begin{align*}
    S^{\mu\nu}
    &= \frac\iup4 \sbr{\gamma^\mu, \gamma^\nu} \\
    &= \frac\iup4 \sbr{\gamma^\mu \gamma^\nu - \gamma^\nu \gamma^\mu} \\
    &= \frac\iup4 \begin{pmatrix}
    \sigma^\mu \bar\sigma^\nu - \sigma^\nu \bar\sigma^\mu & 0 \\
    0 & \bar\sigma^\mu \sigma^\nu - \bar\sigma^\nu \sigma^\mu & 0
    \end{pmatrix} \\
    &= \diag\del{\sigma^{\mu\nu}, \bar\sigma^{\mu\nu}}
\end{align*}

So we see that this commutator of the Dirac matrices in the chiral
representation give a block diagonal matrix. In the exponential map, this will
still be a block diagonal matrix. As we have shown in the previous part of this
problem, the left and right handed parts of the Dirac spinor transform
independently, just like the block diagonal form that we have here.

\subsection{Commutator}

We want to use anticommutation relations to yield $\tens\eta$s which are needed
for the final result. So we start with the given commutator.
\begin{align*}
    \sbr{\tens\gamma^\mu, \tens S^{\rho\sigma}}
    &= \frac\iup4 \sbr{\tens\gamma^\mu, \tens\gamma^\rho \tens\gamma^\sigma + \tens\gamma^\sigma
    \tens\gamma^\rho} \\
    &= \frac\iup4 \sbr{
        \tens\gamma^\mu \tens\gamma^\rho \tens\gamma^\sigma
        + \tens\gamma^\mu \tens\gamma^\sigma \tens\gamma^\rho
        - \tens\gamma^\rho \tens\gamma^\sigma \tens\gamma^\mu
        - \tens\gamma^\sigma \tens\gamma^\rho \tens\gamma^\mu
    }
    \intertext{%
        We now use the anticommutation relation of the Dirac matrices.
    }
    &= \frac\iup4 \sbr{
        \sbr{- \tens\gamma^\rho \tens\gamma^\mu + \eta^{\mu\rho} \tens 1_4} \tens\gamma^\sigma
        + \sbr{- \tens\gamma^\sigma \tens\gamma^\mu + \eta^{\mu\sigma} \tens 1_4} \tens\gamma^\rho
        - \tens\gamma^\rho \tens\gamma^\sigma \tens\gamma^\mu
        - \tens\gamma^\sigma \tens\gamma^\rho \tens\gamma^\mu
    }
    \intertext{%
        Then we expand the inner brackets.
    }
    &= \frac\iup4 \sbr{
        - \tens\gamma^\rho \tens\gamma^\mu \tens\gamma^\sigma
        - \tens\gamma^\sigma \tens\gamma^\mu \tens\gamma^\rho
        - \tens\gamma^\rho \tens\gamma^\sigma \tens\gamma^\mu
        - \tens\gamma^\sigma \tens\gamma^\rho \tens\gamma^\mu
        + \eta^{\mu\rho} \tens 1_4 \tens\gamma^\sigma
        + \eta^{\mu\sigma} \tens 1_4 \tens\gamma^\rho
    }
    \intertext{%
        We do the same thing again to move the $\tens\gamma^\mu$ to the back.
    }
    &= \frac\iup4 \left[
        - \tens\gamma^\rho \sbr{- \tens\gamma^\sigma \tens\gamma^\mu + \eta^{\mu\sigma} \tens 1_4}
        - \tens\gamma^\sigma \sbr{- \tens\gamma^\rho \tens\gamma^\mu + \eta^{\mu\rho} \tens 1_4}
        - \tens\gamma^\rho \tens\gamma^\sigma \tens\gamma^\mu
        - \tens\gamma^\sigma \tens\gamma^\rho \tens\gamma^\mu
        \right. \\ &\qquad \left.
        + \eta^{\mu\rho} \tens 1_4 \tens\gamma^\sigma
        + \eta^{\mu\sigma} \tens 1_4 \tens\gamma^\rho
    \right]
    \intertext{%
        And we factor out again.
    }
    &= \frac\iup4 \left[
        \tens\gamma^\rho \tens\gamma^\sigma \tens\gamma^\mu
        + \tens\gamma^\sigma \tens\gamma^\rho \tens\gamma^\mu
        - \tens\gamma^\rho \tens\gamma^\sigma \tens\gamma^\mu
        - \tens\gamma^\sigma \tens\gamma^\rho \tens\gamma^\mu
        \right. \\ &\qquad \left.
        + \eta^{\mu\rho} \tens 1_4 \tens\gamma^\sigma
        + \eta^{\mu\sigma} \tens 1_4 \tens\gamma^\rho
        - \tens\gamma^\rho \eta^{\mu\sigma} \tens 1_4
        - \tens\gamma^\sigma \eta^{\mu\rho} \tens 1_4
    \right]
    \intertext{%
        The first four terms cancel.
    }
    &= \frac\iup4 \sbr{
        \eta^{\mu\rho} \tens 1_4 \tens\gamma^\sigma
        + \eta^{\mu\sigma} \tens 1_4 \tens\gamma^\rho
        - \tens\gamma^\rho \eta^{\mu\sigma} \tens 1_4
        - \tens\gamma^\sigma \eta^{\mu\rho} \tens 1_4
    }
    \intertext{%
        The components of the matrix tensor are just numbers and commute with
        the matrices.
    }
    &= \frac\iup4 \sbr{
        \eta^{\mu\rho} \tens\gamma^\sigma
        + \eta^{\mu\sigma} \tens\gamma^\rho
        - \eta^{\mu\sigma} \tens\gamma^\rho
        - \eta^{\mu\rho} \tens\gamma^\sigma
    }
\end{align*}
This is just zero, but it should not be zero.

% TODO Do you have anything better?

\subsection{Lorentz invariance of Dirac's equation}

This problem is exactly covered by \textcite[42]{Peskin/QFT/1995}.

Even though $\slashed \partial = \gamma^\mu \partial_\mu$ \emph{looks like} it
would transform like a Lorentz scalar, we cannot assume that. We need to
transform the individual parts and then look at the whole thing. The Dirac
equation is given by
\[
    \sbr{\iup \gamma^\mu \partial_\mu - m} \psi(x) = 0.
\]
Now we transform it passively by adding transformations.
\begin{align*}
    \sbr{\iup \gamma^\mu \Lambda\inv{}^\nu{}_\mu \partial_\nu - m} \Lambda_\text D \psi(\Lambda\inv x)
    \intertext{%
        We premuliply with $\Lambda_\text D \Lambda_\text D\inv$.
    }
    &= \Lambda_\text D \Lambda_\text D\inv \sbr{\iup \gamma^\mu \Lambda\inv{}^\nu{}_\mu \partial_\nu - m} \Lambda_\text D \psi(\Lambda\inv x)
    \intertext{%
        Since $m$ is just a number, it commutes with the transformations. We
        can pull some transformations into the bracket, the $\Lambda_\text D
        \inv \Lambda_\text D$ will just cancel around the $m$.
    }
    &= \Lambda_\text D \sbr{\Lambda_\text D\inv \iup \gamma^\mu \Lambda\inv{}^\nu{}_\mu \partial_\nu \Lambda_\text D - m} \psi(\Lambda\inv x)
    \intertext{%
        The matrix $\Lambda_\text D$ depends on the angles and boosts, but not
        on spacetime. It therefore commutes with the partial derivatives.
    }
    &= \Lambda_\text D \sbr{\Lambda_\text D\inv \iup \gamma^\mu \Lambda_\text D \Lambda\inv{}^\nu{}_\mu \partial_\nu - m} \psi(\Lambda\inv x)
    \intertext{%
        Now we can use the identity that is given on the problem set.
    }
    &= \Lambda_\text D \sbr{\Lambda^\mu_\rho \iup \gamma^\rho \Lambda\inv{}^\nu{}_\mu \partial_\nu - m} \psi(\Lambda\inv x)
    \intertext{%
        There are now the normal and inverse transformation, they contract to a
        Kronecker symbol.
    }
    &= \Lambda_\text D \sbr{\iup \deltaup^\nu_\rho \gamma^\rho \partial_\nu - m} \psi(\Lambda\inv x)
    \intertext{%
        And we are back to a Dirac equation:
    }
    &= \Lambda_\text D \sbr{\iup \slashed\partial - m} \psi(\Lambda\inv x)
\end{align*}
Now the transformation acts on the whole Dirac equation, which then is form
invariant.

\subsection{Klein-Gordon equation}

This brings us to \textcite[43]{Peskin/QFT/1995}.

The Dirac equation is the square root of the Klein-Gordon equation in
the sense of the used Clifford algebra. Hamilton had shown this using
quaternions around 1840 \parencite[619]{penrose-road_to_reality}. To get back
to the Klein-Gordon equation, one has to \emph{square} the equation in the
appropriate sense. The correct way of doing this is to take the modulus squared
of the Dirac operator:
\begin{align*}
    \abs{\iup \slashed\partial - m}^2 \psi
    &= \sbr{- \iup \slashed\partial - m} \sbr{\iup \slashed\partial - m} \psi
    \\
    \intertext{%
        There are only two terms that contribute. One could also use the third
        binomial formula here.
    }
    &= \sbr{\slashed\partial\slashed\partial + m^2} \psi \\
    \intertext{%
        This is in a form that is so compact that one cannot see how to simply
        this. We undo the “slashing“.
    }
    &= \sbr{\gamma^\mu \gamma^\nu \partial_\mu \partial_\nu + m^2} \psi \\
    \intertext{%
        Since the partial derivatives of a differentiable function commute
        (theorem of Schwarz), only the symmetric part of the Dirac matrix
        tensor product contributes. Since we use the non-idempotent form of the
        symmetrization and antisymmetrization notation now, we need to
        explicitly introduce a factor of $1/2$.
    }
    &= \sbr{\frac 12 \gamma^{(\mu} \gamma^{\nu)} \partial_\mu \partial_\nu + m^2} \psi \\
    \intertext{%
        Since we take the symmetric part of a tensor times itself, $\tens
        \gamma \otimes \tens \gamma$, we can also use the anticommutator.
    }
    &= \sbr{\frac 12 \sbr{\gamma^\mu, \gamma^\nu}_+ \partial_\mu \partial_\nu + m^2} \psi \\
    \intertext{%
        The anticommutator is known and we get
    }
    &= \sbr{\eta^{\mu\nu} \partial_\mu \partial_\nu + m^2} \psi \\
    &= \sbr{\dalambert + m^2} \psi.
\end{align*}
This is the Klein-Gordon equation.

\subsection{Lagrange density}

The transformation works like this:
\begin{align*}
    \bar\phi
    &= \phi^\dagger \gamma^0 \\
    &\mapsto [\Lambda \phi]^\dagger \gamma^0 \\
    &= \phi^\dagger \Lambda^\dagger \gamma^0 \\
    \intertext{%
        The transformation is given by this:
    }
    &= \phi^\dagger \exp\del{\frac\iup2 \omega_{\mu\nu} S^{\mu\nu}}^\dagger
    \gamma^0 \\
    &= \phi^\dagger \exp\del{- \frac\iup2 \omega_{\mu\nu} S^{\mu\nu\dagger}}
    \gamma^0. \\
    \intertext{%
        The matrices $S^{ij}$ commute with $\gamma^0$ since these $S$ consist
        of the product of two Dirac matrices. The constituents of said $S$
        anticommute with $\gamma^0$ individually, and two anticommutations mean
        a commutation in total. Since $S^{ij}$ are generators of a compact
        subgroup in the physicist's convention they are hermitian. The
        generators $S^{0i}$ are anti-hermitian but anticommute with $\gamma^0$.
        $\gamma^0$ commutes with itself, which apparently is not trivial, see
        \url{http://physics.stackexchange.com/q/139325/5705}. So the
        anticommutation together with the antihermitian property gives that it
        commutes and is hermitian for this particular case. We can therefore
        commute it and remove the hermitian conjugate.
    }
    &= \phi^\dagger \gamma^0 \exp\del{- \frac\iup2 \omega_{\mu\nu} S^{\mu\nu}}
    \intertext{%
        We can now identify the transformation as an \emph{inverse}
        transformation.
    }
    &= \bar\phi \Lambda\inv_\text D
\end{align*}

We can compute the equation of motion for $\phi$ easily using the
Euler-Lagrange equation for it. We have $\mathscr L = \bar\psi [ \iup
\slashed\partial - m] \psi$. Then the Euler-Lagrange equation is
\begin{align*}
    \pd{\mathscr L}{\bar\phi} - \partial_\mu \pd{\mathscr L}{\bar\phi_{,\mu}}
    &= 0 \\
    [\iup \slashed\partial - m] \psi
    &= 0.
\end{align*}
That is the Dirac equation. The derivation can also be performed for $\psi$,
which will give the hermitian conjugate of the Dirac equation. Since we will do
that in the next problem anyway, we will not typeset that here.

\subsection{Hermitian conjugate}

$\gamma^0$ is a real, symmetric and therefore hermitian matrix. It commutes
with itself and the relation is therefore trivially fulfilled. The $\gamma^i$
in the chiral representation are antihermitian matrices that anticommute with
$\gamma^0$. So in total, all the minus signs match and we use that the Dirac
matrices are their own inverses.

Using this, we can get the equation of motion for $\bar\psi$ from the one for
$\psi$:
\begin{align*}
    [\iup \slashed\partial - m] \psi &= 0 \\
    \iup \gamma^\mu \psi_{,\mu} - m \psi &= 0 \\
    \intertext{%
        We take the hermitian conjugate.
    }
    - \iup \psi_{,\mu}^\dagger \gamma^{\mu\dagger} - m \psi^\dagger &= 0 \\
    \intertext{%
        Now we can use the identity that we have derived earlier.
    }
    - \iup \psi_{,\mu}^\dagger \gamma^0 \gamma^\mu \gamma^0 - m \psi^\dagger &= 0 \\
    \intertext{%
        We identity the first spinor.
    }
    - \iup \bar\psi_{,\mu} \gamma^\mu \gamma^0 - m \psi^\dagger &= 0 \\
    \intertext{%
        Now the postmultiply with $\gamma^0$ and use that its square is the
        identity.
    }
    - \iup \bar\psi_{,\mu} \gamma^\mu - m \bar\psi &= 0 \\
    \intertext{%
        We can now multiply with $-1$ and bring it into the compact form
    }
    [\iup \slashed\partial + m] \bar\phi &= 0.
\end{align*}

\section{Classical solutions}
\label{homework:2}

\subsection{Constraints}

Using the free solutions we have:
\begin{align*}
    [\iup \slashed\partial - m] \psi(x) &= 0 \\
    [\iup \slashed\partial - m] u(p) \exp(- \iup p \cdot x) &= 0 \\
    [\slashed p - m] u(p) \exp(- \iup p \cdot x) &= 0 \\
    [\slashed p - m] u(p) &= 0
\end{align*}

We look at $p = (m, 0, 0, 0)$ and insert this. Then we have:
\begin{align*}
    m \sbr{\gamma^0 - \tens 1_4} u(p) &= 0 \\
    m \begin{pmatrix}
        - \tens 1_2 & \tens 1_2 \\ \tens 1_2 & - \tens 1_2
    \end{pmatrix} u(p) &= 0 \\
    m \begin{pmatrix}
        - \tens 1_2 & \tens 1_2 \\ \tens 1_2 & - \tens 1_2
    \end{pmatrix} \begin{pmatrix}
        \vec \xi \\ \vec \xi
    \end{pmatrix} &= 0
\end{align*}
And the constraint is fulfilled.

\subsection{Boost}

We boost with rapidity $\eta$. Then we have
\[
    \Lambda p = \begin{pmatrix}
        \cosh(\eta) m \\ 0 \\ 0 \\ \sinh(\eta) m
    \end{pmatrix}
    \overset != \begin{pmatrix}
        E \\ 0 \\ 0 \\ p_3
    \end{pmatrix}
\]
The norm of those has to be the same, so either way we get $E^2 - p_3^2 = m^2$
which is not a surprise because a classic theory should be on-shell. We also
get the following equations:
\[
    \cosh(\rho) m = E
    \eqnsep
    \sinh(\rho) m = p_3.
\]
We can write the trigonometric functions in terms of exponentials.
\[
    \sbr{\exp(\rho) + \exp(-\rho)} \frac m2 = E
    \eqnsep
     \sbr{\exp(\rho) - \exp(-\rho)}m = p_3.
\]
Now we can solve this system of equations for the exponentials.
\[
    \exp(\rho) m = E + p_3
    \eqnsep
    \exp(-\rho) m = E - p_3
\]
Taking the positive branch of the square root gives us the following.
\[
    \exp\del{\frac\rho2} \sqrt m = \sqrt{E + p_3}
    \eqnsep
    \exp\del{-\frac\rho2} \sqrt m = \sqrt{E - p_3}
\]

The transformation of the Dirac spinor $\psi$ can be written as
\[
    \Lambda_\text D = \exp\del{- \frac\eta2 \begin{pmatrix}
        \sigma^3 & 0 \\ 0 & -\sigma^3
    \end{pmatrix}}
\]
since we already derived the transformations of the left and right handed parts
and have shown that they can be combined into a block diagonal form.

We will now go along the lines of \textcite[46]{Peskin/QFT/1995}. The
boosted spinor is then given by:
\begin{align*}
    u(p)
    &= \exp\del{- \frac\eta2 \begin{pmatrix} \sigma^3 & 0 \\ 0 & -\sigma^3
    \end{pmatrix}} u(p)
    \intertext{%
        Since the square of the Pauli matrices are always the identity matrix,
        this exponential will split up into two terms, one containing the
        identity and one containing the Pauli matrix. The factors in front will
        give cosh and sinh.
    }
    &= \sbr{
        \cosh\del{\frac\eta2}
        \begin{pmatrix} \tens 1_2 & 0 \\ 0 & \tens 1_2 \end{pmatrix}
        +
        \sinh\del{\frac\eta2}
        \begin{pmatrix} \sigma^3 & 0 \\ 0 & - \sigma^3 \end{pmatrix}
    } u(p)
    \intertext{%
        As the next step one has to write cosh and sinh in terms of
        exponentials to combine the two summands.
    }
    &=
    \begin{pmatrix}
        \exp\del{\frac\eta2} \frac{1-\sigma^3}{2} + \exp\del{-\frac\eta2} \frac{1+\sigma^3}{2}
        & 0 \\ 0 &
        \exp\del{\frac\eta2} \frac{1+\sigma^3}{2} + \exp\del{-\frac\eta2} \frac{1-\sigma^3}{2}
    \end{pmatrix} u(p)
    \intertext{%
        We insert the ansatz.
    }
    &=
    \begin{pmatrix}
        \exp\del{\frac\eta2} \frac{1-\sigma^3}{2} + \exp\del{-\frac\eta2} \frac{1+\sigma^3}{2}
        & 0 \\ 0 &
        \exp\del{\frac\eta2} \frac{1+\sigma^3}{2} + \exp\del{-\frac\eta2} \frac{1-\sigma^3}{2}
    \end{pmatrix} \sqrt m \begin{pmatrix}
        \vec \xi \\ \vec \xi
    \end{pmatrix}
    \intertext{%
        We can now use the equations that we derived for the exponentials
        earlier on.
    }
    &=
    \begin{pmatrix}
        \sqrt{E + p_3} \frac{1-\sigma^3}{2} + \sqrt{E - p_3} \frac{1+\sigma^3}{2}
        & 0 \\ 0 &
        \sqrt{E + p_3} \frac{1+\sigma^3}{2} + \sqrt{E - p_3} \frac{1-\sigma^3}{2}
    \end{pmatrix} \begin{pmatrix}
        \vec \xi \\ \vec \xi
    \end{pmatrix}
    \intertext{%
        This matrix multiplication is trivial, so we can just write it as a
        vector.
    }
    &=
    \begin{pmatrix}
        \sbr{\sqrt{E + p_3} \frac{1-\sigma^3}{2} + \sqrt{E - p_3}
            \frac{1+\sigma^3}{2}} \vec \xi
            \\
            \sbr{\sqrt{E + p_3} \frac{1+\sigma^3}{2} + \sqrt{E - p_3}
            \frac{1-\sigma^3}{2}} \vec\xi
    \end{pmatrix}
\end{align*}

\subsection{Simplification}

The simplification is also the next step by \textcite[46]{Peskin/QFT/1995},
they just do not give any intermediate results. So here they are:
\begin{align*}
    &\sqrt{E + p_3} \frac{\tens 1_2 - \sigma^3}{2} + \sqrt{E - p_3} \frac{\tens
    1_2 + \sigma^3}{2} \\
    &= \sqrt{[E + p_3] \frac{[\tens 1_2 - \sigma^3]^2}{4}} + \sqrt{[E - p_3] \frac{
    [\tens 1_2 + \sigma^3]^2}{4}} \\
    \intertext{%
        Computation shows that the fraction is a projection operator.
    }
    &= \sqrt{[E + p_3] \frac{\tens 1_2 - \sigma^3}{2}} + \sqrt{[E - p_3] \frac{\tens
    1_2 + \sigma^3}{2}} \\
    \intertext{%
        In order to make it hard to read, the now do nothing, really.
    }
    &= \sqrt{\sbr{
    \sqrt{[E + p_3] \frac{\tens 1_2 - \sigma^3}{2}}
    + \sqrt{[E - p_3] \frac{\tens 1_2 + \sigma^3}{2}}
    }^2} \\
    \intertext{%
        We can now apply the first binomial formula.
    }
    &= \sqrt{
    [E + p_3] \frac{\tens 1_2 - \sigma^3}{2}
    + [E - p_3] \frac{\tens 1_2 + \sigma^3}{2}
    + 2 \sqrt{[E + p_3] \frac{\tens 1_2 - \sigma^3}{2}}
    \sqrt{[E - p_3] \frac{\tens 1_2 + \sigma^3}{2}}
    } \\
    \intertext{%
        The last summand does not contribute anything. This can be seen either
        by seeing that the fractions are two orthogonal projection operators or
        by actually calculating it using the third binomial theorem, which will
        then yield $\tens 1_2 - \tens 1_2 = 0$.
    }
    &= \sqrt{
    [E + p_3] \frac{\tens 1_2 - \sigma^3}{2}
    + [E - p_3] \frac{\tens 1_2 + \sigma^3}{2}
    } \\
    \intertext{%
        Then we can compute all of the eight terms, see that most cancel and
        obtain just the following.
    }
    &= \sqrt{E \tens 1_2 - p_3 \sigma^3} \\
    \intertext{%
        Using $p_0 = E$ we can write this as
    }
    &= \sqrt{p_0 \sigma^0 - p_3 \sigma^3}. \\
    \intertext{%
        For our $p$, this can also be written as
    }
    &= \sqrt{p_\mu \sigma^\mu}
\end{align*}
We write the contraction with indices such that it is clear that the zeroth
component is included as well.

The second part has minus signs at the spatial components, so it will come out
with $\bar\sigma$ instead of $\sigma$.

\subsection{Lorentz invariance}

The first relation.
\begin{align*}
    [p \cdot \sigma][p \cdot \bar\sigma]
    &= p_\mu \sigma^\mu p_\nu \sigma^\nu \\
    &= p_\mu p_\nu \sigma^\mu \sigma^\nu \\
    \intertext{%
        Since the $p$ are symmetric, only the symmetric part of the tensor
        product of the Pauli matrices contributes. We can write this with the
        anticommutator.
    }
    &= \frac12 p_\mu p_\nu [\sigma^\mu, \sigma^\nu]_+ \\
    &= p_\mu p_\nu \eta^{\mu\nu} \\
    &= p^2
\end{align*}

Then we have:
\begin{align*}
    u^\dagger u
    &= \xi^\dagger p \cdot \sigma \xi + \xi^\dagger p \cdot \bar\sigma \xi \\
    &= 2 \xi^\dagger E_p \xi \\
    &= 2 E_p \xi^\dagger \xi
\end{align*}

Using a parity makes this invariant:
\begin{align*}
    \bar u u
    &= 2 \xi^\dagger \sqrt{p \cdot \sigma} \sqrt{p \cdot \bar\sigma} \xi \\
    \intertext{%
        And as shown previously, this is just the (rest) mass.
    }
    &= 2 m \xi^\dagger \xi
\end{align*}

\subsection{Changes for negative frequency}

We now have the following constraint:
\begin{align*}
    [\iup \slashed\partial - m] \psi(x) &= 0 \\
    [\iup \slashed\partial - m] u(p) \exp(\iup p \cdot x) &= 0 \\
    [- \slashed p - m] u(p) \exp(\iup p \cdot x) &= 0 \\
    [- \slashed p - m] u(p) &= 0
\end{align*}
So the sign in front of the momentum has changed. So there should be a minus
sign in front of the second component, just as if $\gamma^0$ had been applied,
since the parity connects the left and right handed components.

% TODO Is that legit?

\subsection{Completeness relation}

This one is rather easy to show.
\begin{align*}
    u^s \bar u_s
    &=
    \begin{pmatrix}
        \sqrt{p \cdot \sigma} \xi^s \\ \sqrt{p \cdot \bar\sigma} \xi^s
    \end{pmatrix}
    \begin{pmatrix}
        \xi_s^\dagger \sqrt{p \cdot \sigma} \xi^s & \xi_s^\dagger \sqrt{p \cdot \bar\sigma} 
    \end{pmatrix}
    \intertext{%
        We perform the tensor product and use the identities that we have
        previously computed.
    }
    &=
    \begin{pmatrix}
        m \tens 1_2 & p \cdot \sigma \\
        p \cdot \bar\sigma & m \tens 1_2
    \end{pmatrix}
    \intertext{%
        This can be expanded like so:
    }
    &=
    \begin{pmatrix}
        0 & p_0 \sigma^0 \\
        p_0 \bar\sigma^0 & 0
    \end{pmatrix}
    +
    \begin{pmatrix}
        0 & p_i \sigma^i \\
        p_i \bar\sigma^i & 0
    \end{pmatrix}
    +
    \begin{pmatrix}
        \tens 1_2 & 0 \\
        0 & \tens 1_2
    \end{pmatrix} m
    \intertext{%
        And then we can put in $\bar\sigma$ in terms of $\sigma$.
    }
    &= p_0
    \begin{pmatrix}
        0 & \sigma^0 \\
        \sigma^0 & 0
    \end{pmatrix}
    +
    p_i
    \begin{pmatrix}
        0 & \sigma^i \\
        - \sigma^i & 0
    \end{pmatrix}
    +
    \begin{pmatrix}
        \tens 1_2 & 0 \\
        0 & \tens 1_2
    \end{pmatrix} m
    \intertext{%
        Look! The Dirac matrices in the chiral representation.
    }
    &= p_\mu \gamma^\mu + m \tens 1_4
\end{align*}

\subsection{Negative frequency}

The negative frequency case must the orthogonal to the positive frequency one.
One choice would be \parencite[(3.62)]{Peskin/QFT/1995},
\[
    v^s(p) =
    \begin{pmatrix}
        \sqrt{p \cdot \sigma} \eta^s \\
        - \sqrt{p \cdot \bar\sigma} \eta^s \\
    \end{pmatrix}.
\]

\end{document}

% vim: spell spelllang=en tw=79
