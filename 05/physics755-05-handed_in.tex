\documentclass[11pt, english, fleqn, DIV=15, headinclude, BCOR=1cm]{scrartcl}

\usepackage[bibatend]{../header}
\usepackage{../my-boxes}

\usepackage{booktabs}

\hypersetup{
    pdftitle=
}

\newcounter{totalpoints}
\newcommand\punkte[1]{#1\addtocounter{totalpoints}{#1}}

\newcounter{problemset}
\setcounter{problemset}{5}

\subject{physics755 -- Quantum Field Theory}
\ihead{physics755 -- Problem Set \arabic{problemset}}

\title{Problem Set \arabic{problemset}}

\newcommand\thegroup{Group Tuesday -- Ripunjay Acharya}

\publishers{\thegroup}
\ofoot{\thegroup}

\author{
    Martin Ueding \\ \small{\href{mailto:mu@martin-ueding.de}{mu@martin-ueding.de}}
    \and
    Oleg Hamm
}
\ifoot{Martin Ueding, Oleg Hamm}

\ohead{\rightmark}

\begin{document}

\maketitle

\vspace{3ex}

\begin{center}
    \begin{tabular}{rrr}
        problem & achieved points & possible points \\
        \midrule
        \nameref{homework:1} & & \punkte{7} \\
        \nameref{homework:2} & & \punkte{8} \\
        \midrule
        total & & \arabic{totalpoints}
    \end{tabular}
\end{center}

\section{The Dirac representation}
\label{homework:1}

\subsection{Transformation of Weyl spinors}

We had transformations with $\vec L$ and $\vec K$, ones with $\vec J_+$ and
$\vec J_-$ and lastly ones with $\vec \sigma$. So we look at those new
$\tens\sigma^{\mu\nu}$. First the time component:
\begin{align*}
    \sigma^{00} &= 0 \\
    \intertext{%
        The mixed components:
    }
    \sigma^{0i}
    &= \frac\iup4 \sbr{\sigma^0 \bar\sigma^i - \sigma^i \bar\sigma^0} \\
    &= \frac\iup4 \sbr{- \sigma^i - \sigma^i} \\
    &= - \frac\iup2 \sigma^i \\
    \bar\sigma^{0i} &= - \frac\iup2 \bar\sigma^i \\
    \intertext{%
        The spatial components:
    }
    \sigma^{ij}
    &= \frac\iup4 \sbr{\sigma^i \bar\sigma^j - \sigma^i \bar\sigma^j} \\
    &= - \frac\iup4 \sbr{\sigma^i \sigma^j + \sigma^i \sigma^j} \\
    &= - \frac\iup4 \sbr{\sigma^i, \sigma^j} \\
    &= \epsilonup_{ijk} \sigma^k \\
    \bar\sigma^{ij} &= \sigma^{ij}
\end{align*}

Using those identities, we can rewrite the transformation that we previously
had for left handed spinors.
\begin{align*}
    -\iup \vec\Theta \cdot \frac{\vec \sigma}{2} - \vec\beta \cdot \frac{\vec \sigma}{2}
    &= - \iup \Theta_i \frac{\sigma^i}{2} - \beta_i \frac{\sigma^i}{2} \\
    &= - \frac\iup4 \sbr{\Theta_i \epsilonup_{ijk} \sigma^{jk} + \beta_i
    \sigma^{0i}} \\
    \intertext{%
        Now one defined the antisymmetric angle tensor $\tens\omega$ such that
        $\omega_{ij} = \Theta_i \epsilon_{ijk}$ and $\omega_{0i} = \beta_i$.
        Then we can write
    }
    &= - \frac\iup4 \omega_{\mu\nu} \sigma^{\mu\nu},
\end{align*}
which is the desired result for the left handed spinors. For the left handed
ones, there was an additional minus in front of $\vec\beta$, such that we need
to use $\bar\sigma$ there.

\subsection{Dirac spinor transformation}

We compute the $\tens S$ explicitly, just as done on the lecture on Friday.
\begin{align*}
    S^{\mu\nu}
    &= \frac\iup4 \sbr{\gamma^\mu, \gamma^\nu} \\
    &= \frac\iup4 \sbr{\gamma^\mu \gamma^\nu - \gamma^\nu \gamma^\mu} \\
    &= \frac\iup4 \begin{pmatrix}
    \sigma^\mu \bar\sigma^\nu - \sigma^\nu \bar\sigma^\mu & 0 \\
    0 & \bar\sigma^\mu \sigma^\nu - \bar\sigma^\nu \sigma^\mu & 0
    \end{pmatrix} \\
    &= \diag\del{\sigma^{\mu\nu}, \bar\sigma^{\mu\nu}}
\end{align*}

So we see that this commutator of the Dirac matrices in the chiral
representation give a block diagonal matrix. In the exponential map, this will
still be a block diagonal matrix. As we have shown in the previous part of this
problem, the left and right handed parts of the Dirac spinor transform
independently, just like the block diagonal form that we have here.

\subsection{Commutator}

We want to use anticommutation relations to yield $\tens\eta$s which are needed
for the final result. So we start with the given commutator.
\begin{align*}
    \sbr{\tens\gamma^\mu, \tens S^{\rho\sigma}}
    &= \frac\iup4 \sbr{\tens\gamma^\mu, \tens\gamma^\rho \tens\gamma^\sigma + \tens\gamma^\sigma
    \tens\gamma^\rho} \\
    &= \frac\iup4 \sbr{
        \tens\gamma^\mu \tens\gamma^\rho \tens\gamma^\sigma
        + \tens\gamma^\mu \tens\gamma^\sigma \tens\gamma^\rho
        - \tens\gamma^\rho \tens\gamma^\sigma \tens\gamma^\mu
        - \tens\gamma^\sigma \tens\gamma^\rho \tens\gamma^\mu
    }
    \intertext{%
        We now use the anticommutation relation of the Dirac matrices.
    }
    &= \frac\iup4 \sbr{
        \sbr{- \tens\gamma^\rho \tens\gamma^\mu + \eta^{\mu\rho} \tens 1_4} \tens\gamma^\sigma
        + \sbr{- \tens\gamma^\sigma \tens\gamma^\mu + \eta^{\mu\sigma} \tens 1_4} \tens\gamma^\rho
        - \tens\gamma^\rho \tens\gamma^\sigma \tens\gamma^\mu
        - \tens\gamma^\sigma \tens\gamma^\rho \tens\gamma^\mu
    }
    \intertext{%
        Then we expand the inner brackets.
    }
    &= \frac\iup4 \sbr{
        - \tens\gamma^\rho \tens\gamma^\mu \tens\gamma^\sigma
        - \tens\gamma^\sigma \tens\gamma^\mu \tens\gamma^\rho
        - \tens\gamma^\rho \tens\gamma^\sigma \tens\gamma^\mu
        - \tens\gamma^\sigma \tens\gamma^\rho \tens\gamma^\mu
        + \eta^{\mu\rho} \tens 1_4 \tens\gamma^\sigma
        + \eta^{\mu\sigma} \tens 1_4 \tens\gamma^\rho
    }
    \intertext{%
        We do the same thing again to move the $\tens\gamma^\mu$ to the back.
    }
    &= \frac\iup4 \left[
        - \tens\gamma^\rho \sbr{- \tens\gamma^\sigma \tens\gamma^\mu + \eta^{\mu\sigma} \tens 1_4}
        - \tens\gamma^\sigma \sbr{- \tens\gamma^\rho \tens\gamma^\mu + \eta^{\mu\rho} \tens 1_4}
        - \tens\gamma^\rho \tens\gamma^\sigma \tens\gamma^\mu
        - \tens\gamma^\sigma \tens\gamma^\rho \tens\gamma^\mu
        \right. \\ &\qquad \left.
        + \eta^{\mu\rho} \tens 1_4 \tens\gamma^\sigma
        + \eta^{\mu\sigma} \tens 1_4 \tens\gamma^\rho
    \right]
    \intertext{%
        And we factor out again.
    }
    &= \frac\iup4 \left[
        \tens\gamma^\rho \tens\gamma^\sigma \tens\gamma^\mu
        + \tens\gamma^\sigma \tens\gamma^\rho \tens\gamma^\mu
        - \tens\gamma^\rho \tens\gamma^\sigma \tens\gamma^\mu
        - \tens\gamma^\sigma \tens\gamma^\rho \tens\gamma^\mu
        \right. \\ &\qquad \left.
        + \eta^{\mu\rho} \tens 1_4 \tens\gamma^\sigma
        + \eta^{\mu\sigma} \tens 1_4 \tens\gamma^\rho
        - \tens\gamma^\rho \eta^{\mu\sigma} \tens 1_4
        - \tens\gamma^\sigma \eta^{\mu\rho} \tens 1_4
    \right]
    \intertext{%
        The first four terms cancel.
    }
    &= \frac\iup4 \sbr{
        \eta^{\mu\rho} \tens 1_4 \tens\gamma^\sigma
        + \eta^{\mu\sigma} \tens 1_4 \tens\gamma^\rho
        - \tens\gamma^\rho \eta^{\mu\sigma} \tens 1_4
        - \tens\gamma^\sigma \eta^{\mu\rho} \tens 1_4
    }
    \intertext{%
        The components of the matrix tensor are just numbers and commute with
        the matrices.
    }
    &= \frac\iup4 \sbr{
        \eta^{\mu\rho} \tens\gamma^\sigma
        + \eta^{\mu\sigma} \tens\gamma^\rho
        - \eta^{\mu\sigma} \tens\gamma^\rho
        - \eta^{\mu\rho} \tens\gamma^\sigma
    }
\end{align*}
This is just zero, but it should not be zero.

\section{Classical solutions}
\label{homework:2}

\end{document}

% vim: spell spelllang=en tw=79
