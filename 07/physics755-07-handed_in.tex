\documentclass[11pt, english, fleqn, DIV=15, headinclude, BCOR=1cm]{scrartcl}

\usepackage[bibatend]{../header}
\usepackage{../my-boxes}

\usepackage{booktabs}
\usepackage{slashed}

\hypersetup{
    pdftitle=
}

\newcounter{totalpoints}
\newcommand\punkte[1]{#1\addtocounter{totalpoints}{#1}}

\newcounter{problemset}
\setcounter{problemset}{7}

\subject{physics755 -- Quantum Field Theory}
\ihead{physics755 -- Problem Set \arabic{problemset}}

\title{Problem Set \arabic{problemset}}

\newcommand\thegroup{Group Tuesday -- Ripunjay Acharya}

\publishers{\thegroup}
\ofoot{\thegroup}

\author{
    Martin Ueding \\ \small{\href{mailto:mu@martin-ueding.de}{mu@martin-ueding.de}}
    \and
    Oleg Hamm
}
\ifoot{Martin Ueding, Oleg Hamm}

\ohead{\rightmark}

\begin{document}

\maketitle

\vspace{3ex}

\begin{center}
    \begin{tabular}{rrr}
        problem & achieved points & possible points \\
        \midrule
        \nameref{homework:1} & & \punkte{10} \\
        \nameref{homework:2} & & \punkte{5} \\
        \midrule
        total & & \arabic{totalpoints}
    \end{tabular}
\end{center}

\section{$PCT$ and all that}
\label{homework:1}

\subsection{Table}

Here we have to compute $4 \cdot 6 = 24$ transformation properties. We get
whopping four points for this problem. So this means that we have to write
about four pages to get everything done.

\paragraph{Parity}

Since there are six tensorial quantities, we will just number them for each
transformation operator.

\begin{enumerate}
    \item 
        To transform with the parity, we just sandwich the expression into $P$s
        and use that $P^2$ is the identity since it comes from the cyclic group
        $\Z_2$. We added two independent $\Z_2$ to the universal covering group
        we get from the exponential map of the Lie algebra $\so(1, 3)$ to reach
        the whole Lorentz group.
        \begin{align*}
            \bar\psi(t, \vec x) \psi(t, \vec x)
            &\mapsto P \bar\psi(t, \vec x) \psi(t, \vec x) P \\
            \intertext{%
                We introduce a $P^2 = 1$ in the middle.
            }
            &= P \bar\psi(t, \vec x) P P \psi(t, \vec x) P \\
            \intertext{%
                Now we can use the transformation rule given in Equation~(2) on
                the problem set. The phase $\eta$ here is not the metric tensor
                $\eta$ which shows up in the table given on the problem set. It
                neither is the rapidity $\eta$ which was used on a different
                sheet. This is ambiguous, but luckily, there is always context
                around, to make it clear. Right?
            }
            &= \eta^* \bar\psi(t, -\vec x) \mat\gamma^0 \eta \mat\gamma^0 \psi(t, -\vec x) \\
            \intertext{%
                The phase $\eta$ is a complex number and therefore commutes
                with the matrices.
            }
            &= \eta^* \eta \bar\psi(t, -\vec x) \mat\gamma^0 \mat\gamma^0 \psi(t, -\vec x) \\
            \intertext{%
                The modulus of the phase is unity, so we can drop this. The
                Dirac matrices are their own inverses, so we can drop those as
                well. We are left with the Lorentz scalar as the transformed
                point.
            }
            &= \bar\psi(t, -\vec x) \psi(t, -\vec x)
            \intertext{%
                To make it more apparent that we transformed the whole thing to
                one new coordinate, we can write it like this:
            }
            &= [\bar\psi \psi](t, -\vec x).
        \end{align*}
        Then the additional factor that we get is just $+1$. If we keep up this
        pace we will end up with a quarter of a point per page, so we need to
        pick up the pace a bit.

    \item
        The next one, $\iup \bar\psi \mat\gamma^5 \psi$, has an imaginary unit
        in it, as well as a $\mat\gamma^5$. The parity operator will give us
        $\mat\gamma^0$s, which anticommute with the $\mat\gamma^5$, so that
        will give us the minus sign.

    \item
        The bilinear $\bar\psi \mat\gamma^\mu \mat\gamma^5 \psi$ will transform
        differently depending on the $\mu$. If $\mu = 0$, it commutes with the
        $\mat\gamma^0$ that we get from the parity transformation. If $\mu \neq
        0$, we have to perform an anticommutation. Hence a factor of
        $\eta^{\mu\nu}$.

    \item
        Adding another $\mat\gamma^5$ to the bilinear means another
        anticommutation compared to the previous one and therefore a factor of
        $- \eta^{\mu\mu}$.

    \item
        We expand the transformed bilinear:
        \[
            \bar\psi \mat\gamma^0 \mat\sigma^{\mu\nu} \mat\gamma^0 \psi
            = \frac\iup2 \bar\psi \mat\gamma^0 [\mat\gamma^\mu, \mat\gamma^\nu] \mat\gamma^0 \psi
            = \frac\iup2 \bar\psi \sbr{
                \mat\gamma^0 \mat\gamma^\mu \mat\gamma^\nu \mat\gamma^0
                - \mat\gamma^0 \mat\gamma^\nu \mat\gamma^\mu \mat\gamma^0
        } \psi.
        \]
        One can see that it takes two anticommutations to get the
        $\mat\gamma^0$ to cancel the other one. Those anticommutations
        introduce factors of $\eta^{\mu\mu}$ and $\eta^{\nu\nu}$ respectively.
        Therefore the factor that we end up with is the product of the two.

    \item
        We apply the chain rule here and since we map $\vec x \mapsto - \vec x$
        the chain rule will just give us a minus sign in the spatial
        components. This is expressed in $\eta^{\mu\mu}$.
\end{enumerate}

\paragraph{Time reversal}

We have to do the same thing for the time reversal operation.

\begin{enumerate}
    \item
        We do the first one more explicit as before. We start with the sandwich
        between the time reversal operators.
        \begin{align*}
            \bar\psi \psi
            &\mapsto T \bar\psi \psi T
            \intertext{%
                We add a unit operator in between the field operators.
            }
            &= T \bar\psi T^\dagger T \psi T
            \intertext{%
                To use the formula that is given in Equation~(5) we must assume
                that $T^\dagger = T$. Or are we really using that $T^2$ must be
                the identity and not introduce a $T^\dagger$ at all? Either
                way, we must write
            }
            &= T \bar\psi T T \psi T.
            \intertext{%
                Now we apply Equation~(5) from the problem set.
            }
            &= - \bar\psi(-t, \vec x) \mat\gamma^1 \mat\gamma^3 \mat\gamma^1
            \mat\gamma^3 \psi(-t, \vec x)
            \intertext{%
                It takes one anticommutation in the middle.
            }
            &= \bar\psi(-t, \vec x) \mat\gamma^3 \mat\gamma^1 \mat\gamma^1
            \mat\gamma^3 \psi(-t, \vec x)
            \intertext{%
                We can collapse the Dirac matrices and have a compact result.
            }
            &= [\bar\psi \psi](-t, \vec x)
        \end{align*}

    \item
        The pseudo scalar has another $\mat\gamma^5$ which means another
        anticommutation is needed. This makes the factor $-1$.

    \item
        The vector is also a bit tricky, so we need several steps here
        as well.
        \begin{align*}
            \bar\psi \mat\gamma^5 \psi
            &\mapsto T \bar\psi \mat\gamma^\mu \psi T
            \intertext{%
                We add unit operators in between the elements.
            }
            &= T \bar\psi T T^\dagger \mat\gamma^\mu T^\dagger T \psi T
            \intertext{%
                We use the given transformation formulas on the field
                operators. At the same time, we use that $\gamma^\mu$ is a
                hermitian matrix and factor out the hermitian conjugate.
            }
            &= - \bar\psi(-t, \vec x) \mat\gamma^1 \mat\gamma^3 \sbr{T
            \mat\gamma^\mu T}^\dagger \mat\gamma^1 \mat\gamma^3 \psi(-t, \vec x)
            \intertext{%
                Next we use the hint which tells us that time reversal complex
                conjugates the Dirac matrices: $\mat\gamma^\mu$ will be complex
                conjugated. For an hermitian matrix, this is the same as the
                transpose. For the cases $\mu = 1$ and $\mu = 3$ the transpose
                will have an additional minus sign compared to the matrix
                itself. However, these are also the cases where we need one
                anticommutation less to get the $\mat\gamma^\mu$ across the
                $\mat\gamma^1 \mat\gamma^3$. This effect therefore cancels.
            }
            % TODO
        \end{align*}

    \item
        The pseudovector works in the same way. We have an additional
        $\mat\gamma^5$ here. Since the time reversal generates two Dirac
        matrices additionally, we need two anticommutations to get them across
        the $\mat\gamma^5$ this time. No sign change happens because of this.

    \item
        % TODO

    \item
        The partial derivative transform with the chain rule again. We only
        change the sign in the time direction, so the spatial parts are left
        intact. With our convention of the signature of the metric, this will
        be $- \eta^{\mu\mu}$.
\end{enumerate}

\paragraph{Charge conjugation}

The charge conjugation includes spatial and temporal Dirac matrices in the
transformed results, so there will not be any distinction on the indices $\mu$
or $\nu$ in the final results since there are the same number of
anticommutations each.

\begin{enumerate}
    \item
        We start with the scalar.
        \begin{align*}
            \bar\psi \psi
            &\mapsto C \bar\psi \psi C \\
            \intertext{%
                One needs another unit operator in the middle to use the
                transformation rule.
            }
            &= C \bar\psi C C \psi C
            \intertext{%
                We use the transformation given in Equation~(3).
            }
            &= \sbr{\mat\gamma^0 \mat\gamma^2 \psi}^\mathrm T
            \sbr{\bar\psi \mat\gamma^0 \mat\gamma^2}^\mathrm T
            \intertext{%
                We take the transpose of the whole thing to get the order back
                in the normal one. In this step we change the order of the
                field operators which will give us another minus sign from the
                fermionic anticommutation.
            }
            &= - \sbr{\bar\psi \mat\gamma^0 \mat\gamma^2 \mat\gamma^0 \mat\gamma^2 \psi}^\mathrm T
            \intertext{%
                The expression in the bracket is a scalar, we can therefore
                drop the transpose.
            }
            &= - \bar\psi \mat\gamma^0 \mat\gamma^2 \mat\gamma^0 \mat\gamma^2 \psi
            \intertext{%
                It takes one anticommutation to remove all the Dirac matrices.
            }
            &= \bar\psi \psi
        \end{align*}

    \item
        The additional $\mat\gamma^5$ in the pseudo scalar will lead to two
        additional anticommutations which do not change the sign in the end.

    \item
        The additional $\mat\gamma^\mu$ will cost one anticommutation with the
        Dirac matrices already in the expression. As seen before the transpose
        of the Dirac matrix gives additional minus signs for $\mu = 1$ and $\mu
        = 3$. Here the Dirac matrices for $\mu = 0$ and $\mu = 2$ are present,
        so the minus sign does not cancel but happen in every of the four
        possible terms.

    \item
        An additional $\mat\gamma^5$ in the pseudo vector again flips the sign
        with respect to the vector.

    \item
        % TODO

    \item
        Derivatives of spacetime have nothing to do with charge, so no chain
        rule applies and the sign stays as it is.
\end{enumerate}

\paragraph{All transformations}

This one is easy: Just multiply all the factors from parity, time reversal and
charge conjugation and you get the factor from the combined $CPT$
transformation.

\subsection{Short answers}

There are six questions. We will number them to aid navigation.

\begin{enumerate}
    \item
        Angular momentum probably is the Hodge dual of a 2-form, which does
        changes the sign twice under parity. Just like the magnetic field $\vec
        B$, which is the Hodge dual of a 2-form. It only feels strange to have
        vectorial quantities that are invariant under parity because they are
        constructed from forms and we then forget to mention that. This class
        of vectors is called pseudo vectors or axial vectors and usually arises
        from a cross product, which is related to the wedge product, but only
        in $\R^3$.

    \item
        The direction of propagation flips, the angular momentum does not.
        Together the helicity flips and therefore is called a pseudo scalar.

    \item
        Angular momentum changes its sign under time inversion. So the helicity
        does not change. That would be really strange actually, if the helicity
        would change in time reversal. Think of a movie of a falling gyro
        played backwards.

    \item
        The generators for boosts would have a sign change under both parity
        and time inversion. The generators for rotations would not change under
        either transformation.

    \item
        % TODO

    \item
        % TODO
\end{enumerate}

\section{Wick's theorem}
\label{homework:2}



\end{document}

% vim: spell spelllang=en tw=79
