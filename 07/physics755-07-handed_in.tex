\documentclass[11pt, english, fleqn, DIV=15, headinclude, BCOR=1cm]{scrartcl}

\usepackage[bibatend]{../header}
\usepackage{../my-boxes}

\usepackage{booktabs}
\usepackage{slashed}
\usepackage{simplewick}

\hypersetup{
    pdftitle=
}

\newcounter{totalpoints}
\newcommand\punkte[1]{#1\addtocounter{totalpoints}{#1}}

\newcounter{problemset}
\setcounter{problemset}{7}

\subject{physics755 -- Quantum Field Theory}
\ihead{physics755 -- Problem Set \arabic{problemset}}

\title{Problem Set \arabic{problemset}}

\newcommand\thegroup{Group Tuesday -- Ripunjay Acharya}

\publishers{\thegroup}
\ofoot{\thegroup}

\author{
    Martin Ueding \\ \small{\href{mailto:mu@martin-ueding.de}{mu@martin-ueding.de}}
    \and
    Oleg Hamm
}
\ifoot{Martin Ueding, Oleg Hamm}

\ohead{\rightmark}

\begin{document}

\maketitle

\vspace{3ex}

\begin{center}
    \begin{tabular}{rrr}
        problem & achieved points & possible points \\
        \midrule
        \nameref{homework:1} & & \punkte{10} \\
        \nameref{homework:2} & & \punkte{5} \\
        \midrule
        total & & \arabic{totalpoints}
    \end{tabular}
\end{center}

\section{$PCT$ and all that}
\label{homework:1}

\subsection{Table}

Here we have to compute $4 \cdot 6 = 24$ transformation properties. We get
whopping four points for this problem. So this means that we have to write
about four pages to get everything done. The ratio seems rather bad here as we
only finish on page~\pageref{page:end_of_1_1} with the very first problem.

Often we have to use that the symmetric part of $\tens\gamma \otimes
\tens\gamma$ is the metric tensor:
\[
    \frac12 \sbr{\mat\gamma^\mu, \mat\gamma^\nu} = \eta^{\mu\nu} \mat1_4.
\]
In particular this means that
\[
    [\mat\gamma^\mu]^2 = \eta^{\mu\mu} \mat1_4.
\]

Why is the quantity $\frac\iup2 [\mat\gamma^\mu, \mat\gamma^\nu]$ now called
$\mat\sigma^{\mu\nu}$. We had defined this earlier to be the commutator of the
Pauli matrices. The commutator of the Dirac matrices was called $S^{\mu\nu}$.
Either way, we will use the notation from this problem set.

\paragraph{Parity}

Since there are six tensorial quantities, we will just number them for each
transformation operator.

\begin{enumerate}
    \item 
        To transform with the parity, we just sandwich the expression into $P$s
        and use that $P^2$ is the identity since it comes from the cyclic group
        $\Z_2$. We added two independent $\Z_2$ to the universal covering group
        we get from the exponential map of the Lie algebra $\so(1, 3)$ to reach
        the whole Lorentz group.
        \begin{align*}
            \bar\psi(t, \vec x) \psi(t, \vec x)
            &\mapsto P \bar\psi(t, \vec x) \psi(t, \vec x) P \\
            \intertext{%
                We introduce a $P^2 = 1$ in the middle.
            }
            &= P \bar\psi(t, \vec x) P P \psi(t, \vec x) P \\
            \intertext{%
                Now we can use the transformation rule given in Equation~(2) on
                the problem set. The phase $\eta$ here is not the metric tensor
                $\eta$ which shows up in the table given on the problem set. It
                neither is the rapidity $\eta$ which was used on a different
                sheet. This is ambiguous, but luckily, there is always context
                around, to make it clear. Right?
            }
            &= \eta^* \bar\psi(t, -\vec x) \mat\gamma^0 \eta \mat\gamma^0 \psi(t, -\vec x) \\
            \intertext{%
                The phase $\eta$ is a complex number and therefore commutes
                with the matrices.
            }
            &= \eta^* \eta \bar\psi(t, -\vec x) \mat\gamma^0 \mat\gamma^0 \psi(t, -\vec x) \\
            \intertext{%
                The modulus of the phase is unity, so we can drop this. The
                Dirac matrices are their own inverses, so we can drop those as
                well. We are left with the Lorentz scalar as the transformed
                point.
            }
            &= \bar\psi(t, -\vec x) \psi(t, -\vec x)
            \intertext{%
                To make it more apparent that we transformed the whole thing to
                one new coordinate, we can write it like this:
            }
            &= [\bar\psi \psi](t, -\vec x).
        \end{align*}
        Then the additional factor that we get is just $+1$. If we keep up this
        pace we will end up with a quarter of a point per page, so we need to
        pick up the pace a bit.

    \item
        The next one, $\iup \bar\psi \mat\gamma^5 \psi$, has an imaginary unit
        in it, as well as a $\mat\gamma^5$. The parity operator will give us
        $\mat\gamma^0$s, which anticommute with the $\mat\gamma^5$, so that
        will give us the minus sign.

    \item
        The bilinear $\bar\psi \mat\gamma^\mu \mat\gamma^5 \psi$ will transform
        differently depending on the $\mu$. If $\mu = 0$, it commutes with the
        $\mat\gamma^0$ that we get from the parity transformation. If $\mu \neq
        0$, we have to perform an anticommutation. Hence a factor of
        $\eta^{\mu\nu}$.

    \item
        Adding another $\mat\gamma^5$ to the bilinear means another
        anticommutation compared to the previous one and therefore a factor of
        $- \eta^{\mu\mu}$.

    \item
        We expand the transformed bilinear:
        \[
            \bar\psi \mat\gamma^0 \mat\sigma^{\mu\nu} \mat\gamma^0 \psi
            = \frac\iup2 \bar\psi \mat\gamma^0 [\mat\gamma^\mu, \mat\gamma^\nu] \mat\gamma^0 \psi
            = \frac\iup2 \bar\psi \sbr{
                \mat\gamma^0 \mat\gamma^\mu \mat\gamma^\nu \mat\gamma^0
                - \mat\gamma^0 \mat\gamma^\nu \mat\gamma^\mu \mat\gamma^0
        } \psi.
        \]
        One can see that it takes two anticommutations to get the
        $\mat\gamma^0$ to cancel the other one. Those anticommutations
        introduce factors of $\eta^{\mu\mu}$ and $\eta^{\nu\nu}$ respectively.
        Therefore the factor that we end up with is the product of the two.

    \item
        We apply the chain rule here and since we map $\vec x \mapsto - \vec x$
        the chain rule will just give us a minus sign in the spatial
        components. This is expressed in $\eta^{\mu\mu}$.
\end{enumerate}

\paragraph{Time reversal}

We have to do the same thing for the time reversal operation.

\begin{enumerate}
    \item
        We do the first one more explicit as before. We start with the sandwich
        between the time reversal operators.
        \begin{align*}
            \bar\psi \psi
            &\mapsto T \bar\psi \psi T
            \intertext{%
                We add a unit operator in between the field operators.
            }
            &= T \bar\psi T^\dagger T \psi T
            \intertext{%
                To use the formula that is given in Equation~(5) we must assume
                that $T^\dagger = T$. Or are we really using that $T^2$ must be
                the identity and not introduce a $T^\dagger$ at all? Either
                way, we must write
            }
            &= T \bar\psi T T \psi T.
            \intertext{%
                Now we apply Equation~(5) from the problem set.
            }
            &= - \bar\psi(-t, \vec x) \mat\gamma^1 \mat\gamma^3 \mat\gamma^1
            \mat\gamma^3 \psi(-t, \vec x)
            \intertext{%
                It takes one anticommutation in the middle.
            }
            &= \bar\psi(-t, \vec x) \mat\gamma^3 \mat\gamma^1 \mat\gamma^1
            \mat\gamma^3 \psi(-t, \vec x)
            \intertext{%
                We can collapse the Dirac matrices and have a compact result.
                Both pairs of Dirac matrices give a minus, but that does not
                change the net sign.
            }
            &= [\bar\psi \psi](-t, \vec x)
        \end{align*}

    \item
        The pseudo scalar has another $\mat\gamma^5$ which means another
        anticommutation is needed. This makes the factor $-1$.

    \item
        The vector is also a bit tricky, so we need several steps here
        as well.
        \begin{align*}
            \bar\psi \mat\gamma^\mu \psi
            &\mapsto T \bar\psi \mat\gamma^\mu \psi T
            \intertext{%
                We add unit operators in between the elements.
            }
            &= T \bar\psi T T^\dagger \mat\gamma^\mu T^\dagger T \psi T
            \intertext{%
                We use the given transformation formulas on the field
                operators. At the same time, we use that $\mat\gamma^\mu$ is a
                hermitian matrix and factor out the hermitian conjugate.
            }
            &= - \bar\psi(-t, \vec x) \mat\gamma^1 \mat\gamma^3 \sbr{T
            \mat\gamma^\mu T}^\dagger \mat\gamma^1 \mat\gamma^3 \psi(-t, \vec x)
            \intertext{%
                Next we use the hint which tells us that time reversal complex
                conjugates the Dirac matrices: $\mat\gamma^\mu$ will be complex
                conjugated. For an hermitian matrix, this is the same as the
                transpose. For the cases $\mu = 1$ and $\mu = 3$ the transpose
                will have an additional minus sign compared to the matrix
                itself. However, these are also the cases where we need one
                anticommutation less to get the $\mat\gamma^\mu$ across the
                $\mat\gamma^1 \mat\gamma^3$. This effect therefore cancels.
                The $\mu = 0$ case has no sign change associated with it
                whereas the other cases have a sign change.
            }
            &= - \eta^{\mu\mu} \bar\psi(-t, \vec x) \mat\gamma^1 \mat\gamma^3
            \mat\gamma^1 \mat\gamma^3 \mat\sigma^{\mu\nu} \psi(-t,
            \vec x)
            \intertext{%
                We need to anticommute the Dirac matrices to form nested pairs.
                This will change the sign again.
            }
            &= \eta^{\mu\mu} \bar\psi(-t, \vec x) \mat\gamma^1 \mat\gamma^3
            \mat\gamma^3 \mat\gamma^1 \mat\sigma^{\mu\nu} \psi(-t,
            \vec x)
            \intertext{%
                The two nested pairs will both give a sign change this cancels
                in the end result.
            }
            &= \eta^{\mu\mu} \bar\psi(-t, \vec x) \mat\sigma^{\mu\nu} \psi(-t,
            \vec x)
        \end{align*}

    \item
        The pseudovector works in the same way. We have an additional
        $\mat\gamma^5$ here. Since the time reversal generates two Dirac
        matrices additionally, we need two anticommutations to get them across
        the $\mat\gamma^5$ this time. No sign change happens because of this.

    \item
        The tensor requires care of all the different cases. We start by
        expanding the Pauli matrix tensor.
        \begin{align*}
            T \bar\psi \mat\sigma^{\mu\nu} \psi T
            &= T \bar\psi \frac\iup2 \sbr{\mat\gamma^\mu, \mat\gamma^\nu} \psi T
            \intertext{%
                We add more time reflection operators.
            }
            &= T \bar\psi TT \frac\iup2 \sbr{\mat\gamma^\mu, \mat\gamma^\nu} TT \psi T
            \intertext{%
                There are a lot of sign changes here, so we want to be careful.
                The first one comes from moving $T$ past the imaginary unit.
            }
            &= - T \bar\psi T \frac\iup2 T \sbr{\mat\gamma^\mu, \mat\gamma^\nu} TT \psi T
            \intertext{%
                Now we can apply the time reversal on both the spinors and the
                commutator of Dirac matrices.
            }
            &= \bar\psi \mat\gamma^1 \mat\gamma^3 \frac\iup2
            \sbr{\mat\gamma^{\mu*}, \mat\gamma^{\nu*}} \mat\gamma^1
            \mat\gamma^3 \psi
            \intertext{%
                We perform one anticommutation to the matrices in the right
                order to cancel each other once we have moved them through the
                commutator. This gives us another minus sign.
            }
            &= - \bar\psi \mat\gamma^1 \mat\gamma^3 \frac\iup2
            \sbr{\mat\gamma^{\mu*}, \mat\gamma^{\nu*}} \mat\gamma^3
            \mat\gamma^1 \psi
            \intertext{%
                Only the Dirac matrix $\mat\gamma^2$ is imaginary, all the
                other ones are real in the Weyl representation. The complex
                conjugation therefore gives us a minus sign whenever $\mu = 2$
                or $\nu = 2$. It is symmetric in $\mu$ and $\nu$, so we only
                need to look at six distinct cases. We start with $\mu = 0$ and
                $\nu = 1$. We need one anticommutation for $\mat\gamma^1$ and
                two anticommutations for $\gamma^3$. No minus sign will come
                from the complex conjugation. We therefore write this as
                \[
                    N^{\mu\nu} = N_{\mat\gamma^1} + N_{\mat\gamma^3} + N_{*}
                    = 1 + 2 + 0 = 3
                \]
                sign changes in total.
            }
            \intertext{%
                Going through all the cases in this
                fashion, we end up with the following symmetric matrix:
                \[
                    \mat N =
                    \begin{pmatrix}
                        & 1+2+0=3 & 2+2+1=5 & 2+1+0=3 \\
                        & & 1+2+1=4 & 1+1+0=2 \\
                        & & & 2+1+1=4
                    \end{pmatrix}.
                \]
                This can be summed up in
            }
            &= - \eta^{\mu\mu} \eta^{\nu\nu} \bar\psi \mat\gamma^1
            \mat\gamma^3 \mat\gamma^3 \mat\gamma^1 \gamma^{\mu\nu} \psi.
            \intertext{%
                Now we can remove the pairs of Dirac matrices. Each pair will
                give a minus sign, so the total sign does not change.
            }
            &= - \eta^{\mu\mu} \eta^{\nu\nu} \bar\psi \psi
        \end{align*}

    \item
        The partial derivative transform with the chain rule again. We only
        change the sign in the time direction, so the spatial parts are left
        intact. With our convention of the signature of the metric, this will
        be $- \eta^{\mu\mu}$.
\end{enumerate}

\paragraph{Charge conjugation}

The charge conjugation includes spatial and temporal Dirac matrices in the
transformed results, so there will not be any distinction on the indices $\mu$
or $\nu$ in the final results since there are the same number of
anticommutations each.

\begin{enumerate}
    \item
        We start with the scalar.
        \begin{align*}
            \bar\psi \psi
            &\mapsto C \bar\psi \psi C \\
            \intertext{%
                One needs another unit operator in the middle to use the
                transformation rule.
            }
            &= C \bar\psi C C \psi C
            \intertext{%
                We use the transformation given in Equation~(3).
            }
            &= - \sbr{\mat\gamma^0 \mat\gamma^2 \psi}^\mathrm T
            \sbr{\bar\psi \mat\gamma^0 \mat\gamma^2}^\mathrm T
            \intertext{%
                We take the transpose of the whole thing to get the order back
                in the normal one. In this step we change the order of the
                field operators which will give us another minus sign from the
                fermionic anticommutation.
            }
            &= \sbr{\bar\psi \mat\gamma^0 \mat\gamma^2 \mat\gamma^0 \mat\gamma^2 \psi}^\mathrm T
            \intertext{%
                The expression in the bracket is a scalar, we can therefore
                drop the transpose.
            }
            &= \bar\psi \mat\gamma^0 \mat\gamma^2 \mat\gamma^0 \mat\gamma^2 \psi
            \intertext{%
                It takes one anticommutation to order the Dirac matrices into
                nested pairs.
            }
            &= - \bar\psi \mat\gamma^0 \mat\gamma^2 \mat\gamma^2 \mat\gamma^0 \psi
            \intertext{%
                The pair $\mat\gamma^2$ will give us a minus sign, the pair
                with $\mat\gamma^0$ will not. This is another sign change.
            }
            &= \bar\psi \psi
        \end{align*}

    \item
        The additional $\mat\gamma^5$ in the pseudo scalar will lead to two
        additional anticommutations which do not change the sign in the end.

    \item
        The additional $\mat\gamma^\mu$ will cost one anticommutation with the
        Dirac matrices already in the expression. As seen before the transpose
        of the Dirac matrix gives additional minus signs for $\mu = 1$ and $\mu
        = 3$. Here the Dirac matrices for $\mu = 0$ and $\mu = 2$ are present,
        so the minus sign does not cancel but happen in every of the four
        possible terms.

    \item
        An additional $\mat\gamma^5$ in the pseudo vector again flips the sign
        with respect to the vector.

    \item
        The tensor bilinear under charge conjugation is next.
        \begin{align*}
            \bar\psi \mat\sigma^{\mu\nu} \psi
            &\mapsto C \bar\psi \mat\sigma^{\mu\nu} \psi C
            \intertext{%
                We expand the matrix commutator and add more conjugation
                operators. We use the property that the charge conjugation
                operator commutes with all the Dirac matrices.
            }
            &= C \bar\psi C \frac\iup2 \sbr{\mat\gamma^\mu, \mat\gamma^\nu} C \psi C
            \intertext{%
                Then we write out the transformed spinors.
            }
            &= - \sbr{\mat\gamma^0 \mat\gamma^2 \psi}^\mathrm T
            \frac\iup2 \sbr{\mat\gamma^\mu, \mat\gamma^\nu}
            \sbr{\bar\psi \mat\gamma^0 \mat\gamma^2}^\mathrm T
            \intertext{%
                We do the same thing as with the scalar, to get everything into
                a transpose bracket. After that, we can drop it since the
                transpose of a scalar still is a scalar.
            }
            &= - \bar\psi \mat\gamma^0 \mat\gamma^2
            \frac\iup2 \sbr{\mat\gamma^\mu, \mat\gamma^\nu}^\mathrm T
            \mat\gamma^0 \mat\gamma^2 \psi
            \intertext{%
                We perform the anticommutation of the Dirac matrices now and
                remove the leading minus sign.
            }
            &= \bar\psi \mat\gamma^0 \mat\gamma^2
            \frac\iup2 \sbr{\mat\gamma^\mu, \mat\gamma^\nu}^\mathrm T
            \mat\gamma^2 \mat\gamma^0 \psi
        \end{align*}
        The transpose of the commutator will give introduce additional minus
        signs for $\mu = 2$ or $\nu = 2$. In those cases we need one
        anticommutation less, so there is no net sign change when this is the
        case. If the other index is 0, one should need one less anticommutation
        and therefore incur a sign change.

        Once everything is brought through the commutator, the nested pairs of
        Dirac matrices will give an additional minus sign.

        % TODO

    \item
        Derivatives of spacetime have nothing to do with charge, so no chain
        rule applies and the sign stays as it is.
\end{enumerate}

\paragraph{All transformations}

This one is easy: Just multiply all the factors from parity, time reversal and
charge conjugation and you get the factor from the combined $CPT$
transformation.

\label{page:end_of_1_1}

\subsection{Short answers}

There are six questions. We will number them to aid navigation.

\begin{enumerate}
    \item
        Angular momentum probably is the Hodge dual of a 2-form, which does
        changes the sign twice under parity. Just like the magnetic field $\vec
        B$, which is the Hodge dual of a 2-form. It only feels strange to have
        vectorial quantities that are invariant under parity because they are
        constructed from forms and we then forget to mention that. This class
        of vectors is called pseudo vectors or axial vectors and usually arises
        from a cross product, which is related to the wedge product, but only
        in $\R^3$.

    \item
        The direction of propagation flips, the angular momentum does not.
        Together the helicity flips and therefore is called a pseudo scalar.

    \item
        Angular momentum changes its sign under time inversion. So the helicity
        does not change. That would be really strange actually, if the helicity
        would change in time reversal. Think of a falling gyro which has a
        meridian marked. The helix described by this thing should not change
        with time reversal.

    \item
        Which representation is meant? We assume that it is the natural one on
        the four-vectors of Minkowski space.

        The generators for boosts would have a sign change under both parity
        and time inversion. Write such a generator $\mat K$ with either
        transformation: $P \mat K P$. The one parity operator will flip the
        sign in the spatial columns of $\mat K$, the other one will flip the
        sign in the spatial rows. Together, only the signs of the $K^{0i}$
        components are flipped. The time reversal will slip the temporal row
        and column, yielding the same result.

        The generators for rotations would not change under either
        transformation. The parity will change the spatial components twice.
        The rotation generator $\mat J$ only has spatial components that are
        nonzero, so it does not change at all. The time reversal will not
        change the spatial components at all, so $\mat J$ is invariant here as
        well. 

    \item
        The representations of the Group $\SO(1, 3)$ that we know are (a)
        $\SO(1, 3)$ matrices that act on the four-vectors of Minkowski space,
        (b) differential operators on the scalar functions, (c) $2 \times 2$
        matrices on Weyl spinors and (d) $4 \times 4$ matrices on the Dirac
        spinors. Representation (d) is just representation (c) repeated twice
        in a block diagonal form.

        % TODO

    \item
        % TODO
\end{enumerate}

\subsection{Complex Klein-Gordon field}

The real field $\phi(\vec x)$ is given as:
\[
    \phi(\vec x) = \int \frac{\dif^3 p}{[2\piup]^3} \frac{1}{\sqrt{E_{\vec p}}}
    \sbr{
        a_{\vec p} \exp(- \iup \vec p \cdot \vec x)
        + a_{\vec p}^\dagger \exp(\iup \vec p \cdot \vec x)
    }.
\]
Since we have a complex Klein-Gordon field we need two different sets of ladder
operators. We call the second one $b$ and $b^\dagger$.
\[
    \phi(\vec x) = \int \frac{\dif^3 p}{[2\piup]^3} \frac{1}{\sqrt{E_{\vec p}}}
    \sbr{
        a_{\vec p} \exp(- \iup \vec p \cdot \vec x)
        + b_{\vec p}^\dagger \exp(\iup \vec p \cdot \vec x)
    }.
\]

We can add a time dependence by making this Schrödinger picture operator into a
Heisenberg picture operator.
\[
    \phi_\mathrm H(t, \vec x) = \int \frac{\dif^3 p}{[2\piup]^3} \frac{1}{\sqrt{E_{\vec p}}}
    U^\dagger(t) \sbr{
        a_{\vec p} \exp(- \iup \vec p \cdot \vec x)
        + b_{\vec p}^\dagger \exp(\iup \vec p \cdot \vec x)
    } U(t)
\]
This does not help us much, yet.

We need to write this more compact, with an explicit time dependence in the
exponentials. \textcite[25]{Peskin/QFT/1995} look at the commutator of the
Hamiltonian operator $H$ with the annihilation operator $a_{\vec p}$ and use
this. Here we focus on $a$ only. We will show the commutator first.
\begin{align*}
    \sbr{H, a_{\vec p}}
    &= \int \frac{\dif^3 p'}{[2\piup]^3} \omega_{\vec p'}
    \sbr{a_{\vec p'}^\dagger a_{\vec p'}, a_{\vec p}}
    \intertext{%
        The annihilation operator commutes with itself, we can extract that
        from the commutator.
    }
    &= \int \frac{\dif^3 p'}{[2\piup]^3} \omega_{\vec p'}
    \sbr{a_{\vec p'}^\dagger, a_{\vec p}} a_{\vec p'}
    \intertext{%
        By definition, this is a Dirac $\deltaup$-distribution. Executing the
        integral just gives us
    }
    &= \omega_{\vec p} a_{\vec p}.
\end{align*}
This result can then be used to remove the unitary time evolution operators $U$
and replace them with a straight time dependence in the exponential by using
this relation $n$ times in each summand of the exponential. The same procedure
can be done for $b^\dagger$ as well, the sign will be the opposite, though.

We now have
\[
    \phi_\mathrm H(t, \vec x) = \int \frac{\dif^3 p}{[2\piup]^3} \frac{1}{\sqrt{E_{\vec p}}}
    \sbr{
        a_{\vec p} \exp\del{\iup [E_{\vec p} t - \vec p \cdot \vec x]}
        + b_{\vec p}^\dagger \exp\del{- \iup [E_{\vec p} t - \vec p \cdot \vec x]}
    }.
\]
We finally have the field expressed in terms of creation and annihilation
operators with explicit time dependence. One could use four-vectors to compress
the notation a bit, but we need to look at space and time individually in the
next parts anyway.

\paragraph{Parity}

The action of our transformation operators shall be given in terms of their
action on the ladder operators. To flip $\vec x$ here, we should map $a_{\vec
p} \mapsto a_{- \vec p}$ and then use the symmetry of the integral with respect
to $p$. The exponential would have the exact same value if one looks at $\vec x
\mapsto - \vec x$, which is the desired transformation.

\paragraph{Time reversal}

The time reversal is trickier, as expected. We need to exchange the
exponentials and also perform the parity. So we have $a_{\vec
p} \mapsto b^\dagger_{- \vec p}$ and the corresponding other way around.

\paragraph{Charge conjugation}

The complex Klein-Gordon field has distinct particles and antiparticles. To
conjugate the charge, we need to exchange those two types of particles. This
can be done by exchanging $a \mapsto b$ and $b \mapsto a$.

\paragraph{Transformation properties of current}

\begin{itemize}
    \item
        The transformation with the parity is:
        \begin{align*}
            J^\mu
            &\mapsto P J^\mu P \\
            &= \iup \sbr{P \phi^* P P \partial^\mu P P \phi P - P \phi P P
            \partial^\mu P P \phi^* P} \\
            &= - \iup \sbr{\phi^*(t, -\vec x) \partial^\mu \phi(t, -\vec x) -
            \phi(t, -\vec x) \partial^\mu \phi^*(t, -\vec x)} \\
            &= - J^\mu(t, -\vec x)
        \end{align*}
        The current is a regular vector. The current density actually is a
        three-form since it assigns a scalar to three vectors (one timelike,
        two spacelike). Therefore it has an odd parity.

    \item
        The charge conjugation should flip the sign of the current. And indeed
        it does. The charge density is purely imaginary, taking the complex
        conjugate reverses the sign:
        \[
            J^\mu \mapsto - J^\mu.
        \]

    \item
        The time reversal works similarly, except that there is no sign change
        from the partial derivative. However, we get a sign change from the
        imaginary unit and the fields get complex conjugated. This just changes
        the order and also gives a sign change with respect to the original
        current density.
        \[
            J^\mu \mapsto J^{\mu}(-t, \vec x)
        \]
\end{itemize}

\subsection{CPT invariance}

When we computed the table with the transformation properties we have seen that
both scalars and pseudo-scalars composed of $\psi$ are invariant under $CPT$.
Something only is a Lorentz scalar if it transforms as such. And scalars do not
change their sign under parity or time reversal. The charge conjugation could
give us the complex conjugate in principle. However, hermitian scalars must be
real, therefore the complex conjugation does not change anything. All in all,
hermitian (and therefore real) scalars are invariant under $CPT$.

\section{Wick's theorem}
\label{homework:2}

\newcommand\timeorder{\mathscr T}
\newcommand\normorder{\mathscr N}

This problems introduces another overload of the letter $T$, it now is both
time ordering and time reversal operator. We have yet to define another $T$ to
make it even with the thrice defined $\eta$. To make it easier for the reader,
we chose to introduce script letters here, mostly because \LaTeX\ can do this
and they are not used yet. The time ordering will be denoted with $\timeorder$
and the normal ordering with $\normorder$.

\subsection{Two fields}

We have to look at vacuum matrix elements here.
\textcite[Section~4.3]{Peskin/QFT/1995} do this as well, and we need it to get
a Feynman propagator in the end. We start by expanding the time ordering in
terms of Heaviside step functions.
\begin{align*}
    \Braket{0 | \timeorder(\phi_1 \phi_2) | 0}
    &= \Theta(x_1^0 - x_2^0) \Braket{0 | \phi_1 \phi_2 | 0}
    + \Theta(x_2^0 - x_1^0) \Braket{0 | \phi_2 \phi_1 | 0}
    \intertext{%
        We can identify this with the definition of the Feynman propagator.
    }
    &= D_\mathrm F(\tens x - \tens y)
    \intertext{%
        The vacuum state is normalized, we can add a bracket around this scalar
        value.
    }
    &= \Braket{0 | D_\mathrm F(\tens x - \tens y) | 0}
    \intertext{%
        The normal ordering of any operator does not contribute anything in
        vacuum matrix element, because all the occupation numbers are zero in
        the vacuum.
    }
    &= \Braket{0 | \normorder(\phi_1 \phi_2) | 0}
    + \Braket{0 | D_\mathrm F(\tens x - \tens y) | 0}
    \intertext{%
        We use the linearity of the braket.
    }
    &= \Braket{0 | \normorder(\phi_1 \phi_2) + D_\mathrm F(\tens x - \tens y) | 0}
    \intertext{%
        The contraction is defined via the Feynman propagator. We use this the
        other way around and insert a contraction.
    }
    &= \Braket{0 | \normorder(\phi_1 \phi_2) +
    \contraction{}{\phi_1}{}{\phi_2}
    \phi_1 \phi_2
    | 0}
    \intertext{%
        And as a last step we extend the normal ordering to the scalar.
    }
    &= \Braket{0 | \normorder\del{\phi_1 \phi_2 +
    \contraction{}{\phi_1}{}{\phi_2}
    \phi_1 \phi_2}
    | 0}
\end{align*}
Dropping the matrix element again, we can conclude that Equation~(11) holds:
\[
    \timeorder(\phi_1 \phi_2)
    = \normorder\del{\phi_1 \phi_2 +
    \contraction{}{\phi_1}{}{\phi_2}
    \phi_1 \phi_2}
\]

\subsection{Proof of Wick's theorem}

We saw in the previous subsection that the vacuum matrix elements are needed to
get the propagators, but can be omitted, as long as we keep the vacuum in mind.

The theorem is proven for $n = 2$ fields. The case $n = 1$ is not very
interesting since there is not much to reorder if there is just one field. The
foundation for the induction is laid, we need the induction step next. We chose
the variant to start with the left side at $n+1$, use the theorem at $n$ and
bring the other side to $n+1$ as well. The first step is extract the newest
element from the time ordering.

\textcite[90]{Peskin/QFT/1995} define the numbering of the fields such that
$x_k^0 \geq x_{k+1}^0$, such that they are in time order already. If we do not
use this trick as well, we would have $n+1$ terms when we write down the time
ordering using Heaviside step functions explicitly. Therefore we will also use
this trick. This means that $\phi_{n+1}$ has the smallest time of all and
already is in the correct time order when it is at the very end. This does not
mean that we loose generality by this restriction. We can always relabel the
fields such that this holds: The time ordering $\timeorder$ will order them by
time anyway. And the normal ordering $\normorder$ splits up positive and
negative frequency parts which can commute among themselves freely. All
possible contractions are build up, so reordering the fields would not change
anything either.
\begin{align*}
    \timeorder(\phi_1 \phi_2 \ldots \phi_n \phi_{n+1})
    &= \timeorder(\phi_1 \phi_2 \ldots \phi_n) \phi_{n+1}
    \intertext{%
        Then we can apply the theorem for $n$. To make the notation a tad more
        precise we call the sum of all contractions up to and including the
        field $n$ the sum $C_n$.
    }
    &= \normorder(\phi_1 \phi_2 \ldots \phi_n + C_n) \phi_{n+1}
    \intertext{%
        Here we need to split up the positive and negative frequency parts.
    }
    &= \normorder(\phi_1 \phi_2 \ldots \phi_n + C_n) \phi_{n+1}^+
    + \normorder(\phi_1 \phi_2 \ldots \phi_n + C_n) \phi_{n+1}^-
    \intertext{%
        The normal ordering has the negative frequency parts on the left. This
        means that the positive frequency part already is in the right position
        and we can just insert that. The negative frequency part is on the
        wrong side to be normal ordered. We therefore need to commute it with
        every single positive frequency part. Every commutation will give us a
        commutator, which is a Feynman propagator when sandwiched into a vacuum
        matrix element.
    }
    &= \normorder(\phi_1 \phi_2 \ldots \phi_n \phi_{n+1}^+ + C_n \phi_{n+1}^+)
    + \normorder(\phi_{n+1}^- \phi_1 \phi_2 \ldots \phi_n + \phi_{n+1}^- C_n)
    \\&\qquad
    + \sbr{\normorder(\phi_1 \phi_2 \ldots \phi_n), \phi_{n+1}^-}
    + \sbr{\normorder(C_n), \phi_{n+1}^-}
    \intertext{%
        The first summands of the terms in the first line can be combined to
        give the normal ordering of all $n+1$ fields. The terms with $C_n$ will
        give all the contractions where $\phi_{n+1}$ is not involved.
        The first commutator will then give all the contractions that
        $\phi_{n+1}$ is involved in, but nothing else is contracted. Finally,
        the commutator of $\phi_{n+1}$ with $C_n$ will give all the
        contractions where $\phi_{n+1}$ is involved and other things are
        contracted as well. All in all we arrive at the final result:
    }
    &= \normorder(\phi_1 \phi_2 \ldots \phi_n \phi_{n+1}).
\end{align*}
By the method of induction, this theorem holds for all $n \geq 1$ now.

\end{document}

% vim: spell spelllang=en tw=79
