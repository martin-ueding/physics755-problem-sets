\documentclass[11pt, english, fleqn, DIV=15, headinclude, BCOR=1cm]{scrartcl}

\usepackage[bibatend]{../header}
\usepackage{../my-boxes}

\usepackage{booktabs}

\hypersetup{
    pdftitle=
}

\newcounter{totalpoints}
\newcommand\punkte[1]{#1\addtocounter{totalpoints}{#1}}

\newcounter{problemset}
\setcounter{problemset}{3}

\subject{physics755 -- Quantum Field Theory}
\ihead{physics755 -- Problem Set \arabic{problemset}}

\title{Problem Set \arabic{problemset}}

\newcommand\thegroup{Group Tuesday -- Ripunjay Acharya}

\publishers{\thegroup}
\ofoot{\thegroup}

\author{
    Martin Ueding \\ \small{\href{mailto:mu@martin-ueding.de}{mu@martin-ueding.de}}
    \and
    Oleg Hamm
}
\ifoot{Martin Ueding}

\ohead{\rightmark}

\begin{document}

\maketitle

\vspace{3ex}

\begin{center}
    \begin{tabular}{rrr}
        problem & achieved points & possible points \\
        \midrule
        \nameref{homework:1} & & \punkte{10} \\
        \midrule
        total & & \arabic{totalpoints}
    \end{tabular}
\end{center}

\section{Lorentz algebra 2}
\label{homework:1}

\subsection{Rotations and boosts}

The generators were derived on the previous problem set. They are:
\[
    J_{\rho\sigma} = x_{[\sigma} \partial_{\rho]}
\]
where we have used the antisymmetrization notation in the non-idempotent form
since that seems to be used in this class. This uses the Mathematician's
convention of real antisymmetric generators. To get the Physicists's notation,
we add $-\iup$ to the generators and $\iup$ into the exponential map.
\[
    J_{\rho\sigma} = \iup x_{[\rho} \partial_{\sigma]}
\]
Then $L^3$ is given by $\iup x_{[1} \partial_{2]}$. We can now write a finite
transformation:
\begin{align*}
    \Phi'(x) &= \exp(- \iup \Theta L^3) \, \Phi(x)
    \intertext{%
        We insert the generator.
    }
    &= \exp\del{\Theta x_{[1} \partial_{2]}} \, \Phi(x)
    \intertext{%
        Next we expand the antisymmetrization bracket.
    }
    &= \exp\del{\Theta [x_1 \partial_2 - x_2 \partial_1]} \, \Phi(x)
    \intertext{%
        So far, the generator is a scalar operator and act on functions. We
        want to convert it into a matrix representation that acts on the
        four-vectors. We can now write the generator as a matrix.
    }
    &= \exp\del{\Theta
    \begin{pmatrix}
        0 & 0 & 0 & 0 \\
        0 & 0 & -x & 0 \\
        0 & y & 0 & 0 \\
        0 & 0 & 0 & 0
    \end{pmatrix}
    \cdot \vec\partial
    } \, \Phi(x) \\
    \intertext{%
        One can think about this matrix term as something that originates from
        a chain rule. So calling the matrix $\tens A \cdot \vec x$, so
        $\tens A$ only contains $-1$ and $1$, we can write this as
    }
    &= \exp\del{\Theta
    \sbr{\pd{[\Theta \tens A \vec x]^\mu}{\Theta}}_{\Theta=0}
    \partial_\mu
    } \, \Phi(x) \\
    \intertext{%
        From here, the Taylor series can be seen.
    }
    &= \Phi(\tens A(\theta) \vec x)
\end{align*}

\end{document}

% vim: spell spelllang=en tw=79
