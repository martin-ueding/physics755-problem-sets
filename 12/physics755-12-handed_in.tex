\documentclass[11pt, english, fleqn, DIV=15, headinclude, BCOR=1cm]{scrartcl}

\usepackage[bibatend]{../header}
\usepackage{../my-boxes}

\usepackage{booktabs}
\usepackage{slashed}
\usepackage{simplewick}

\usepackage{feynmp-auto}
\usepackage{adjustbox}
\newenvironment{fmfwrapper}{\begin{adjustbox}{margin=5mm}}{\end{adjustbox}}

\usepackage{multicol}
\usepackage{lastpage}

\usepackage{tikz}
\usepackage{pgfplots}
\pgfplotsset{compat=1.10}

\newcommand\timeorder{\mathscr T}
\newcommand\normorder{\mathscr N}

\hypersetup{
    pdftitle=
}

\newcounter{totalpoints}
\newcommand\punkte[1]{#1\addtocounter{totalpoints}{#1}}

\newcounter{problemset}
\setcounter{problemset}{12}

\subject{physics755 -- Quantum Field Theory}
\ihead{physics755 -- Problem Set \arabic{problemset}}

\title{Problem Set \arabic{problemset}}

\newcommand\thegroup{Group Tuesday -- Ripunjay Acharya}

\publishers{\thegroup}
\ofoot{\thegroup}

\author{
    Martin Ueding \\ \small{\href{mailto:mu@martin-ueding.de}{mu@martin-ueding.de}}
    \and
    Oleg Hamm
}
\ifoot{Martin Ueding, Oleg Hamm}

\ohead{\rightmark}

\begin{document}

\maketitle

\vspace{3ex}

\begin{center}
    \begin{tabular}{rrr}
        problem & achieved points & possible points \\
        \midrule
        \nameref{homework:1} & & \punkte{15} \\
        \midrule
        total & & \arabic{totalpoints}
    \end{tabular}
\end{center}

\vspace{3ex}

\begin{center}
    \begin{small}
        This document consists of \pageref{LastPage} pages.
    \end{small}
\end{center}

\section{Dimensional regularization of $\phi^4$ theory at 1-loop}
\label{homework:1}

\subsection{Substitution}

We perform the Wick rotation by $p^0 \mapsto \iup p^0$.
\[
    \int \frac{\dif^d p}{[2\piup]^d} \frac{1}{[\tens p^2 - A]^n}
    = \int \frac{\dif^d p}{[2\piup]^d} \frac{1}{[[p^0]^2 - \vec p^2 - A]^n}
    \mapsto \int \frac{\iup \dif^d p}{[2\piup]^d} \frac{1}{[- [p^0]^2 - \vec
    p^2 - A]^n}.
\]

One thing \textcite{penrose-road_to_reality} warns to be careful about is that
the Wick rotation can turn a non-compact group like $\SO(1,3)$ into a compact
group like $\SO(4)$. Then one works with the compact group and goes back. One
should be careful.

\subsection{Proof of identity}

This problem is also somewhat covered by \textcite[249--250]{Peskin/QFT/1995}.
They also cover a similar subject on pp.\ 189--194.

We will divide this into a couple parts, just like they also do.
% FIXME Remove the next part once the errors are gone.
Since we were not able to get to the correct results, writing “Theorem” and
“Proof” next to obviously wrong things looks rather stupid.

\begin{theorem}[Surface of unit sphere]
    \label{the:surface}

    The surface of the $d$-dimensional unit sphere, i.e. the one where the
    surface itself has dimension $d-1$ is given by
    \[
        \int \dif \Omega_d = \frac{2 \piup^{\frac d2}}{\Gammaup\del{\frac d2}}.
    \]
\end{theorem}

\begin{proof}
    The measure is given by Equation~(3) on the problem set:
    \[
        d^d p = P^{d-1} dPd\phi \sin \theta_1 d\theta_1 \sin^2\theta_2 d
        \theta_2 \ldots \sin^{d-2} \theta_{d-1} d\theta_{d-2}.
    \]
    We think that this is unnecessarily hard to read. We write it without
    the ellipsis and upright exterior derivative operators like so:
    \[
        \dif^d p = P^{d-1} \dif P \dif \phi
        \prod_{k=1}^{d-2} \sin(\theta_k)^k \dif \theta_k
    \]

    The unit sphere has $P = 1$. We will set $P$ to unity. Then we can
    integrate over it and will end up with the area of the unit sphere.
    \[
        \int \dif \Omega_d = \int_0^{2\piup} \dif \phi
        \prod_{k=1}^{d-2} \int_0^\piup \sin(\theta_k)^k \dif \theta_k
    \]
    The $\phi$ integral will just give $2\piup$, but the other factors are not
    that easily computed. Mathematica can compute the $\theta_k$ integral and
    gives expressions with multiple $\Gammaup$-functions.

    Since we are not able to perform the integration of the factors which yield
    hypergeometric ${}_2F_1$ functions by hand, we will use the derivation by
    \textcite[249]{Peskin/QFT/1995}. They start by using the simplest Gaussian
    integral:
    \begin{align*}
        [\sqrt{\piup}]^d
        &= \sbr{\int \dif x \, \exp(-x^2)}^d
        \intertext{%
            Then they just factor out the integral and chose $\R^d$ as the
            domain of integration. The exponential factors are grouped together
            such that there is a sum in the integration.
        }
        &= \int \dif^d x \, \exp\del{- \sum_{i = 1}^d x_i^2}
        \intertext{%
            Now they change into those generalized polar coordinates with $P =
            x$.
        }
        &= \int x^{d-1} \dif x \dif \phi \prod_{k=1}^{d-2} \sin(\theta_k)^k
        \dif \theta_k \exp\del{- \sum_{i = 1}^d x_i^2}
        \intertext{%
            Since we are in polar coordinates, we can apply the Pythagorean
            theorem $d$ times in the argument of the exponential and write it
            as $- \vec x^2$. We also move the angular dependence to the front
            of the expression.
        }
        &= \int \dif \phi \prod_{k=1}^{d-2} \sin(\theta_k)^k
        \dif \theta_k \int \dif x \, x^{d-1} \exp\del{- \tens x^2}
        \intertext{%
            The first integral is just the surface area of the unit sphere. The
            second integral needs the substitution $z := x^2$ with $\dif z = 2x
            \dif x$ and therefore $\dif x = \dif z/ [2x]$.
        }
        &= \frac12 \int \dif \Omega_d \int \dif z \, z^{d/2-1} \exp\del{- z}
        \intertext{%
            And this $z$ integral is $\Gammaup(d/2)$.
        }
        &= \frac12 \Gammaup\del{\frac d2} \int \dif \Omega_d
    \end{align*}

    We reorder the parts and obtain the equation we wanted to show.
\end{proof}

Now that we are done with the proof, we realized that Equation~(5) from the
problem set will help with the sine integrals. We will leave it like it
currently is.

\begin{theorem}[Radial integration]
    \label{the:radial}

    The radial integral has the following solution:
    \[
        \int_0^\infty \dif P \frac{P^{d-1}}{[P^2 - A]^n}
        = \frac{[-1]^n \iup}{2^{d+1} \piup^d} \frac{\Gammaup\del{n-\frac d2}
        \Gammaup\del{\frac d2}}{\Gamma(n)} A^{\frac d2 - n}.
    \]
\end{theorem}

\begin{proof}
    The trick is to find the right substitution; twice.
    \Textcite[250]{Peskin/QFT/1995} give them in their account, they just do
    this for the $n = 2$ case and have a plus sign in front of their $A$. First
    one has to substitute
    \[
        \rho := P^2
        \eqnsep
        \dif \rho = 2 P \dif P.
    \]
    Using that, we obtain
    \begin{align*}
        \int_0^\infty \dif P \frac{P^{d-1}}{[P^2 - A]^n}
        &= \int_0^\infty \dif \rho \frac{\rho^{\frac d2-1}}{[\rho - A]^n}.
        \intertext{%
            Then we have to substitute again. This time we use
            \[
                z := \frac{A}{\rho - A}
                \eqnsep
                \dif z := - \frac{A}{[\rho - A]^2} \dif \rho
            \]
            Using this, we can rewrite the integral again. The bounds are
            transformed like $0 \mapsto 1$ and $\infty \mapsto 0$. We exchange
            the bounds and also get rid of the minus sign in the substation in
            one step.
        }
        &= \frac12 \int_0^1 \dif z \frac{[\rho - A]^2}{A}
        \frac{\rho^{\frac d2-1}}{[\rho - A]^n}
        \intertext{%
            Now we cancel the $[\rho - A]^2$. Then we still need to replace all
            the $\rho$ by $z$.
        }
        &= \frac12 \int_0^1 \dif z \frac{1}{A}
        \frac{\rho^{\frac d2-1}}{[\rho - A]^{n-2}}
        \intertext{%
            The denominator can be simplified like so:
            \[
                \frac{1}{[\rho - A]^{n-2}}
                = \frac{A^{n-2}}{[\rho - A]^{n-2}} A^{2-n}
                = z^{n-2} A^{2-n}.
            \]
            This simplifies the whole expression to
        }
        &= \frac12 \int_0^1 \dif z \, \rho^{\frac d2-1} A^{1-n} z^{n-2}.
        \intertext{%
            Then we need to replace the left $\rho$ with $z$. We have
            \[
                z = \frac{A}{\rho - A}
                \iff
                \rho - A = \frac Az
                \iff
                \rho = A \sbr{\frac 1z + 1}.
            \]
            Using that, we obtain
        }
        &= \frac12 \int_0^1 \dif z \,
        A^{\frac d2-n} \sbr{\frac 1z + 1}^{\frac d2-1} z^{n-2}.
        \intertext{%
            Then
            \[
                \frac 1z + 1 = \frac{1 + z}{z}
            \]
            allows us to cancel some of the $z$ and we obtain some $[1+z]$.
        }
        &= \frac12 \int_0^1 \dif z \,
        A^{\frac d2-n} [z+1]^{\frac d2-1} z^{n-1-\frac d2}
        \intertext{%
            At this point we can recognize the Beta function. We have
            \[
                a = \frac d2 - n
                \eqnsep
                -2 -a -b = n - \frac d2 -1
                \iff
                -a -b = - \sbr{\frac d2 - n} + 1
                \iff
                b = -1.
            \]
            We can then write the expression using the Beta function.
        }
        &= \frac12 A^{\frac d2-n} B\del{n - \frac d2 + 1, 0}
    \end{align*}
    Since $\Gammaup(0) \simeq \infty$, this result does not make any
    sense.

    \Textcite[250]{Peskin/QFT/1995} get usable $\Gammaup$ functions
    with their plus sign. We cannot do that since our denominator contains a
    minus sign. In a previous iteration of this problem we have mixed it up and
    had a minus sign in the denominator and a plus sign in the $z$
    substitution. With that we obtain
    \[
        \frac12 A^{\frac d2 - n} \frac{\Gammaup\del{n-\frac d2}
        \Gammaup\del{\frac d2}}{\Gamma(n)}.
    \]
    This is the result \textcite[250]{Peskin/QFT/1995} have, it does not lead
    to the desired expression in Equation~(2) on the problem set.

    Maybe it is possible to smuggle in imaginary unit into there and get a
    better result.
\end{proof}

With those theorems at hand, we can start to prove Equation~(2) from the
problem set. We start by replacing the measure in the integral. $\tens p^2$
does not depend on the angles and we can replace that with $P^2$. The angular
part from the measure is just the surface of the sphere which we know from
Theorem~\ref{the:surface}.
\begin{align*}
    \int \frac{\dif^d p}{[2 \piup]^d} \frac{1}{[\tens p^2 - A]^n}
    &= \frac{1}{[2 \piup]^d} \frac{2 \piup^{d/2}}{\Gammaup(d/2)} \int_0^\infty \dif P
    \frac{P^{d-1}}{[P^2 - A]^n}
    \intertext{%
        We can directly simplify the fraction.
    }
    &= \frac{1}{2^{d - 1} \piup^{d/2} \Gammaup(d/2)} \int_0^\infty \dif P
    \frac{P^{d-1}}{[P^2 - A]^n}
\end{align*}

Now we need some expression for this integral. Theorem~\ref{the:radial} should
give it us, but we were not able to prove it or figure out what it should
actually be. We do not have it, so we cannot further simplify this.

\subsection{Vertex rule}

In the $\phi^4$ theory we had so far, the interaction term was
\[
    \frac \lambda{4!} \phi^4.
\]
Now we have
\[
    \frac \lambda{4!} \mu^{2 [2-\omega]} \phi^4.
\]
Therefore the momentum space Feynman rule changes from $- \iup \lambda$ to $-
\iup \lambda \mu^{2 [2-\omega]}$.

\subsection{Tadpole diagram}

We iterate through the Feynman rules in momentum space.

\begin{itemize}
    \item
        The incoming and outgoing line give us a factor of one.

    \item
        The vertex gives $- \iup \lambda \mu^{2 [2-\omega]}$ as derived in the
        previous part of this problem.

    \item
        The loop will give us one propagator. This is
        \[
            \frac{\iup}{p^2 - m^2}.
        \]
        We have left the $+ \iup \epsilon$ out.

    \item
        We impose momentum conservation at each vertex. Let the incoming
        momentum be $q$. The loop momentum was already chosen to be $p$. From
        the total momentum conversation we know that the outgoing momentum has
        to be $q$ again. Keeping the direction in mind we have
        \[
            q + p - p - q = 0,
        \]
        which means that the momentum is conserved at the vertex if we do those
        choices. It also means that the momentum is conserved for any value of
        $p$ which is probably the cause for the singularities.

    \item
        The momentum $p$ is not determined, so we have to integrate over it.
        This gives another factor
        \[
            \int \frac{\dif^{2\omega} p}{[2\piup]^{2\omega}}.
        \]

    \item
        The symmetry factor of this diagram is two due to the loop.
\end{itemize}

Together we have
\begin{align*}
    \iup \mathscr M
    &= - \frac 12
    \iup \lambda \mu^{2 [2-\omega]}
    \int \frac{\dif^{2\omega} p}{[2\piup]^{2\omega}}
    \frac{\iup}{p^2 - m^2} \\
    &= \frac 12
    \lambda \mu^{2 [2-\omega]}
    \int \frac{\dif^{2\omega} p}{[2\piup]^{2\omega}}
    \frac{1}{p^2 - m^2}.
    \intertext{%
        We already tried and failed to derive the value of this integral. We
        can just insert Equation~(2) from the problem set. Here $n = 1$ and $d
        = 2 \omega$.
    }
    &= \frac 12
    \lambda \mu^{2 [2-\omega]}
    \frac{-\iup}{[4\piup]^\omega} \frac{\Gammaup(1 - \omega)}{\Gamma(1)} m^{2
    \omega - 1}
    \intertext{%
        Since $\Gammaup(1) = 1$ we can simplify further.
    }
    &= - 
    \frac{\iup \lambda \mu^{2 [2-\omega]}}{2 [4\piup]^\omega}
    \Gammaup(1 - \omega) m^{2 \omega - 1}
    \intertext{%
        We write the $\Gammaup$-function such that we can expand it in the next
        step.
    }
    &= - 
    \frac{\iup \lambda \mu^{2 [2-\omega]}}{2 [4\piup]^\omega}
    \Gammaup(- [\omega - 1] + \epsilon) m^{2 \omega - 1}
    \intertext{%
        Now we can use Equation~(9) from the problem set.
    }
    &=
    \frac{[-1]^\omega \iup \lambda \mu^{2 [2-\omega]}}{2 [4\piup]^\omega
    [\omega+1]!}
    \sbr{\frac 1\epsilon + \psi(\omega) + \mathrm O(\epsilon)} m^{2 \omega - 1}
\end{align*}

The problem statement asks to expand the $\Gammaup$-function. We did that. Now
we are supposed to expand the term $[\ldots]^{2 - \omega}$ neglecting terms
$\mathrm O(\omega -2)$ or higher. How is that “$\mathrm O(\omega -2)$” to be
interpreted? The only factor that looks like $[\ldots]^{2 - \omega}$ is the
term with the mass $\mu^2$. There is no power series at this point, so we
cannot expand the sum and ignore terms higher than this.

The Digramma function $\psi$ can be expanded in various ways, for instance the
harmonic series with the Euler-Mascheroni constant. That would also explain why
the constant Euler-Mascheroni $\gamma$ is given in the problem statement
although it does not appear at all in the formulas given there. The harmonic
series would have factors which are just numbers up to $1/[\omega-1]$. This
series expansion is finite and does not contain any powers of any dimensionful
quantity.

\subsection{Symmetry factor}

We hope that we have already included the symmetry factor in the previous
subsection.

\subsection{Fish diagram}

\subsection{Further expansion}

\end{document}

% vim: spell spelllang=en tw=79
