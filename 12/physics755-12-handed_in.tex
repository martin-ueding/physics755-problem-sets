\documentclass[11pt, english, fleqn, DIV=15, headinclude, BCOR=1cm]{scrartcl}

\usepackage[bibatend]{../header}
\usepackage{../my-boxes}

\usepackage{booktabs}
\usepackage{slashed}
\usepackage{simplewick}

\usepackage{feynmp-auto}
\usepackage{adjustbox}
\newenvironment{fmfwrapper}{\begin{adjustbox}{margin=5mm}}{\end{adjustbox}}

\usepackage{multicol}
\usepackage{lastpage}

\usepackage{tikz}
\usepackage{pgfplots}
\pgfplotsset{compat=1.10}

\newcommand\timeorder{\mathscr T}
\newcommand\normorder{\mathscr N}

\hypersetup{
    pdftitle=
}

\newcounter{totalpoints}
\newcommand\punkte[1]{#1\addtocounter{totalpoints}{#1}}

\newcounter{problemset}
\setcounter{problemset}{12}

\subject{physics755 -- Quantum Field Theory}
\ihead{physics755 -- Problem Set \arabic{problemset}}

\title{Problem Set \arabic{problemset}}

\newcommand\thegroup{Group Tuesday -- Ripunjay Acharya}

\publishers{\thegroup}
\ofoot{\thegroup}

\author{
    Martin Ueding \\ \small{\href{mailto:mu@martin-ueding.de}{mu@martin-ueding.de}}
    \and
    Oleg Hamm
}
\ifoot{Martin Ueding, Oleg Hamm}

\ohead{\rightmark}

\begin{document}

\maketitle

\vspace{3ex}

\begin{center}
    \begin{tabular}{rrr}
        problem & achieved points & possible points \\
        \midrule
        \nameref{homework:1} & & \punkte{15} \\
        \midrule
        total & & \arabic{totalpoints}
    \end{tabular}
\end{center}

\vspace{3ex}

\begin{center}
    \begin{small}
        This document consists of \pageref{LastPage} pages.
    \end{small}
\end{center}

\section{Dimensional regularization of $\phi^4$ theory at 1-loop}
\label{homework:1}

\subsection{Substitution}

We perform the Wick rotation by $p^0 \mapsto \iup p^0$.
\[
    \int \frac{\dif^d p}{[2\piup]^d} \frac{1}{[\tens p^2 - A]^n}
    = \int \frac{\dif^d p}{[2\piup]^d} \frac{1}{[[p^0]^2 - \vec p^2 - A]^n}
    \mapsto \int \frac{\iup \dif^d p}{[2\piup]^d} \frac{1}{[- [p^0]^2 - \vec
    p^2 - A]^n}.
\]

One thing \textcite{penrose-road_to_reality} warns to be careful about is that
the Wick rotation can turn a non-compact group like $\SO(1,3)$ into a compact
group like $\SO(4)$. Then one works with the compact group and goes back. One
should be careful.

\subsection{Proof of identity}

This problem is also covered by \textcite[249--250]{Peskin/QFT/1995}.

We will divide this into a couple parts, just like they also do.

\begin{theorem}[Surface of unit sphere]
    \label{the:surface}

    The surface of the $d$-dimensional unit sphere, i.e. the one where the
    surface itself has dimension $d-1$ is given by
    \[
        \int \dif \Omega_d = \frac{2 \piup^{\frac d2}}{\Gammaup\del{\frac d2}}.
    \]
\end{theorem}

\begin{proof}
    The measure is given by Equation~(3) on the problem set:
    \[
        d^d p = P^{d-1} dPd\phi \sin \theta_1 d\theta_1 \sin^2\theta_2 d
        \theta_2 \ldots \sin^{d-2} \theta_{d-1} d\theta_{d-2}.
    \]
    We think that this is unnecessarily hard to read. We write it without
    the ellipsis and upright exterior derivative operators like so:
    \[
        \dif^d p = P^{d-1} \dif P \dif \phi
        \prod_{k=1}^{d-2} \sin(\theta_k)^k \dif \theta_k
    \]

    The unit sphere has $P = 1$. We will set $P$ to unity. Then we can
    integrate over it and will end up with the area of the unit sphere.
    \[
        \int \dif \Omega_d = \int_0^{2\piup} \dif \phi
        \prod_{k=1}^{d-2} \int_0^\piup \sin(\theta_k)^k \dif \theta_k
    \]
    The $\phi$ integral will just give $2\piup$, but the other factors are not
    that easily computed. Mathematica can compute the $\theta_k$ integral and
    gives expressions with multiple $\Gammaup$-functions.

    Since we are not able to perform the integration of the factors which yield
    hypergeometric ${}_2F_1$ functions by hand, we will use the derivation by
    \textcite[249]{Peskin/QFT/1995}. They start by using the simplest Gaussian
    integral:
    \begin{align*}
        [\sqrt{\piup}]^d
        &= \sbr{\int \dif x \, \exp(-x^2)}^d
        \intertext{%
            Then they just factor out the integral and chose $\R^d$ as the
            domain of integration. The exponential factors are grouped together
            such that there is a sum in the integration.
        }
        &= \int \dif^d x \, \exp\del{- \sum_{i = 1}^d x_i^2}
        \intertext{%
            Now they change into those generalized polar coordinates with $P =
            x$.
        }
        &= \int x^{d-1} \dif x \dif \phi \prod_{k=1}^{d-2} \sin(\theta_k)^k
        \dif \theta_k \exp\del{- \sum_{i = 1}^d x_i^2}
        \intertext{%
            Since we are in polar coordinates, we can apply the Pythagorean
            theorem $d$ times in the argument of the exponential and write it
            as $- \vec x^2$. We also move the angular dependence to the front
            of the expression.
        }
        &= \int \dif \phi \prod_{k=1}^{d-2} \sin(\theta_k)^k
        \dif \theta_k \int \dif x \, x^{d-1} \exp\del{- \tens x^2}
        \intertext{%
            The first integral is just the surface area of the unit sphere. The
            second integral needs the substitution $z := x^2$ with $\dif z = 2x
            \dif x$ and therefore $\dif x = \dif z/ [2x]$.
        }
        &= \frac12 \int \dif \Omega_d \int \dif z \, z^{d/2-1} \exp\del{- z}
        \intertext{%
            And this $z$ integral is $\Gammaup(d/2)$.
        }
        &= \frac12 \Gammaup\del{\frac d2} \int \dif \Omega_d
    \end{align*}

    We reorder the parts and obtain the equation we wanted to show.
\end{proof}

Now that we are done with the proof, we realized that Equation~(5) from the
problem set will help with the sine integrals. We will leave it like it
currently is.

\begin{theorem}[Radial integration]
    \label{the:radial}

    The radial integral has the following solution:
    \[
        \int_0^\infty \dif P \frac{P^{d-1}}{[P^2 - A]^n}
        = \frac12 A^{\frac d2 - n} \frac{\Gammaup\del{n-\frac d2}
        \Gammaup\del{\frac d2}}{\Gamma(n)}.
    \]
\end{theorem}

\begin{proof}
    The trick is to find the right substitution; twice.
    \Textcite[250]{Peskin/QFT/1995} give them in their account, they just do
    this for the $n = 2$ case. First one has to substitute
    \[
        \rho := P^2
        \eqnsep
        \dif \rho = 2 P \dif P.
    \]
    Using that, we obtain
    \begin{align*}
        \int_0^\infty \dif P \frac{P^{d-1}}{[P^2 - A]^n}
        &= \int_0^\infty \dif \rho \frac{\rho^{\frac d2-1}}{[\rho - A]^n}.
        \intertext{%
            Then we have to substitute again. This time we use
            \[
                z := \frac{A}{\rho + A}
                \eqnsep
                \dif z := - \frac{A}{[\rho + A]^2} \dif \rho
            \]
            Using this, we can rewrite the integral again. The bounds are
            transformed like $0 \mapsto 1$ and $\infty \mapsto 0$. We exchange
            the bounds and also get rid of the minus sign in the substation in
            one step.
        }
        &= \frac12 \int_0^1 \dif z \frac{[\rho + A]^2}{A}
        \frac{\rho^{\frac d2-1}}{[\rho - A]^n}
        \intertext{%
            Now we cancel the $[\rho + A]^2$. Then we still need to replace all
            the $\rho$ by $z$.
            % TODO Write more steps here, if there is time left.
        }
        &= \frac12 \int_0^1 \dif z \, A^{\frac d2-n} [1-z]^{\frac d2 -1} z^{n-\frac d2-1}
        \intertext{%
            At this point we can recognize the Beta function.
        }
        &= \frac12 A^{\frac d2-n} B\del{n - \frac d2, \frac d2}
        = \frac12 A^{\frac d2 - n} \frac{\Gammaup\del{n-\frac d2}
        \Gammaup\del{\frac d2}}{\Gamma(n)}.
    \end{align*}
    And that is the final result we want.
\end{proof}

With those theorems at hand, we can start to prove Equation~(2) from the
problem set. We start by replacing the measure in the integral. $\tens p^2$
does not depend on the angles and we can replace that with $P^2$. The angular
part from the measure is just the surface of the sphere which we know from
Theorem~\ref{the:surface}.
\begin{align*}
    \int \frac{\dif^d p}{[2 \piup]^d} \frac{1}{[\tens p^2 - A]^n}
    &= \frac{1}{[2 \piup]^d} \frac{2 \piup^{d/2}}{\Gammaup(d/2)} \int_0^\infty \dif P
    \frac{P^{d-1}}{[P^2 - A]^n}
    \intertext{%
        We can directly simplify the fraction.
    }
    &= \frac{1}{2^{d - 1} \piup^{d/2} \Gammaup(d/2)} \int_0^\infty \dif P
    \frac{P^{d-1}}{[P^2 - A]^n}
    \intertext{%
        Now can now use Theorem~\ref{the:radial} to replace the integral.
    }
    &= \frac{1}{[4\piup]^{d/2}} A^{\frac d2 - n} \frac{\Gammaup\del{n-\frac d2}
    }{\Gamma(n)}.
\end{align*}

This matches the result given by \textcite[250]{Peskin/QFT/1995}, which
means that we copied their results well.

% FIXME P&S have a plus in front of the Delta while we have a minus in front of
% our A. Go through this again and fix the signs.

\end{document}

% vim: spell spelllang=en tw=79
