\documentclass[11pt, english, fleqn, DIV=15, headinclude, BCOR=1cm]{scrartcl}

\usepackage[bibatend]{../header}
\usepackage{../my-boxes}

\usepackage{booktabs}

\hypersetup{
    pdftitle=
}

\newcounter{totalpoints}
\newcommand\punkte[1]{#1\addtocounter{totalpoints}{#1}}

\newcounter{problemset}
\setcounter{problemset}{4}

\subject{physics755 -- Quantum Field Theory}
\ihead{physics755 -- Problem Set \arabic{problemset}}

\title{Problem Set \arabic{problemset}}

\newcommand\thegroup{Group Tuesday -- Ripunjay Acharya}

\publishers{\thegroup}
\ofoot{\thegroup}

\author{
    Martin Ueding \\ \small{\href{mailto:mu@martin-ueding.de}{mu@martin-ueding.de}}
    \and
    Oleg Hamm
}
\ifoot{Martin Ueding, Oleg Hamm}

\ohead{\rightmark}

\begin{document}

\maketitle

\vspace{3ex}

\begin{center}
    \begin{tabular}{rrr}
        problem & achieved points & possible points \\
        \midrule
        \nameref{homework:1} & & \punkte{15} \\
        \midrule
        total & & \arabic{totalpoints}
    \end{tabular}
\end{center}

\section{Lorentz algebra 2}
\label{homework:1}

\subsection{Rotations and boosts}

\paragraph{Rotation}

The generators were derived on the previous problem set. They are:
\[
    J_{\rho\sigma} = x_{[\sigma} \partial_{\rho]}
\]
where we have used the antisymmetrization notation in the non-idempotent form
since that seems to be used in this class. This uses the Mathematician's
convention of real antisymmetric generators. To get the Physicists's notation,
we add $-\iup$ to the generators and $\iup$ into the exponential map.
\[
    J_{\rho\sigma} = \iup x_{[\rho} \partial_{\sigma]}
\]
Then $L^3$ is given by $\iup x_{[1} \partial_{2]}$. We found it easier to start
with the representation on $\vec x'$ and show that the generator of that
passive transformation is the same as the one given here. So we start with the
rotation $\tens R(\Theta)$ around the $x^3$-axis:
\begin{align*}
    \Phi(x') &= \Phi\del{\tens R^{-1}(\Theta) \cdot \vec x}
    \intertext{%
        Then we can expand this around $\Theta = 0$.
    }
    &= \sum_{n=0}^\infty \frac{1}{n!} \odx{n}{\Phi}{\Theta}(\vec x) \Theta^n
    \intertext{%
        Next we write this in operator form,
    }
    &= \sum_{n=0}^\infty \frac{1}{n!} \Theta^n \odx{n}{}{\Theta} \Phi(\vec x),
    \intertext{%
        and then as an exponential:
    }
    &= \exp\del{\Theta \od{}{\Theta}} \Phi(\vec x),
\end{align*}
We see that the (Physicist's) generator is given by
\begin{align*}
    T &= - \iup \od{}\Theta.
    \intertext{%
        Now we need to apply the chain rule for this expression.
    }
    &= - \iup \od{x'^\mu}{\Theta}(0) \pd{}{x'^\mu}
\end{align*}
The $\vec x'$ is given by
\[
    \vec x' = \tens R^{-1}(\Theta) \cdot \vec x.
\]
The explicit form of this inverse passive transformation looks like the active
rotation around the $x^3$-axis:
\[
    \tens R^{-1}(\Theta) =
    \begin{pmatrix}
        1 & & & \\
          & \cos(\Theta) & -\sin(\Theta) & \\
          & \sin(\Theta) & \cos(\Theta) & \\
          & & & 1 \\
    \end{pmatrix}.
\]
The derivative with respect to $\Theta$ at $\Theta = 0$ is then given by
\[
    \pd{\tens R^{-1}}\Theta (0) =
    \begin{pmatrix}
        0 & & & \\
          & 0 & -1 & \\
          & 1 & 0 & \\
          & & & 0 \\
    \end{pmatrix}.
\]
Contracting this derivative of $\tens R$ with $\vec x$ to give the derivative
of $\vec x'$ gives
\[
    \od{x'^\mu}{\Theta} (0) =
    \begin{pmatrix}
        0 \\ -x^2 \\ x^1 \\ 0
    \end{pmatrix}.
\]
As a last step with put that into the generator expression:
\[
    T = - \iup \od{x'^\mu}{\Theta}(0) \pd{}{x'^\mu}
    = - \iup 
    \begin{pmatrix}
        0 \\ -x^2 \\ x^1 \\ 0
    \end{pmatrix}^\mu \partial_\mu
    = - \iup \sbr{ x^1 \partial_2 - x^2 \partial_1}
    = \iup \sbr{ x_1 \partial_2 - x_2 \partial_1}.
\]
This is the same generator, so the transformation should be the same.

\paragraph{Boost}

We have
\[
    K^3 = J^{03} = \iup x_{[0} \partial_{3]} =  \iup \sbr{x_0 \partial_3 - x_3
    \partial_0}
\]
for the generator in the functional representation.

The same derivation as above can be applied to the boost as well. The general
inverse passive boost is given by:
\[
    \tens B(\eta) =
    \begin{pmatrix}
        \cosh(\eta) & & & -\sinh(\eta) \\
                    & 1 & & \\
                    & & 1 & \\
        -\sinh(\eta) & & & \cosh(\eta)
    \end{pmatrix}
    .
\]
From this, the generator of the matrix is given by:
\[
    - \iup
    \begin{pmatrix}
        0 & & & -1 \\
          & 0 & & \\
          & & 0 & \\
        -1 & & & 0
    \end{pmatrix}.
\]
Contracting that with $x^\mu$ we get:
\[
    \begin{pmatrix}
        -x^3 \\ 0 \\ 0 \\ -x^0
    \end{pmatrix}.
\]
And contracting that with $\vec \partial$ we get
\[
    \iup [x^0 \partial_3 + x^3 \partial_0]
    = 
    \iup [x_0 \partial_3 - x_3 \partial_0].
\]
So this checks out.

\subsection{Commutators}

The definition of the $L^i$ is
\[
    L^i = \frac12 \epsilon^{ijk} J_{jk}.
\]
This can be written as $L = \hodge J$ and therefore also as $\hodge L = J$
which in components is
\[
    \epsilon_{ijk} L^k = P_{ij}.
\]

We are not sure how many indices we may use. We think that it is not possible
to show this without indices at all. We will use the commutator $[x^i,
\partial_j] = \deltaup^i_j$ instead of the full commutator of the $J$ that is
given on the problem set.

\paragraph{Commutations among $L^i$}

\begin{align*}
    \sbr{L^i, L^a}
    &= \epsilon^{ijk} \epsilon^{abc} \sbr{x_j \partial_k, x_b \partial_c}
    \intertext{%
        Now we can use the commutator identity to split this.
    }
    &= \epsilon^{ijk} \epsilon^{abc} \sbr{
        x_j \sbr{\partial_k, x_b \partial_c}
        + \sbr{x_j, x_b \partial_c} \partial_k
    }
    \intertext{%
        We need to split this up again, so we flip the signs in order to use
        the same identity.
    }
    &= - \epsilon^{ijk} \epsilon^{abc} \sbr{
        x_j \sbr{x_b \partial_c, \partial_k}
        + \sbr{x_b \partial_c, x_j} \partial_k
    }
    \intertext{%
        We expand again.
    }
    &= - \epsilon^{ijk} \epsilon^{abc} \sbr{
        x_j x_b \sbr{\partial_c, \partial_k}
        + x_j \sbr{x_b, \partial_k} \partial_c
        + x_b \sbr{\partial_c, x_j} \partial_k
        + \sbr{x_b, x_j} \partial_k \partial_c
    }
    \intertext{%
        We use the commutator of $\vec x$ and $\vec \partial$ mentioned above.
    }
    &= - \epsilon^{ijk} \epsilon^{abc} \sbr{
        x_j \deltaup_{bk} \partial_c
        - x_b \deltaup_{cj} \partial_k
    }
    \intertext{%
        We absorb the minus sign in the ordering within the bracket.
    }
    &= \epsilon^{ijk} \epsilon^{abc} \sbr{
        x_b \deltaup_{cj} \partial_k
        - x_j \deltaup_{bk} \partial_c
    }
    \intertext{%
        Then we factor out.
    }
    &= \epsilon^{ijk} \epsilon^{abc} x_b \deltaup_{cj} \partial_k
    - \epsilon^{ijk} \epsilon^{abc} x_j \deltaup_{bk} \partial_c
    \intertext{%
        Next we can apply the Kronecker symbol.
    }
    &= \epsilon^{ijk} \epsilon^{ab}{}_j x_b \partial_k
    - \epsilon^{ijk} \epsilon^a{}_k{}^c x_j \partial_c
    \intertext{%
        We reorder the indices such that the last indices are contracted.
    }
    &= \epsilon^{kij} \epsilon^{ab}{}_j x_b \partial_k
    - \epsilon^{ijk} \epsilon^{ca}{}_k x_j \partial_c
    \intertext{%
        Then we can contract the two Levi-Civita symbols.
    }
    &= \sbr{\deltaup^{ka} \deltaup^{ib} - \deltaup^{kb} \deltaup^{ia}}
    x_b \partial_k
    - \sbr{\deltaup^{ic} \deltaup^{ja} - \deltaup^{ia} \deltaup^{jc}}
    x_j \partial_c
    \intertext{%
        Then we get four terms.
    }
    &= \deltaup^{ka} \deltaup^{ib} x_b \partial_k
    - \deltaup^{kb} \deltaup^{ia} x_b \partial_k
    - \deltaup^{ic} \deltaup^{ja} x_j \partial_c
    + \deltaup^{ia} \deltaup^{jc} x_j \partial_c
    \intertext{%
        And then we can contract those.
    }
    &= x_i \partial_a
    - \deltaup^{ia} x_f \partial_f
    - x_a \partial_i
    + \deltaup^{ia} x_f \partial_f
    \intertext{%
        The second and third term cancel, the remainder then is just
    }
    &= x^{[i} \partial^{a]}.
    \intertext{%
        And we can write this as
    }
    &= - \iup \epsilon^{iak} L^k.
\end{align*}
The structure constant should be just $\epsilon_{ijk}$, not with a minus sign
to match the algebra $\mathfrak{su}(2)$.

\paragraph{Commutations among $K^i$}

% TODO

\paragraph{Commutations among $J_+^i$}

% TODO

\paragraph{Commutations among $J_-^i$}

% TODO

\paragraph{Commutation of $J_+^i$ and $J_-^i$}

\begin{align*}
    \sbr{\vec J_+, \vec J_-}^i
    &= \sbr{\vec L + \iup \vec K, \vec L - \iup \vec K}^i
    \intertext{%
        We use the bilinearity of the commutator.
    }
    &= \sbr{\vec L, \vec L}^i
    + \sbr{\vec L, - \iup \vec K}^i
    + \sbr{\iup \vec K, \vec L}^i
    + \sbr{\iup \vec K, - \iup \vec K}^i
    \intertext{%
        Then we extract the factors to the front.
    }
    &= \sbr{\vec L, \vec L}^i
    - \iup \sbr{\vec L, \vec K}^i
    + \iup \sbr{\vec K, \vec L}^i
    + \sbr{\vec K,\vec K}^i
    \intertext{%
        The second and first term are the same, so we can join them together.
        The first and last commutator contain the exact same element on the
        left and right side, so they are zero.
    }
    &= - 2 \iup \sbr{\vec L, \vec K}^i
    \intertext{%
        Now we can expand the definitions.
    }
    &= 2 \iup \sbr{\frac12 \epsilon^{ijk} P_{jk}, P^{0i}}
    \intertext{%
        We also expand the $P$.
    }
    &= 2 \iup \sbr{\frac12 \epsilon^{ijk} x_{[j} \partial_{k]}, x^{[0}
    \partial^{i]}}
    \intertext{%
        The Levi-Civita symbol acting on the antisymmetric tensor in the first
        argument gives a factor of 2. We expand the second argument
    }
    &= 2 \iup \epsilon^{ijk} \sbr{x_{j} \partial_{k}, x^{0} \partial^{i} - x^i
    \partial^0}
    \intertext{%
        Now use the bilinearity again.
    }
    &= 2 \iup \epsilon^{ijk} \sbr{
        \sbr{x_{j} \partial_{k}, x^{0} \partial^{i}}
        - \sbr{x_{j} \partial_{k}, \partial^0 x^i}
    }
    \intertext{%
        Since there is only one time like component in each term, we can pull
        that out of the commutator.
    }
    &= 2 \iup \epsilon^{ijk} \sbr{
        \sbr{x_{j} \partial_{k}, \partial^{i}} x^{0}
        - \sbr{x_{j} \partial_{k}, x^i} \partial^0
    }
    \intertext{%
        Then we can also pull out the $\partial_k$ in the first term and the
        $x_j$ in the second.
    }
    &= 2 \iup \epsilon^{ijk} \sbr{
        \sbr{x_{j}, \partial^{i}} \partial_{k} x^{0}
        - x_{j} \sbr{\partial_{k}, x^i} \partial^0
    }
    \intertext{%
        The commutators are simple, we now have:
    }
    &= 2 \iup \epsilon^{ijk} \sbr{
        \deltaup^i_j \partial_{k} x^{0} + x_{j} \deltaup^i_k \partial^0
    }.
    \intertext{%
        We can now replace the indices $j$ and $k$ with $i$ in each term. There
        is no summation on the $i$s!
    }
    &= 2 \iup \epsilon^{iik} \deltaup^i_j x^{0}
    + 2 \iup \epsilon^{iji} x_{j} \partial^0
    \intertext{%
        Since the Levi-Civita symbol is antisymmetric and the index $i$ appears
        twice, this is just zero.
    }
    &= 0
\end{align*}
Therefore $\vec J_+$ and $\vec J_-$ commute.

\subsection{Left and right-handed spinors}

\paragraph{Dimension}

Using the given formula, the dimension is
\[
    \sbr{2 \cdot \frac12 + 1} [2 \cdot 0 + 1] = 2 \cdot 1 = 2.
\]
Those matrices act on vectors from $\C^2$, which has 2 (complex) dimensions.

\paragraph{Action on left/right-handed spinor}

For a simple scalar field we had the Lorentz transformation given in
Equation~(4) on the problem set:
\[
    \Phi \mapsto \exp(- \iup \vec \Theta \cdot \vec L - \iup \vec \beta \cdot
    \vec K) \, \Phi
\]
Using the definitions of $\vec J_+$ and $\vec J_-$, we can rewrite $\vec L$ and
$\vec K$ in terms of those:
\[
    \vec L = \vec J_+ + \vec J_-
    \eqnsep
    \vec K = \frac1\iup \sbr{\vec J_+ - \vec J_-}
\]
This lets us rewrite the mapping.
\begin{align*}
    \Phi
    &\mapsto
    \exp(- \iup \vec \Theta \cdot \vec L - \iup \vec \beta \cdot
    \vec K) \, \Phi \\
    &= \exp\del{-\iup \vec\Theta\cdot\sbr{\vec J_+ + \vec J_-} - \vec
    \beta\cdot\sbr{\vec J_+ - \vec J_-}} \, \Phi \\
    \intertext{%
        We have shown earlier that the $\vec J_+$ and $\vec J_-$ commute with
        each other. This allows us to split the exponent into two parts.
    }
    &= \exp\del{-\iup \vec\Theta\cdot\vec J_+ - \vec \beta\cdot\vec J_+}
    \exp\del{-\iup \vec\Theta\cdot\vec J_- + \vec \beta\cdot \vec J_-}
    \, \Phi
\end{align*}
There are two parts here, belonging to different representations of the Lorentz
group. We therefore have for the left and right-handed spinors:
\[
    \psi_\text L \mapsto \exp\del{-\iup \vec\Theta\cdot\vec J_- + \vec
    \beta\cdot \vec J_-} \,\psi_\text L
    \eqnsep
    \psi_\text R \mapsto \exp\del{-\iup \vec\Theta\cdot\vec J_+ - \vec
    \beta\cdot\vec J_+} \,\psi_\text R
\]

\end{document}

% vim: spell spelllang=en tw=79
