\documentclass[11pt, english, fleqn, DIV=15, headinclude, BCOR=1cm]{scrartcl}

\usepackage[bibatend]{../header}
\usepackage{../my-boxes}

\usepackage{booktabs}

\hypersetup{
    pdftitle=
}

\newcounter{totalpoints}
\newcommand\punkte[1]{#1\addtocounter{totalpoints}{#1}}

\newcounter{problemset}
\setcounter{problemset}{4}

\subject{physics755 -- Quantum Field Theory}
\ihead{physics755 -- Problem Set \arabic{problemset}}

\title{Problem Set \arabic{problemset}}

\newcommand\thegroup{Group Tuesday -- Ripunjay Acharya}

\publishers{\thegroup}
\ofoot{\thegroup}

\author{
    Martin Ueding \\ \small{\href{mailto:mu@martin-ueding.de}{mu@martin-ueding.de}}
    \and
    Oleg Hamm
}
\ifoot{Martin Ueding}

\ohead{\rightmark}

\begin{document}

\maketitle

\vspace{3ex}

\begin{center}
    \begin{tabular}{rrr}
        problem & achieved points & possible points \\
        \midrule
        \nameref{homework:1} & & \punkte{15} \\
        \midrule
        total & & \arabic{totalpoints}
    \end{tabular}
\end{center}

\section{Lorentz algebra 2}
\label{homework:1}

\subsection{Rotations and boosts}

\paragraph{Rotation}

The generators were derived on the previous problem set. They are:
\[
    J_{\rho\sigma} = x_{[\sigma} \partial_{\rho]}
\]
where we have used the antisymmetrization notation in the non-idempotent form
since that seems to be used in this class. This uses the Mathematician's
convention of real antisymmetric generators. To get the Physicists's notation,
we add $-\iup$ to the generators and $\iup$ into the exponential map.
\[
    J_{\rho\sigma} = \iup x_{[\rho} \partial_{\sigma]}
\]
Then $L^3$ is given by $\iup x_{[1} \partial_{2]}$. We found it easier to start
with the representation on $\vec x'$ and show that the generator of that
passive transformation is the same as the one given here. So we start with the
rotation $\tens R(\Theta)$ around the $x^3$-axis:
\begin{align*}
    \Phi(x') &= \Phi\del{\tens R^{-1}(\Theta) \cdot \vec x}
    \intertext{%
        Then we can expand this around $\Theta = 0$.
    }
    &= \sum_{n=0}^\infty \frac{1}{n!} \odx{n}{\Phi}{\Theta}(\vec x) \Theta^n
    \intertext{%
        Next we write this in operator form,
    }
    &= \sum_{n=0}^\infty \frac{1}{n!} \Theta^n \odx{n}{}{\Theta} \Phi(\vec x),
    \intertext{%
        and then as an exponential:
    }
    &= \exp\del{\Theta \od{}{\Theta}} \Phi(\vec x),
\end{align*}
We see that the (Physicist's) generator is given by
\begin{align*}
    T &= - \iup \od{}\Theta.
    \intertext{%
        Now we need to apply the chain rule for this expression.
    }
    &= - \iup \od{x'^\mu}{\Theta}(0) \pd{}{x'^\mu}
\end{align*}
The $\vec x'$ is given by
\[
    \vec x' = \tens R^{-1}(\Theta) \cdot \vec x.
\]
The explicit form of this inverse passive transformation looks like the active
rotation around the $x^3$-axis:
\[
    \tens R^{-1}(\Theta) =
    \begin{pmatrix}
        1 & & & \\
          & \cos(\Theta) & -\sin(\Theta) & \\
          & \sin(\Theta) & \cos(\Theta) & \\
          & & & 1 \\
    \end{pmatrix}.
\]
The derivative with respect to $\Theta$ at $\Theta = 0$ is then given by
\[
    \pd{\tens R^{-1}}\Theta (0) =
    \begin{pmatrix}
        0 & & & \\
          & 0 & -1 & \\
          & 1 & 0 & \\
          & & & 0 \\
    \end{pmatrix}.
\]
Contracting this derivative of $\tens R$ with $\vec x$ to give the derivative
of $\vec x'$ gives
\[
    \od{x'^\mu}{\Theta} (0) =
    \begin{pmatrix}
        0 \\ -x^2 \\ x^1 \\ 0
    \end{pmatrix}.
\]
As a last step with put that into the generator expression:
\[
    T = - \iup \od{x'^\mu}{\Theta}(0) \pd{}{x'^\mu}
    = - \iup 
    \begin{pmatrix}
        0 \\ -x^2 \\ x^1 \\ 0
    \end{pmatrix}^\mu \partial_\mu
    = - \iup \sbr{ x^1 \partial_2 - x^2 \partial_1}
    = \iup \sbr{ x_1 \partial_2 - x_2 \partial_1}.
\]
This is the same generator, so the transformation should be the same.

\paragraph{Boost}

We have
\[
    K^3 = J^{03} = \iup x_{[0} \partial_{3]} =  \iup \sbr{x_0 \partial_3 - x_3
    \partial_0}
\]
for the generator in the functional representation.

The same derivation as above can be applied to the boost as well. The general
inverse passive boost is given by:
\[
    \tens B(\eta) =
    \begin{pmatrix}
        \cosh(\eta) & & & -\sinh(\eta) \\
                    & 1 & & \\
                    & & 1 & \\
        -\sinh(\eta) & & & \cosh(\eta)
    \end{pmatrix}
    .
\]
From this, the generator of the matrix is given by:
\[
    - \iup
    \begin{pmatrix}
        0 & & & -1 \\
          & 0 & & \\
          & & 0 & \\
        -1 & & & 0
    \end{pmatrix}.
\]
Contracting that with $x^\mu$ we get:
\[
    \begin{pmatrix}
        -x^3 \\ 0 \\ 0 \\ -x^0
    \end{pmatrix}.
\]
And contracting that with $\vec \partial$ we get
\[
    \iup [x^0 \partial_3 + x^3 \partial_0]
    = 
    \iup [x_0 \partial_3 - x_3 \partial_0].
\]
So this checks out.

\end{document}

% vim: spell spelllang=en tw=79
