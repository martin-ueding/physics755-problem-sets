\documentclass[11pt, english, fleqn, DIV=15, headinclude, BCOR=1cm]{scrartcl}

\usepackage[bibatend]{../header}
\usepackage{../my-boxes}

\usepackage{booktabs}
\usepackage{slashed}
\usepackage{simplewick}

\usepackage{feynmp-auto}
\usepackage{adjustbox}
\newenvironment{fmwrapper}{\begin{adjustbox}{margin=5mm}}{\end{adjustbox}}

\newcommand\timeorder{\mathscr T}
\newcommand\normorder{\mathscr N}

\hypersetup{
    pdftitle=
}

\newcounter{totalpoints}
\newcommand\punkte[1]{#1\addtocounter{totalpoints}{#1}}

\newcounter{problemset}
\setcounter{problemset}{8}

\subject{physics755 -- Quantum Field Theory}
\ihead{physics755 -- Problem Set \arabic{problemset}}

\title{Problem Set \arabic{problemset}}

\newcommand\thegroup{Group Tuesday -- Ripunjay Acharya}

\publishers{\thegroup}
\ofoot{\thegroup}

\author{
    Martin Ueding \\ \small{\href{mailto:mu@martin-ueding.de}{mu@martin-ueding.de}}
    \and
    Oleg Hamm
}
\ifoot{Martin Ueding, Oleg Hamm}

\ohead{\rightmark}

\begin{document}

\maketitle

\vspace{3ex}

\begin{center}
    \begin{tabular}{rrr}
        problem & achieved points & possible points \\
        \midrule
        \nameref{homework:1} & & \punkte{8} \\
        \nameref{homework:2} & & \punkte{6} \\
        \midrule
        total & & \arabic{totalpoints}
    \end{tabular}
\end{center}

\section{Wick gymnastics}
\label{homework:1}

In Equation~(1) the symbol $T$ is used as time ordering operator and the bounds
of the integration. This does not make too much sense. We will use $\timeorder$
for time ordering like on the last sheet.

\subsection{First order}

\paragraph{Linearization}

We are supposed to compute the numerator to first order in $\lambda$. The
numerator is given as
\begin{gather*}
    \bra 0
    \timeorder\del{
        \phi_\mathrm I(x_1)
        \phi_\mathrm I(x_2)
        \phi_\mathrm I(x_3)
        \phi_\mathrm I(x_4)
        \exp\del {
            - \iup \int_{-T}^T \dif t \, H_\mathrm I(t)
        }
    }
    \ket 0
    \intertext{%
        We expand the Hamiltonian in the interacting picture.
    }
    = \bra 0
    \timeorder\del{
        \phi_\mathrm I(x_1)
        \phi_\mathrm I(x_2)
        \phi_\mathrm I(x_3)
        \phi_\mathrm I(x_4)
        \exp\del {
            - \frac{\iup \lambda}{4!} \int_{-T}^T \dif t
            \int_{\R^3} \dif^3 x \, \phi^4
        }
    }
    \ket 0
    \intertext{%
        Since we are only interested in terms linear in $\lambda$, we linearize
        the exponential and use the linearity of the $L^2$ scalar product.
    }
    = \bra 0
    \timeorder\del{
        \phi_\mathrm I(x_1)
        \phi_\mathrm I(x_2)
        \phi_\mathrm I(x_3)
        \phi_\mathrm I(x_4)
    }
    \ket 0 \\\quad
    - \frac{\iup \lambda}{4!} \bra 0
    \timeorder\del{
        \int_{-T}^T \dif t
        \int_{\R^3} \dif^3 x \,
        \phi_\mathrm I(x_1)
        \phi_\mathrm I(x_2)
        \phi_\mathrm I(x_3)
        \phi_\mathrm I(x_4)
        \phi(x) \phi(x) \phi(x) \phi(x)
    }
    \ket 0
    + \mathrm O(\lambda^2)
    \intertext{%
        We use Wick's theorem in the first summand and convert the time
        ordering into normal ordering and contractions. Only fully contracted
        terms survive the vacuum (bad pun intended). We therefore only have to
        keep the three different contractions. The integrals and the time
        ordering should commute since the time ordering only applies to the
        integrands, not the integral signs.
    }
    = D(x_1 - x_3) D(x_2 - x_4) + D(x_1 - x_4) D(x_2 - x_3)
    + D(x_1 - x_2) D(x_3 - x_4)
    \\\quad
    - \frac{\iup \lambda}{4!}
    \int_{-T}^T \dif t
    \int_{\R^3} \dif^3 x \,
    \bra 0
    \timeorder\del{
        \phi_\mathrm I(x_1)
        \phi_\mathrm I(x_2)
        \phi_\mathrm I(x_3)
        \phi_\mathrm I(x_4)
        \phi(x) \phi(x) \phi(x) \phi(x)
    }
    \ket 0
    + \mathrm O(\lambda^2)
\end{gather*}

\paragraph{Counting contractions}

We have eight field operators in the time ordering. There are $7!! =
105$ possible contractions, which are not necessarily unique. We
organize those contractions into three groups, grouped by the number of
self-contractions of the $\phi^4$ term.

\begin{itemize}
    \item 
        The group with no self-contractions contains $4! = 24$ elements and
        they are all equal since the $\phi(x)$ terms are all equal.

    \item
        If we have one self contraction there are $4^2 \cdot 3^2 / 2 = 72$
        possible contractions. We get to this number like so: In the beginning
        we have four $\phi_\mathrm I$ and four $\phi$ fields to choose from.
        This gives $4^2$. For the second contraction we can choose another
        $\phi_\mathrm I$ and one $\phi$ field, we have three of each kind. This
        gives $3^2$ possibilities. The two remaining $\phi_\mathrm I$ fields
        are contracted with each other and so are the remaining $\phi$ fields.
        We have over counted by a factor of two since the order of performing
        the two contractions does not matter. We could get around this division
        and say that the second contraction must chose a $x_j$ such that $j >
        i$ when $x_i$ was chosen in the first contraction. We will use that
        below.

    \item
        If we contract all the $\phi$ terms among themselves, we have $3!! = 3$
        possibilities to contract the $\phi_\mathrm I$ terms. Since there are
        also three possibilities to contract the $\phi$ fields independently,
        we have $3^3 = 9$ contractions in this group.
\end{itemize}

Together, those are the 105 contractions that we have calculated earlier. Some
of them have the same values, some are distinct. They only have the same value
if they only differ in the order of the contractions of the $\phi$ fields.

We have computed the $\lambda^0$ order already and we will focus of the time
ordering of the $\lambda^1$ order inside of the integral. We are not supposed
to carry out integrations, so looking at the integrand should be sufficient.
This problem demands a considerable dose of consistency, otherwise we would be
lost. We call the groups with $i$ self-contractions $C_i$:
\[
    \bra 0
    \timeorder\del{
        \phi_\mathrm I(x_1)
        \phi_\mathrm I(x_2)
        \phi_\mathrm I(x_3)
        \phi_\mathrm I(x_4)
        \phi(x) \phi(x) \phi(x) \phi(x)
    }
    \ket 0
    = C_0 + C_1 + C_2.
\]
Then we will look into each group separately.

\paragraph{No self contractions}

So we start with the terms in the group without self contractions. Those
terms are all equal. Again, we only consider fully contracted terms here, since
the vacuum expectation values of all other normal ordered products is zero. We
choose one representative contraction and multiply it by $4!$.
\[
    C_0
    = 4! \bra 0
    %
    \contraction{}{\phi_\mathrm I} {(x_1)\phi_\mathrm I(x_2) \phi_\mathrm I(x_3) \phi_\mathrm I(x_4)} {\phi}
    %
    \contraction[1.5ex]{\phi_\mathrm I(x_1)}{\phi_\mathrm I}{(x_2) \phi_\mathrm I(x_3)
    \phi_\mathrm I(x_4) \phi(x)} {\phi}
    %
    \bcontraction{\phi_\mathrm I(x_1) \phi_\mathrm I(x_2)}{\phi_\mathrm I}{(x_3)
    \phi_\mathrm I(x_4) \phi(x) \phi(x)} {\phi}
    %
    \bcontraction[1.5ex]{\phi_\mathrm I(x_1) \phi_\mathrm I(x_2) \phi_\mathrm I(x_3)
    }{\phi_\mathrm I}{(x_4) \phi(x) \phi(x) \phi(x)} {\phi}
    %
    \phi_\mathrm I(x_1)
    \phi_\mathrm I(x_2)
    \phi_\mathrm I(x_3)
    \phi_\mathrm I(x_4)
    \phi(x) \phi(x) \phi(x) \phi(x)
    \ket 0
    = 4! \prod_{i = 1}^4 D(x_i - x)
\]

This can be pictured as the following diagram:

\begin{fmwrapper}
    \begin{fmffile}{cross}
        \begin{fmfgraph*}(40, 25)
            \fmfleft{i1,i2}
            \fmfright{o1,o2}

            \fmflabel{$x_1$}{i1}
            \fmflabel{$x_2$}{i2}
            \fmflabel{$x_3$}{o1}
            \fmflabel{$x_4$}{o2}
            \fmflabel{$x$}{v1}

            \fmfdot{v1,i1,i2,o1,o2}

            \fmf{plain}{i1,v1}
            \fmf{plain}{i2,v1}
            \fmf{plain}{o1,v1}
            \fmf{plain}{o2,v1}
        \end{fmfgraph*}
    \end{fmffile}
\end{fmwrapper}

\paragraph{One self contraction}

We have to come up 72 contractions here. However, there are only six unique
configuration. This is easiest to see since there are only six ways to form one
contraction of the $\phi_\mathrm I$ uniquely. The duplicity will be $72 / 6 =
12$ for each term. There will be a $D(0)$ from the single contraction of the
$\phi^4$ term.
\begin{align*}
    C_2 &= 12 \left[
        D(x_1 - x_2) D(x_3 - x) D(x_4 - x)
        + D(x_1 - x_3) D(x_2 - x) D(x_4 - x)
        \right. \\ &\quad
        + D(x_1 - x_4) D(x_2 - x) D(x_3 - x)
        + D(x_2 - x_3) D(x_1 - x) D(x_4 - x)
        \\ &\quad \left.
        + D(x_2 - x_4) D(x_1 - x) D(x_3 - x)
        + D(x_3 - x_4) D(x_1 - x) D(x_2 - x)
    \right] D(0)
\end{align*}
In total, there are $12 \cdot 6 = 72$ terms, this is fine. Sorry for not
typesetting the contractions here, they just eat too much time. Creating
Feynman diagrams did cost some time as well. I guess I am hypocritical here
:-). Anyway, the first summand can be pictured like this:

\begin{fmwrapper}
    \begin{fmffile}{one-self-1}
        \begin{fmfgraph*}(40, 40)
            \fmfleft{x1,x2}
            \fmfright{x3,x4}

            \fmflabel{$x_1$}{x1}
            \fmflabel{$x_2$}{x2}
            \fmflabel{$x_3$}{x3}
            \fmflabel{$x_4$}{x4}
            \fmflabel{$x$}{x}

            \fmfdot{x,x1,x2,x3,x4}

            \fmf{plain}{x1,x2}
            \fmf{plain}{x3,x}
            \fmf{plain}{x4,x}

            \fmf{plain,tension=0.4}{x,x}
        \end{fmfgraph*}
    \end{fmffile}
\end{fmwrapper}

All the other graphs are variations of this.


\paragraph{Two self contractions}

As written earlier, the contractions where the $\phi^4$ term is fully
contracted with itself are independent of the contractions of the $\phi_\mathrm
I$ with each other. We see the zeroth order term with an additional factor from
the $\phi^4$ term. There are three different contractions within $\phi^4$, so
we get an additional factor three.
\[
    C_2
    = 3 \sbr{ D(x_1 - x_3) D(x_2 - x_4) + D(x_1 - x_4) D(x_2 - x_3)
    + D(x_1 - x_2) D(x_3 - x_4)} D(0)^2
\]

The first summand can be pictured like this:

\begin{fmwrapper}
    \begin{fmffile}{two-self}
        \begin{fmfgraph*}(100, 40)
            \fmftop{x1,x2,p1}
            \fmfbottom{x3,x4,p2}

            \fmflabel{$x_1$}{x1}
            \fmflabel{$x_2$}{x2}
            \fmflabel{$x_3$}{x3}
            \fmflabel{$x_4$}{x4}
            \fmflabel{$x$}{x}

            \fmfdot{x,x1,x2,x3,x4}

            \fmf{plain}{x1,x3}
            \fmf{plain}{x2,x4}

            \fmf{phantom,tension=5}{p1,x,p2}

            \fmf{plain,tension=0.3,right=0}{x,x}
            \fmf{plain,tension=0.3,right=180}{x,x}
        \end{fmfgraph*}
    \end{fmffile}
\end{fmwrapper}

This is the first time that we have a disconnected piece in our diagram.

\paragraph{End result}

Now we just have to combine all three terms together to get the desired result.
\begin{multline*}
    - \frac{\iup \lambda}{4!}
    \int_{-T}^T \dif t
    \int_{\R^3} \dif^3 x \,
    \bra 0
    \timeorder\del{
        \phi_\mathrm I(x_1)
        \phi_\mathrm I(x_2)
        \phi_\mathrm I(x_3)
        \phi_\mathrm I(x_4)
        \phi(x) \phi(x) \phi(x) \phi(x)
    }
    \ket 0 \\
    = - \frac{\iup \lambda}{4!}
    \int_{-T}^T \dif t
    \int_{\R^3} \dif^3 x \,
    \bra 0 \Bigg[
    4! \prod_{i = 1}^4 D(x_i - x) \\
    \quad + 12 \left[
        D(x_1 - x_2) D(x_3 - x) D(x_4 - x)
        + D(x_1 - x_3) D(x_2 - x) D(x_4 - x)
        \right. \\ \quad
        + D(x_1 - x_4) D(x_2 - x) D(x_3 - x)
        + D(x_2 - x_3) D(x_1 - x) D(x_4 - x)
        \\ \quad \left.
        + D(x_2 - x_4) D(x_1 - x) D(x_3 - x)
        + D(x_3 - x_4) D(x_1 - x) D(x_2 - x)
    \right] D(0) \\
    + 3 \sbr{ D(x_1 - x_3) D(x_2 - x_4) + D(x_1 - x_4) D(x_2 - x_3)
+ D(x_1 - x_2) D(x_3 - x_4)} D(0)^2 \Bigg] \ket 0
\end{multline*}

Each term has four propagators in it, which should be the case.


\subsection{Odd number of fields}

With an odd number of fields we cannot fully contract the fields. The one field
operator that is left will be normal ordered and annihilate the vacuum. Those
Green's functions are zero.
\section{A glimpse at the Path integral in zero dimensions}
\label{homework:2}



\end{document}

% vim: spell spelllang=en tw=79
