\documentclass[11pt, english, fleqn, DIV=15, headinclude, BCOR=1cm]{scrartcl}

\usepackage[bibatend]{../header}
\usepackage{../my-boxes}

\usepackage{booktabs}
\usepackage{slashed}
\usepackage{simplewick}

\newcommand\timeorder{\mathscr T}
\newcommand\normorder{\mathscr N}

\hypersetup{
    pdftitle=
}

\newcounter{totalpoints}
\newcommand\punkte[1]{#1\addtocounter{totalpoints}{#1}}

\newcounter{problemset}
\setcounter{problemset}{8}

\subject{physics755 -- Quantum Field Theory}
\ihead{physics755 -- Problem Set \arabic{problemset}}

\title{Problem Set \arabic{problemset}}

\newcommand\thegroup{Group Tuesday -- Ripunjay Acharya}

\publishers{\thegroup}
\ofoot{\thegroup}

\author{
    Martin Ueding \\ \small{\href{mailto:mu@martin-ueding.de}{mu@martin-ueding.de}}
    \and
    Oleg Hamm
}
\ifoot{Martin Ueding, Oleg Hamm}

\ohead{\rightmark}

\begin{document}

\maketitle

\vspace{3ex}

\begin{center}
    \begin{tabular}{rrr}
        problem & achieved points & possible points \\
        \midrule
        \nameref{homework:1} & & \punkte{8} \\
        \nameref{homework:2} & & \punkte{6} \\
        \midrule
        total & & \arabic{totalpoints}
    \end{tabular}
\end{center}

\section{Wick gymnastics}
\label{homework:1}

In Equation~(1) the symbol $T$ is used as time ordering operator and the bounds
of the integration. This does not make too much sense. We will use $\timeorder$
to time ordering like on the last sheet.

\subsection{First order}

We are supposed to compute the numerator to first order in $\lambda$. The
numerator is given as
\begin{gather*}
    \bra 0
    \timeorder\del{
        \phi_\mathrm I(x_1)
        \phi_\mathrm I(x_2)
        \phi_\mathrm I(x_3)
        \phi_\mathrm I(x_4)
        \exp\del {
            - \iup \int_{-T}^T \dif t \, H_\mathrm I(t)
        }
    }
    \ket 0
    \intertext{%
        We expand the Hamiltonian in the interacting picture.
    }
    = \bra 0
    \timeorder\del{
        \phi_\mathrm I(x_1)
        \phi_\mathrm I(x_2)
        \phi_\mathrm I(x_3)
        \phi_\mathrm I(x_4)
        \exp\del {
            - \frac{\iup \lambda}{4!} \int_{-T}^T \dif t
            \int_{\R^3} \dif^3 x \, \phi^4
        }
    }
    \ket 0
    \intertext{%
        Since we are only interested in terms linear in $\lambda$, we linearize
        the exponential and use the linearity of the $L^2$ scalar product.
    }
    = \bra 0
    \timeorder\del{
        \phi_\mathrm I(x_1)
        \phi_\mathrm I(x_2)
        \phi_\mathrm I(x_3)
        \phi_\mathrm I(x_4)
    }
    \ket 0 \\\quad
    - \frac{\iup \lambda}{4!} \bra 0
    \timeorder\del{
        \int_{-T}^T \dif t
        \int_{\R^3} \dif^3 x \,
        \phi_\mathrm I(x_1)
        \phi_\mathrm I(x_2)
        \phi_\mathrm I(x_3)
        \phi_\mathrm I(x_4)
        \phi(x) \phi(x) \phi(x) \phi(x)
    }
    \ket 0
    \intertext{%
        We use Wick's theorem in the first summand and convert the time
        ordering into normal ordering and contractions. Only fully contracted
        terms survive the vacuum (bad pun intended). We therefore only have to
        keep the three different contractions. The integrals and the time
        ordering should commute since the time ordering only applies to the
        integrands, not the integral signs.
    }
    = D(x_1 - x_3) D(x_2 - x_4) + D(x_1 - x_4) D(x_2 - x_3)
    + D(x_1 - x_2) D(x_3 - x_4)
    \\\quad
    - \frac{\iup \lambda}{4!}
    \int_{-T}^T \dif t
    \int_{\R^3} \dif^3 x \,
    \bra 0
    \timeorder\del{
        \phi_\mathrm I(x_1)
        \phi_\mathrm I(x_2)
        \phi_\mathrm I(x_3)
        \phi_\mathrm I(x_4)
        \phi(x) \phi(x) \phi(x) \phi(x)
    }
    \ket 0
\end{gather*}

\section{A glimpse at the Path integral in zero dimensions}
\label{homework:2}



\end{document}

% vim: spell spelllang=en tw=79
