\documentclass[11pt, english, fleqn, DIV=15, headinclude, BCOR=1cm]{scrartcl}

\usepackage[bibatend]{../header}
\usepackage{../my-boxes}

\usepackage{booktabs}
\usepackage{slashed}

\hypersetup{
    pdftitle=
}

\newcounter{totalpoints}
\newcommand\punkte[1]{#1\addtocounter{totalpoints}{#1}}

\newcounter{problemset}
\setcounter{problemset}{6}

\subject{physics755 -- Quantum Field Theory}
\ihead{physics755 -- Problem Set \arabic{problemset}}

\title{Problem Set \arabic{problemset}}

\newcommand\thegroup{Group Tuesday -- Ripunjay Acharya}

\publishers{\thegroup}
\ofoot{\thegroup}

\author{
    Martin Ueding \\ \small{\href{mailto:mu@martin-ueding.de}{mu@martin-ueding.de}}
    \and
    Oleg Hamm
}
\ifoot{Martin Ueding, Oleg Hamm}

\ohead{\rightmark}

\begin{document}

\maketitle

\vspace{3ex}

\begin{center}
    \begin{tabular}{rrr}
        problem & achieved points & possible points \\
        \midrule
        \nameref{homework:1} & & \punkte{15} \\
        \midrule
        total & & \arabic{totalpoints}
    \end{tabular}
\end{center}

\section{Canonical quantization of the electromagnetic field}
\label{homework:1}

\subsection{Lagrangian}

\paragraph{Lagrangian}

We start with the given Lagrange density in convenient short notation for the
derivatives.
\begin{align*}
    \mathscr L'
    &= - \frac{1}{4} F_{\mu\nu} F^{\mu\nu} - \frac{\kappa}{2}
    \sbr{A^\mu{}_{,\mu}}^2
    \intertext{%
        Then we expand the field strength tensor using the non-idempotent
        antisymmetrization notation. We directly apply the Lorenz gauge to get
        rid of terms.
    }
    &= - \frac{1}{4} A_{[\mu,\nu]} \; A^{[\mu,\nu]}
    \intertext{%
        There is no need to antisymmetrize both of them, we can just add a
        factor of two and omit one bracket.
    }
    &= - \frac{1}{2} A_{\mu,\nu} \; A^{[\mu,\nu]}
    \intertext{%
        We write it out explicitly.
    }
    &= - \frac{1}{2} A_{\mu,\nu} \; A^{\mu,\nu} + \frac{1}{2} A_{\mu,\nu} \; A^{\nu,\mu}
    \intertext{%
        The first term is the one that we want, we have to get rid of the
        second one. We note that
        \[
            \partial_\mu A_\nu A^{\mu,\nu} = A_{\nu,\mu} \; A^{\mu,\nu} + A_\nu
            A^{\mu,\nu}{}_{,\mu}.
        \]
        The second summand is a divergence of $\tens A$ and can therefore be
        set to zero. We replace the unwanted term on the Lagrangian with the
        total derivative.
    }
    &= - \frac{1}{2} A_{\mu,\nu} \; A^{\mu,\nu} + \partial_\mu A_\nu A^{\mu,\nu}
    \intertext{%
        As a last step we argue that the equations of motions do not change
        when we add a total derivative to the Lagrangian. We can therefore omit
        since we are only interested in the physics. That is the definition of
        a new Lagrangian density that is equivalent to the previous one.
    }
    \tilde{\mathscr L}
    &= - \frac{1}{2} A_{\mu,\nu} \; A^{\mu,\nu} + \partial_\mu A_\nu A^{\mu,\nu}
\end{align*}

\paragraph{Canonical momenta}

We have four fields here: $\{ A^\mu \}$. Each of those fields has its own
momentum.
\[
    \pi^{\mu} = \pd{\mathscr L}{\dot A_{\mu}} = - \frac 12 \dot A^{\mu}.
\]

% TODO Should there be an additional factor of 2 because we take the derivative
% of the square of A?

\subsection{Field expansion}

\paragraph{Equations of motion}

For a single field, the Euler-Lagrange equations look like this:
\[
    \partial_\mu \pd{\mathscr L}{\phi_{,\mu}} - \pd{\mathscr L}{\phi} = 0
\]
We have four fields here, so we have four equations of motion, indexed by
$\mu$:
\[
    \partial_\lambda \pd{\mathscr L}{A_{\mu,\lambda}} - \pd{\mathscr L}{A_\mu}
    = 0
    \iff
    \partial_\mu A^{\mu,\lambda} = 0
    \iff
    A^{\mu,\lambda}{}_{,\lambda} = 0
    \iff
    \dalambert A^\mu = 0
\]
This looks familiar. If there had been some source term $\tens J$ in the
theory, this would have shown up here as well. This theory only describes
electromagnetic fields without any charged particles.

\paragraph{Field expansion}

Here are the arguments to expand it like Equation~(5) on the problem set:

\begin{itemize}
    \item
        Every square integrable function can be written as a Fourier series.
        Since one assumes that the non-zero domain $\tens A$ is bounded, this
        is the case. This motivates the $\int \dif^3 p$.

    \item
        To make the integral Lorentz invariant, the factor $[2
        \omega_p]^{-1/2}$ is added to the integral.

    \item
        Since the equation of motion for $A^\mu$ is the Klein-Gordon equation
        for massless particles, we can think of the electric field as an
        infinite amount of harmonic oscillators, just that there are four
        fields now. This lets us write the modes as annihilation and creation
        operators.

    \item
        In contrast to the scalar (or pseudoscalar?) Klein-Gordon field with
        spin 0, we have a field which is vector polarized and therefore has
        spin 1. The gravitational field is tensor polarized and therefore the
        graviton has spin 2, by the way. Since the particles in our theory, the
        photons, are vector polarized, the creation and annihilation operators
        need to take the polarization axis $\lambda$ as an additional argument.
        This $\lambda$ has nothing to do with the wavelength of the photon.
\end{itemize}

\paragraph{Form of negative frequency part}

The coefficient function $\epsilon$ is complex conjugated because we want $A$
to be a hermitian operator to use it as an observable with real eigenvalues.

\paragraph{Momenta expansion}

By analogy and looking at the case for the Klein-Gordon field from
\textcite[(2.26)]{Peskin/QFT/1995} we know that we have to introduce a minus
between the summands. Then there are different factors in front. In total we
get:
\[
    \pi^\mu(\vec x) = \int \frac{\dif^3 p}{[2\piup]^3} [-\iup]
    \sqrt{\frac{\omega_p}{2}} \sum_{\lambda=0}^3 \sbr{
        \epsilon^\mu(p, \lambda) a_{\vec p, \lambda} \exp(\iup \vec p \cdot
        \vec x)
        -
        \epsilon^{\mu*}(p, \lambda) a_{\vec p, \lambda}^\dagger \exp(-\iup \vec p \cdot
        \vec x)
    }
\]
The index “comma $\lambda$” is meant as a juxtaposition of indices, not as a
partial derivative with respect to $x^\lambda$.

\subsection{Canonical commutations relations}

We are supposed to compute the given commutator. We start by splitting temporal
and spatial parts in the implicit sum.
\begin{align*}
    \sbr{A^\mu{}_{,\mu}(\vec x, t), A^\nu(\vec y, t)}
    &= \sbr{\dot A^0(\vec x, t), A^\nu(\vec y, t)}
    + \sbr{A^i{}_{,i}(\vec x, t), A^\nu(\vec y, t)}
    \intertext{%
        Now we can insert the canonical momentum that we have calculated.
    }
    &= - 2 \sbr{\pi^0(\vec x, t), A^\nu(\vec y, t)}
    + \sbr{A^i{}_{,i}(\vec x, t), A^\nu(\vec y, t)}
    \intertext{%
        That commutator is given in Equation~(6) on the problem set.
    }
    &= 2 \iup \eta^{0\nu} \deltaup^{(3)}(\vec x - \vec y)
    + \sbr{A^i{}_{,i}(\vec x, t), A^\nu(\vec y, t)}
\end{align*}
It depends on the commutation of the spatial derivatives whether this might go
back to zero. If we had applied the Lorenz gauge before computing the
commutator, it would definitely be zero.

\subsection{Expressions for ladder operators}

\paragraph{Annihilation operator}

We take the expressions for the field and the canonical momentum and add them
after multiplying the second equation with an with appropriate factor. We get
one equation:
\begin{gather*}
    \int \frac{\dif^3 p}{[2\piup]^3} \frac{1}{\sqrt{2\omega_p}} \sum_{\lambda
    = 0}^3 2 \epsilon^\mu(p, \lambda) \, a_{\vec p, \lambda} \eup^{\iup \vec p
    \cdot \vec x}
    =
    A^\mu(\vec x) - \frac{\iup}{\omega_p} \pi^\mu(\vec x).
    \intertext{%
        The isolation of the annihilation operator is the goal, so we
        already move as many factors to the other side as possible. We already
        add a $-\iup$ to the left hand side as well.
    }
    \int \frac{\dif^3 p}{[2\piup]^3} \sum_{\lambda
    = 0}^3 \epsilon^\mu(p, \lambda) \, a_{\vec p, \lambda} \eup^{\iup \vec p
    \cdot \vec x}
    =
    - \iup \frac{1}{\sqrt{2\omega_p}} \sbr{\pi^\mu(\vec x) + \iup \omega_p A^\mu(\vec x)}
    \intertext{%
        We contract the whole expression with $\epsilon_\mu(p, \kappa)$.
    }
    \int \frac{\dif^3 p}{[2\piup]^3} \sum_{\lambda
    = 0}^3 \epsilon_\mu(p, \kappa) \epsilon^\mu(p, \lambda) \, a_{\vec p, \lambda} \eup^{\iup \vec p
    \cdot \vec x}
    =
    - \iup \frac{1}{\sqrt{2\omega_p}} \epsilon_\mu(p, \kappa) \sbr{\pi^\mu(\vec x) + \iup \omega_p A^\mu(\vec x)}
    \intertext{%
        The right hand side now contains the orthonormality relation for the
        polarization vectors, we insert it.
    }
    \int \frac{\dif^3 p}{[2\piup]^3} \sum_{\lambda
= 0}^3 \eta_{\kappa\lambda} \, a_{\vec p, \lambda} \eup^{\iup \vec p
    \cdot \vec x}
    =
    - \iup \frac{1}{\sqrt{2\omega_p}} \epsilon_\mu(p, \kappa) \sbr{\pi^\mu(\vec x) + \iup \omega_p A^\mu(\vec x)}
    \intertext{%
        The only contributing term in the sum $\sum_{\lambda = 0}^3
        \eta_{\kappa\lambda}$ will be $\eta_{\kappa\kappa}$. Since that is just
        $\pm 1$ we can safely move it to the other side of the equation.
    }
    \int \frac{\dif^3 p}{[2\piup]^3} a_{\vec p, \lambda} \eup^{\iup \vec p
    \cdot \vec x}
    =
    - \iup \eta_{\kappa\lambda} \frac{1}{\sqrt{2\omega_p}} \epsilon_\mu(p, \kappa) \sbr{\pi^\mu(\vec x) + \iup \omega_p A^\mu(\vec x)}
    \intertext{%
        The order of steps that we took caused a problem: $p$ is not defined on
        the left hand side since it is the integration variable of the right
        hand side. We will now fix this by adding a Fourier transform into
        momentum space on both side. We should have done this before even
        adding the equations, but that would have lead to much clutter. We are
        aware that this is not very precise, but since the spatial integral on
        the right hand side will yield a $\deltaup$-distribution, it would not
        have changed anything. Since we only get three points for this problem
        which contains four subproblems, we are just going to leave it like
        that and let it squeal.
    }
    \int \frac{\dif^3 p}{[2\piup]^3} \dif^3x \, \eup^{\iup [\vec p - \vec k] \cdot \vec x}  a_{\vec p, \lambda}
    =
    - \iup \eta_{\kappa\lambda} \int \dif^3x \, \eup^{-\iup \vec k \cdot \vec x} \frac{1}{\sqrt{2\omega_k}} \epsilon_\mu(k, \kappa) \sbr{\pi^\mu(\vec x) + \iup \omega_k A^\mu(\vec x)}
    \intertext{%
        We get a $\deltaup(\vec p - \vec k)$ from the spatial integral. Then we
        can apply the momentum integral and set all the momenta to $\vec k$.
    }
    a_{\vec k, \lambda}
    =
    - \iup \eta_{\kappa\lambda} \int \dif^3x \, \eup^{-\iup \vec k \cdot \vec x} \frac{1}{\sqrt{2\omega_k}} \epsilon_\mu(k, \kappa) \sbr{\pi^\mu(\vec x) + \iup \omega_k A^\mu(\vec x)}
    \intertext{%
        Since the metric tensor is always diagonal, we have to put a $\lambda$
        there again as well. We rename $\vec k$ back to $\vec p$. Then we
        arrived at the expression that was wanted.
    }
    a_{\vec p, \lambda}
    =
    - \iup \eta_{\lambda\lambda} \int \dif^3x \, \eup^{-\iup \vec p \cdot \vec x} \frac{1}{\sqrt{2\omega_p}} \epsilon_\mu(p, \lambda) \sbr{\pi^\mu(\vec x) + \iup \omega_p A^\mu(\vec x)}
\end{gather*}
Again, we should have done the Fourier transform as the first step and rename
the variables with hindsight, but the idea is still correct.

\paragraph{Creation operator}

We should be able to take the hermitian conjugate of the whole expression and
get the result for the creation operator.
\[
    a_{\vec p, \lambda}^\dagger
    =
    \iup \eta_{\lambda\lambda} \int \dif^3x \, \eup^{\iup \vec p \cdot \vec x}
    \frac{1}{\sqrt{2\omega_p}} \epsilon_\mu(p, \lambda)^* \sbr{\pi^\mu(\vec x) - \iup \omega_p A^\mu(\vec x)}
\]

\paragraph{Mixed commutator}

We compute the mixed commutator. The crucial step is to rename all the local
variables in the second term.
\begin{align*}
    \sbr{a_{\vec p, \lambda}, a_{\vec p', \lambda'}}
    &= \left[
        - \iup \eta_{\lambda\lambda} \int \dif^3x \, \eup^{-\iup \vec p \cdot
        \vec x} \frac{1}{\sqrt{2\omega_p}} \epsilon_\mu(p, \lambda)
        \sbr{\pi^\mu(\vec x) + \iup \omega_p' A^\mu(\vec x)}
        ,
        \right. \\ & \quad \left.
        \iup \eta_{\lambda'\lambda'} \int \dif^3x' \, \eup^{\iup \vec p' \cdot
        \vec x'}
        \frac{1}{\sqrt{2\omega_p'}} \epsilon_\nu(p', \lambda')^*
        \sbr{\pi^\nu(\vec x') - \iup \omega_p' A^\nu(\vec x')}
    \right]
    \intertext{%
        We can move some of the constant factors up front.
    }
    &= \eta_{\lambda\lambda} \eta_{\lambda'\lambda'} \left[
        \int \dif^3x \, \eup^{-\iup \vec p \cdot \vec x}
        \frac{1}{\sqrt{2\omega_p}} \epsilon_\mu(p, \lambda) \sbr{\pi^\mu(\vec
        x) + \iup \omega_p A^\mu(\vec x)}
        ,
        \right. \\ & \quad \left.
        \int \dif^3x' \, \eup^{\iup \vec p' \cdot
        \vec x'} \frac{1}{\sqrt{2\omega_p'}} \epsilon_\nu(p', \lambda')^*
        \sbr{\pi^\nu(\vec x') - \iup \omega_p' A^\nu(\vec x')}
    \right]
    \intertext{%
        We can take this further and move everything except the field and the
        momentum out of the commutator.
    }
    &= \eta_{\lambda\lambda} \eta_{\lambda'\lambda'}
    \int \dif^3x \, \dif^3x' \, \eup^{-\iup \vec p \cdot \vec x}
    \eup^{\iup \vec p' \cdot \vec x'}
    \frac{1}{\sqrt{2\omega_p}}
    \frac{1}{\sqrt{2\omega_p'}}
    \epsilon_\mu(p, \lambda)
    \epsilon_\nu(p', \lambda')^*
    \\&\quad
    \sbr{
        \pi^\mu(\vec x) + \iup \omega_p A^\mu(\vec x)
        ,
        \pi^\nu(\vec x') - \iup \omega_p' A^\nu(\vec x')
    }
    \intertext{%
        We simplify.
    }
    &= \eta_{\lambda\lambda} \eta_{\lambda'\lambda'}
    \int \dif^3x \, \dif^3x' \, 
    \eup^{\iup \vec p' \cdot \vec x' - \iup \vec p \cdot \vec x}
    \frac{1}{2 \sqrt{\omega_p \omega_p'}}
    \epsilon_\mu(p, \lambda)
    \epsilon_\nu(p', \lambda')^*
    \\&\quad
    \sbr{
        \pi^\mu(\vec x) + \iup \omega_p A^\mu(\vec x)
        ,
        \pi^\nu(\vec x') - \iup \omega_p' A^\nu(\vec x')
    }
    \intertext{%
        The only contributing commutators are the mixed ones.
    }
    &= \eta_{\lambda\lambda} \eta_{\lambda'\lambda'}
    \int \dif^3x \, \dif^3x' \, 
    \eup^{\iup \vec p' \cdot \vec x' - \iup \vec p \cdot \vec x}
    \frac{1}{2 \sqrt{\omega_p \omega_p'}}
    \epsilon_\mu(p, \lambda)
    \epsilon_\nu(p', \lambda')^*
    \\&\quad
    \sbr{
        \iup \omega_p' \sbr{
            A^\nu(\vec x')
            ,
            \pi^\mu(\vec x)
        }
        + \iup \omega_p \sbr{
            A^\mu(\vec x)
            ,
            \pi^\nu(\vec x')
        }
    }
    \intertext{%
        Since those commutators were imposed earlier, we can just insert them.
    }
    &= \eta_{\lambda\lambda} \eta_{\lambda'\lambda'}
    \int \dif^3x \, \dif^3x' \, 
    \eup^{\iup \vec p' \cdot \vec x' - \iup \vec p \cdot \vec x}
    \frac{1}{2 \sqrt{\omega_p \omega_p'}}
    \epsilon_\mu(p, \lambda)
    \epsilon_\nu(p', \lambda')^*
    \\&\quad
    \iup \sbr{\omega_p' + \omega_p} \eta^{\mu\nu}
    \deltaup^{(3)}(\vec x - \vec x')
    \intertext{%
        The perform the integration over $x'$.
    }
    &= \iup \eta_{\lambda\lambda} \eta_{\lambda\lambda}
    \int \dif^3x \,
    \eup^{\iup [\vec p' - \vec p] \cdot \vec x}
    \frac{1}{2 \sqrt{\omega_p \omega_p'}}
    \epsilon_\mu(p, \lambda)
    \epsilon_\nu(p', \lambda')^*
    \sbr{\omega_p' + \omega_p} \eta^{\mu\nu}
    \intertext{%
        The integration over $x$ now gives another such distribution. Since the
        term would vanish otherwise, we can also set $p'$ to $p$.
    }
    &= \iup \eta_{\lambda\lambda} \eta_{\lambda'\lambda'}
    \deltaup^{(3)}(\vec p - \vec p')
    \frac{1}{2 \omega_p}
    \epsilon_\mu(p, \lambda)
    \epsilon_\nu(p, \lambda')^*
    2 \omega_p \eta^{\mu\nu}
    \intertext{%
        We simplify more and use the orthonormality relation of the
        polarization vectors. Equation~(11) from the problem set does not have
        the complex conjugation, though. Since they are vectors in the
        Minkowski space, they are probably real anyway. That is in contrast
        with Equation~(5) where the complex conjugate of this polarization
        vector appears. Ignoring this, it works out, so this either cancels a
        different mistake or the complex conjugation does not matter here.
    }
    &= \iup \eta_{\lambda\lambda} \eta_{\lambda'\lambda'} \eta_{\lambda\lambda'}
    \deltaup^{(3)}(\vec p - \vec p')
    \intertext{%
        The metric tensor is diagonal in special relativity, so we have to have
        $\lambda' = \lambda$ to get any contributing terms. If that is the
        case, the first to components of the metric tensor will be both $+1$ or
        $-1$, they therefore cancel. Only the one with the mixed $\lambda$ and
        $\lambda'$ remains. This is the final result:
    }
    &= \iup \eta_{\lambda\lambda'} \deltaup^{(3)}(\vec p - \vec p').
\end{align*}

\paragraph{Reflexive commutators}

The commutators here contain the same type of ladder operator twice. This means
that they have the same sign between field and momentum. The order of the two
elements differ, so we incur a minus sign when we swap the field in front of
the momentum. The commutators are then equal and cancel. Therefore, the whole
commutator of the two ladder operators of the same type vanishes.

\subsection{Hamiltonian}

\paragraph{Ordered Hamiltonian}

We assemble $H$ from $\mathscr L$ and $\pi$.
\begin{align*}
    H
    &= \int \dif^3 x \sbr{\pi_\mu \, \dot A^\mu - \mathscr L}
    \intertext{%
        We insert the momentum.
    }
    &= \frac 12 \int \dif^3 x \sbr{- \dot A_\mu \, \dot A^\mu + A_{\mu,\nu} \;
    A^{\mu,\nu}}
    \intertext{%
        The first summand substract all the time derivatives from the second
        one. We denote that by using Latin indices there.
    }
    &= \frac 12 \int \dif^3 x \, A_{\mu,j} \; A^{\mu,j}
    \intertext{%
        This is where the multi line equations start again. We insert the
        partial derivatives of $A$. It is important to chose different local
        variables here as well.
    }
    &= - \frac 12 \int \dif^3 x \frac{\dif^3 p}{[2\piup]^3} \frac{\dif^3 p'}{[2\piup]^3}
    \frac{p^j p'_j}{2 \sqrt{\omega_p \omega_{p'}}}
    \sum_{\lambda = 0}^3 \sum_{\lambda' = 0}^3
    \\&\qquad
    \sbr{
        \epsilon^\mu(p, \lambda) a_{\vec p, \lambda}
        \eup^{\iup \vec p \cdot \vec x}
        -
        \epsilon^{\mu^*}(p, \lambda) a_{\vec p, \lambda}^\dagger 
        \eup^{-\iup \vec p \cdot \vec x}
    }
    \sbr{
        \epsilon_\mu(p', \lambda') a_{\vec p', \lambda'}
        \eup^{\iup \vec p' \cdot \vec x}
        -
        \epsilon_{\mu^*}(p', \lambda') a_{\vec p', \lambda'}^\dagger
        \eup^{-\iup \vec p' \cdot \vec x}
    }
    \intertext{%
        Expanding this will even get worse since there are four terms now.
    }
    &= - \frac 12 \int \dif^3 x \frac{\dif^3 p}{[2\piup]^3} \frac{\dif^3 p'}{[2\piup]^3}
    \frac{p^j p'_j}{2 \sqrt{\omega_p \omega_{p'}}}
    \sum_{\lambda = 0}^3 \sum_{\lambda' = 0}^3
    \\&\qquad
    \left[
        \epsilon^\mu(p, \lambda) a_{\vec p, \lambda}
        \eup^{\iup \vec p \cdot \vec x}
        \epsilon_\mu(p', \lambda') a_{\vec p', \lambda'}
        \eup^{\iup \vec p' \cdot \vec x}
        -
        \epsilon^{\mu^*}(p, \lambda) a_{\vec p, \lambda}^\dagger 
        \eup^{-\iup \vec p \cdot \vec x}
        \epsilon_\mu(p', \lambda') a_{\vec p', \lambda'}
        \eup^{\iup \vec p' \cdot \vec x}
        \right.
    \\&\qquad
        \left.
        -
        \epsilon^\mu(p, \lambda) a_{\vec p, \lambda}
        \eup^{\iup \vec p \cdot \vec x}
        \epsilon_{\mu^*}(p', \lambda') a_{\vec p', \lambda'}^\dagger
        \eup^{-\iup \vec p' \cdot \vec x}
        +
        \epsilon^{\mu^*}(p, \lambda) a_{\vec p, \lambda}^\dagger 
        \eup^{-\iup \vec p \cdot \vec x}
        \epsilon_{\mu^*}(p', \lambda') a_{\vec p', \lambda'}^\dagger
        \eup^{-\iup \vec p' \cdot \vec x}
    \right]
    \intertext{%
        We pull out all the polarization vectors and ignore the complex
        conjugate. That seemed like it would work before, and it does work out
        here also. We also combine the exponentials.
    }
    &= - \frac 12 \int \dif^3 x \frac{\dif^3 p}{[2\piup]^3} \frac{\dif^3 p'}{[2\piup]^3}
    \frac{p^j p'_j}{2 \sqrt{\omega_p \omega_{p'}}}
    \sum_{\lambda = 0}^3 \sum_{\lambda' = 0}^3
    \epsilon^\mu(p, \lambda) \epsilon_\mu(p', \lambda') 
    \\&\qquad
    \left[
        a_{\vec p, \lambda}
        a_{\vec p', \lambda'}
        \eup^{\iup [\vec p + \vec p'] \cdot \vec x}
        -
        a_{\vec p, \lambda}^\dagger 
        a_{\vec p', \lambda'}
        \eup^{-\iup [\vec p - \vec p'] \cdot \vec x}
        -
        a_{\vec p, \lambda}
        a_{\vec p', \lambda'}^\dagger
        \eup^{\iup [\vec p - \vec p'] \cdot \vec x}
        +
        a_{\vec p, \lambda}^\dagger 
        a_{\vec p', \lambda'}^\dagger
        \eup^{-\iup [\vec p + \vec p'] \cdot \vec x}
    \right]
    \intertext{%
        The exponential functions now give $\deltaup$-distributions with the
        integration over $\vec x$. The factor $p^j p'_j$ gives a sign change
        when we identify $\vec p' = -\vec p$ in the terms where the exponential
        carries a plus between the $\vec p$ and $\vec p'$. We will also
        integrate over $\vec x$ to save a few steps.
    }
    &= \frac 12 \int \frac{\dif^3 p}{[2\piup]^3}
    \frac{p^j p_j}{2 \omega_p}
    \sum_{\lambda = 0}^3 \sum_{\lambda' = 0}^3
    \epsilon^\mu(p, \lambda) \epsilon_\mu(p, \lambda') 
    \\&\qquad
    \left[
        a_{\vec p, \lambda}
        a_{- \vec p, \lambda'}
        +
        a_{\vec p, \lambda}^\dagger 
        a_{\vec p, \lambda'}
        +
        a_{\vec p, \lambda}
        a_{\vec p, \lambda'}^\dagger
        +
        a_{\vec p, \lambda}^\dagger 
        a_{- \vec p, \lambda'}^\dagger
    \right]
    \intertext{%
        We can use the orthonormality in the polarization vectors. We also
        directly sum over $\lambda'$. In order to make this work, we need
        $a_{\vec p,\lambda} = a_{-\vec p, \lambda}^\dagger$. This makes sense
        in the way that a particle going one direction can be regarded as an
        antiparticle going in the other direction. The fraction $p^j p_j /
        \omega_p$ is just $\omega_p$. The factor $1/2$ from that fraction is
        canceled with the factor 2 we get from bundling the ladder operator
        products.
    }
    &= \frac 12 \int \frac{\dif^3 p}{[2\piup]^3} \omega_p
     \sum_{\lambda = 0}^3 \eta_{\lambda\lambda}
    \left[
        a_{\vec p, \lambda}^\dagger 
        a_{\vec p, \lambda}
        +
        a_{\vec p, \lambda}
        a_{\vec p, \lambda}^\dagger
    \right]
    \intertext{%
        We commute the second summand and just ignore the infinite amount that
        will come from the commutator. This can be argued since we can only
        measure energy differences.
    }
    &\simeq \int \frac{\dif^3 p}{[2\piup]^3} \omega_p
    \sum_{\lambda = 0}^3 \eta_{\lambda\lambda}
    a_{\vec p, \lambda}^\dagger a_{\vec p, \lambda}
    \intertext{%
        We split off the $\lambda = 0$ case as our last step.
    }
    &= \int \frac{\dif^3 p}{[2\piup]^3} E_p
    \sbr{
        a_{\vec p, 0}^\dagger a_{\vec p, 0}
        -
        \sum_{\lambda = 0}^3
        a_{\vec p, \lambda}^\dagger a_{\vec p, \lambda}
    }
\end{align*}
There is a missing minus sign, though.

\end{document}

% vim: spell spelllang=en tw=79
