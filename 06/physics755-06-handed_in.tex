\documentclass[11pt, english, fleqn, DIV=15, headinclude, BCOR=1cm]{scrartcl}

\usepackage[bibatend]{../header}
\usepackage{../my-boxes}

\usepackage{booktabs}
\usepackage{slashed}

\hypersetup{
    pdftitle=
}

\newcounter{totalpoints}
\newcommand\punkte[1]{#1\addtocounter{totalpoints}{#1}}

\newcounter{problemset}
\setcounter{problemset}{6}

\subject{physics755 -- Quantum Field Theory}
\ihead{physics755 -- Problem Set \arabic{problemset}}

\title{Problem Set \arabic{problemset}}

\newcommand\thegroup{Group Tuesday -- Ripunjay Acharya}

\publishers{\thegroup}
\ofoot{\thegroup}

\author{
    Martin Ueding \\ \small{\href{mailto:mu@martin-ueding.de}{mu@martin-ueding.de}}
    \and
    Oleg Hamm
}
\ifoot{Martin Ueding, Oleg Hamm}

\ohead{\rightmark}

\begin{document}

\maketitle

\vspace{3ex}

\begin{center}
    \begin{tabular}{rrr}
        problem & achieved points & possible points \\
        \midrule
        \nameref{homework:1} & & \punkte{15} \\
        \midrule
        total & & \arabic{totalpoints}
    \end{tabular}
\end{center}

\section{Canonical quantization of the electromagnetic field}
\label{homework:1}

\subsection{Lagrangian}

\paragraph{Lagrangian}

We start with the given Lagrange density in convenient short notation for the
derivatives.
\begin{align*}
    \mathscr L'
    &= - \frac{1}{4} F_{\mu\nu} F^{\mu\nu} - \frac{\kappa}{2}
    \sbr{A^\mu{}_{,\mu}}^2
    \intertext{%
        Then we expand the field strength tensor using the non-idempotent
        antisymmetrization notation. We directly apply the Lorenz gauge to get
        rid of terms.
    }
    &= - \frac{1}{4} A_{[\mu,\nu]} \; A^{[\mu,\nu]}
    \intertext{%
        There is no need to antisymmetrize both of them, we can just add a
        factor of two and omit one bracket.
    }
    &= - \frac{1}{2} A_{\mu,\nu} \; A^{[\mu,\nu]}
    \intertext{%
        We write it out explicitly.
    }
    &= - \frac{1}{2} A_{\mu,\nu} \; A^{\mu,\nu} + \frac{1}{2} A_{\mu,\nu} \; A^{\nu,\mu}
    \intertext{%
        The first term is the one that we want, we have to get rid of the
        second one. We note that
        \[
            \partial_\mu A_\nu A^{\mu,\nu} = A_{\nu,\mu} \; A^{\mu,\nu} + A_\nu
            A^{\mu,\nu}{}_{,\mu}.
        \]
        The second summand is a divergence of $\tens A$ and can therefore be
        set to zero. We replace the unwanted term on the Lagrangian with the
        total derivative.
    }
    &= - \frac{1}{2} A_{\mu,\nu} \; A^{\mu,\nu} + \partial_\mu A_\nu A^{\mu,\nu}
    \intertext{%
        As a last step we argue that the equations of motions do not change
        when we add a total derivative to the Lagrangian. We can therefore omit
        since we are only interested in the physics. That is the definition of
        a new Lagrangian density that is equivalent to the previous one.
    }
    \tilde{\mathscr L}
    &= - \frac{1}{2} A_{\mu,\nu} \; A^{\mu,\nu} + \partial_\mu A_\nu A^{\mu,\nu}
\end{align*}

\paragraph{Canonical momenta}

We have four fields here: $\{ A^\mu \}$. Each of those fields has its own
momentum 4-vector. Although it probably does not transform like a tensor, we
can write it with two indices:
\[
    \pi^{\mu\nu} = \pd{\mathscr L}{A_{\mu,\nu}} = - \frac 12 A^{\mu,\nu}.
\]

% TODO Should there be an additional factor of 2 because we take the derivative
% of the square of A?

\subsection{Field expansion}

\paragraph{Equations of motion}

For a single field, the Euler-Lagrange equations look like this:
\[
    \partial_\mu \pd{\mathscr L}{\phi_{,\mu}} \pd{\mathscr L}{\phi} = 0
\]
We have four fields here, so we have four equations of motion, indexed by
$\mu$:
\[
    \partial_\lambda \pd{\mathscr L}{A_{\mu,\lambda}} - \pd{\mathscr L}{A_\mu}
    = 0
    \iff
    \partial_\mu A^{\mu,\lambda} = 0
    \iff
    A^{\mu,\lambda}{}_{,\lambda} = 0
    \iff
    \dalambert A^\mu = 0
\]
This looks familiar. If there had been some source term $\tens J$ in the
theory, this would have shown up here as well. This theory only describes
electromagnetic fields without any charged particles.

\paragraph{Field expansion}

% TODO

\paragraph{Momenta expansion}

% TODO

\end{document}

% vim: spell spelllang=en tw=79
