\documentclass[11pt, english, fleqn, DIV=15, headinclude, BCOR=1cm]{scrartcl}

\usepackage[bibatend]{../header}
\usepackage{../my-boxes}

\usepackage{booktabs}
\usepackage{slashed}
\usepackage{simplewick}

\usepackage{feynmp-auto}
\usepackage{adjustbox}
\newenvironment{fmwrapper}{\begin{adjustbox}{margin=5mm}}{\end{adjustbox}}

\usepackage{multicol}
\usepackage{lastpage}

\newcommand\timeorder{\mathscr T}
\newcommand\normorder{\mathscr N}

\hypersetup{
    pdftitle=
}

\newcounter{totalpoints}
\newcommand\punkte[1]{#1\addtocounter{totalpoints}{#1}}

\newcounter{problemset}
\setcounter{problemset}{9}

\subject{physics755 -- Quantum Field Theory}
\ihead{physics755 -- Problem Set \arabic{problemset}}

\title{Problem Set \arabic{problemset}}

\newcommand\thegroup{Group Tuesday -- Ripunjay Acharya}

\publishers{\thegroup}
\ofoot{\thegroup}

\author{
    Martin Ueding \\ \small{\href{mailto:mu@martin-ueding.de}{mu@martin-ueding.de}}
    \and
    Oleg Hamm
}
\ifoot{Martin Ueding, Oleg Hamm}

\ohead{\rightmark}

\begin{document}

\maketitle

\vspace{3ex}

\begin{center}
    \begin{tabular}{rrr}
        problem & achieved points & possible points \\
        \midrule
        \nameref{homework:1} & & \punkte{15} \\
        \midrule
        total & & \arabic{totalpoints}
    \end{tabular}
\end{center}

\vspace{3ex}

\begin{center}
    \begin{small}
        This document consists of \pageref{LastPage} pages.
    \end{small}
\end{center}

\section{The S-Matrix}
\label{homework:1}

\subsection{Momentum conservation}

This problem covers the material also shown by \textcite[94 --
95]{Peskin/QFT/1995}.

\paragraph{Momentum conservation at vertex}

Each vertex gives us a factor $- \iup \lambda \int \dif^4 z$, where $\tens z$
is the four-vector for the particular vertex. Each vertex has to have its own
internal integration variable. Then each propagator gives us $D(\tens x - \tens
y)$. Looking at the vertex and all the propagators (labeled with $i$) that end
in this vertex, we get the following term, which we label $A$ to continue
working on it.
\[
    A := - \iup \lambda \int \dif^4 z \, \prod_i D(\tens x_i - \tens z)
\]

To say anything about the momentum, we have to Fourier transform the
propagator. That is given by \textcite[(4.46]{Peskin/QFT/1995} as
\[
    D(\tens x - \tens y) = \int \frac{\dif^4 p}{[2\piup]^4} \frac{\iup}{\tens
    p^2 - m^2 + \iup \epsilon} \exp\del{-\iup \tens p \cdot [\tens x - \tens
    y]}.
\]
We insert this into $A$.
\begin{align*}
    A
    &= - \iup \lambda \int \dif^4 z \, \prod_i \int \frac{\dif^4 p_i}{[2\piup]^4} \frac{\iup}{\tens
    p_i^2 - m^2 + \iup \epsilon} \exp\del{-\iup \tens p_i \cdot [\tens x_i - \tens
    z]}
    \intertext{%
        We can move all the $\tens z$ dependence to the very back of the
        expression to isolate the part which is of interest here.
    }
    &= - \iup \lambda \prod_i \int \frac{\dif^4 p_i}{[2\piup]^4} \frac{\iup}{\tens
    p_i^2 - m^2 + \iup \epsilon}
    \exp(-\iup \tens p_i \cdot \tens x_i)
    \int \dif^4 z \, \exp(\iup \tens p_i \cdot \tens z)
    \intertext{%
        We execute the product for the last term to get the expression that we
        can work with.
    }
    &= - \iup \lambda \sbr{\prod_i \int \frac{\dif^4 p_i}{[2\piup]^4} \frac{\iup}{\tens
        p_i^2 - m^2 + \iup \epsilon}
    \exp(-\iup \tens p_i \cdot \tens x_i)}
    \int \dif^4 z \, \exp\del{\iup \sbr{\sum_i \tens p_i} \cdot \tens z}
    \intertext{%
        This last integral will give us the desired result of a momentum
        conserving $\deltaup$-distribution.
    }
    &= - \iup \lambda \sbr{\prod_i \int \frac{\dif^4 p_i}{[2\piup]^4} \frac{\iup}{\tens
        p_i^2 - m^2 + \iup \epsilon}
    \exp(-\iup \tens p_i \cdot \tens x_i)}
    [2\piup]^4 \deltaup^{(4)}\del{\sum_i \tens p_i}
\end{align*}

Each external line has a factor $\exp(- \iup \tens p \cdot \tens x)$ associated
with it. So this works for external lines just the same as it does for
propagators ending on the vertex.

\paragraph{Overall momentum conservation}

We have momentum conservation in each vertex as derived above. Just like with
Kirchoff's law for the electric current, momentum can be transfered between the
the vertices but is conserved in total.

\paragraph{Fixed internal momenta}

Each propagator brings a momentum with it. If there are too few propagators in
the diagram, everything is fixed. Whenever there is a loop in the diagram,
there is a freedom in momentum there, both can be increased by the same fixed
amount and total momentum is conserved.

\begin{fmffile}{loop}
    \begin{fmfgraph}(100, 40)
        \fmfleft{x}
        \fmfright{y}

        \fmfdot{v1,v2}

        \fmf{fermion}{x,v1}
        \fmf{fermion}{v2,y}
        \fmf{fermion,left,tension=0.7}{v1,v2,v1}
    \end{fmfgraph}
\end{fmffile}

Internal momenta are fixed if there is no closed path that connects vertices.
So if the graph is a tree, all internal momenta are fixed by momentum
conversion.

\end{document}

% vim: spell spelllang=en tw=79
