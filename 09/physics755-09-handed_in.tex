\documentclass[11pt, english, fleqn, DIV=15, headinclude, BCOR=1cm]{scrartcl}

\usepackage[bibatend]{../header}
\usepackage{../my-boxes}

\usepackage{booktabs}
\usepackage{slashed}
\usepackage{simplewick}

\usepackage{feynmp-auto}
\usepackage{adjustbox}
\newenvironment{fmwrapper}{\begin{adjustbox}{margin=5mm}}{\end{adjustbox}}

\usepackage{multicol}
\usepackage{lastpage}

\newcommand\timeorder{\mathscr T}
\newcommand\normorder{\mathscr N}

\hypersetup{
    pdftitle=
}

\newcounter{totalpoints}
\newcommand\punkte[1]{#1\addtocounter{totalpoints}{#1}}

\newcounter{problemset}
\setcounter{problemset}{9}

\subject{physics755 -- Quantum Field Theory}
\ihead{physics755 -- Problem Set \arabic{problemset}}

\title{Problem Set \arabic{problemset}}

\newcommand\thegroup{Group Tuesday -- Ripunjay Acharya}

\publishers{\thegroup}
\ofoot{\thegroup}

\author{
    Martin Ueding \\ \small{\href{mailto:mu@martin-ueding.de}{mu@martin-ueding.de}}
    \and
    Oleg Hamm
}
\ifoot{Martin Ueding, Oleg Hamm}

\ohead{\rightmark}

\begin{document}

\maketitle

\vspace{3ex}

\begin{center}
    \begin{tabular}{rrr}
        problem & achieved points & possible points \\
        \midrule
        \nameref{homework:1} & & \punkte{15} \\
        \midrule
        total & & \arabic{totalpoints}
    \end{tabular}
\end{center}

\vspace{3ex}

\begin{center}
    \begin{small}
        This document consists of \pageref{LastPage} pages.
    \end{small}
\end{center}

\section{The S-Matrix}
\label{homework:1}

\subsection{Momentum conservation}

\begin{remark}
    This problem covers the material also shown by
    \textcite[94]{Peskin/QFT/1995}.
\end{remark}

\paragraph{Momentum conservation at vertex}

Each vertex gives us a factor $- \iup \lambda \int \dif^4 z$, where $\tens z$
is the four-vector for the particular vertex. Each vertex has to have its own
internal integration variable. Then each propagator gives us $D(\tens x - \tens
y)$. Looking at the vertex and all the propagators (labeled with $i$) that end
in this vertex, we get the following term, which we label $A$ to continue
working on it.
\[
    A := - \iup \lambda \int \dif^4 z \, \prod_i D(\tens x_i - \tens z)
\]

To say anything about the momentum, we have to Fourier transform the
propagator. That is given by \textcite[(4.46]{Peskin/QFT/1995} as
\[
    D(\tens x - \tens y) = \int \frac{\dif^4 p}{[2\piup]^4} \frac{\iup}{\tens
    p^2 - m^2 + \iup \epsilon} \exp\del{-\iup \tens p \cdot [\tens x - \tens
    y]}.
\]
We insert this into $A$.
\begin{align*}
    A
    &= - \iup \lambda \int \dif^4 z \, \prod_i \int \frac{\dif^4 p_i}{[2\piup]^4} \frac{\iup}{\tens
    p_i^2 - m^2 + \iup \epsilon} \exp\del{-\iup \tens p_i \cdot [\tens x_i - \tens
    z]}
    \intertext{%
        We can move all the $\tens z$ dependence to the very back of the
        expression to isolate the part which is of interest here.
    }
    &= - \iup \lambda \prod_i \int \frac{\dif^4 p_i}{[2\piup]^4} \frac{\iup}{\tens
    p_i^2 - m^2 + \iup \epsilon}
    \exp(-\iup \tens p_i \cdot \tens x_i)
    \int \dif^4 z \, \exp(\iup \tens p_i \cdot \tens z)
    \intertext{%
        We execute the product for the last term to get the expression that we
        can work with.
    }
    &= - \iup \lambda \sbr{\prod_i \int \frac{\dif^4 p_i}{[2\piup]^4} \frac{\iup}{\tens
        p_i^2 - m^2 + \iup \epsilon}
    \exp(-\iup \tens p_i \cdot \tens x_i)}
    \int \dif^4 z \, \exp\del{\iup \sbr{\sum_i \tens p_i} \cdot \tens z}
    \intertext{%
        This last integral will give us the desired result of a momentum
        conserving $\deltaup$-distribution.
    }
    &= - \iup \lambda \sbr{\prod_i \int \frac{\dif^4 p_i}{[2\piup]^4} \frac{\iup}{\tens
        p_i^2 - m^2 + \iup \epsilon}
    \exp(-\iup \tens p_i \cdot \tens x_i)}
    [2\piup]^4 \deltaup^{(4)}\del{\sum_i \tens p_i}
\end{align*}

Each external line has a factor $\exp(- \iup \tens p \cdot \tens x)$ associated
with it. So this works for external lines just the same as it does for
propagators ending on the vertex.

\paragraph{Overall momentum conservation}

We have momentum conservation in each vertex as derived above. Just like with
Kirchoff's law for the electric current (which follows from the continuity
equation and that $\dot \varrho = 0$ in regular wires), momentum can be
transfered between the vertices but is conserved in total.

\paragraph{Fixed internal momenta}

Each propagator brings a momentum with it. If there are too few propagators in
the diagram, everything is fixed. Whenever there is a loop in the diagram,
there is a freedom in momentum there, both can be increased by the same amount
and total momentum is conserved.

\begin{fmffile}{loop}
    \begin{fmfgraph}(100, 40)
        \fmfleft{x}
        \fmfright{y}

        \fmfdot{v1,v2}

        \fmf{fermion}{x,v1}
        \fmf{fermion}{v2,y}
        \fmf{fermion,left,tension=0.7}{v1,v2,v1}
    \end{fmfgraph}
\end{fmffile}

Internal momenta are fixed if there is no closed path that connects vertices.
So if the graph is a tree, all internal momenta are fixed by momentum
conversion.

\subsection{Momentum space Feynman rules}

\begin{remark}
    This problem covers the material also shown by
    \textcite[95]{Peskin/QFT/1995}.
\end{remark}

The number of vertices should still give the order in $\lambda$, so we will
associate the factor of $- \iup \lambda$ with the vertices.

We have seen in the $A$ that the remainder of the propagator is
\[
    \int \frac{\dif^4 p}{[2\piup]^4} \frac{\iup}{\tens
        p^2 - m^2 + \iup \epsilon}
    \exp(-\iup \tens p \cdot \tens x)
\]
It will be convenient to split this up into several parts. The integration over
$\tens p$ becomes a rule that all internal momenta need to be integrated over
in the end. Each propagator gets the fraction assigned. The exponential can be
assigned to the external point connecting the propagator.

This way we can have the same rules that are given by
\textcite[95]{Peskin/QFT/1995}:
\begin{enumerate}
    \item
        Each propagator gets a factor
        \[
            \frac{\iup}{\tens p^2 - m^2 + \iup \epsilon}
        \]
        assigned.

    \item
        Each vertex gets just the factor $- \iup \lambda$, which then also
        determines the order of the diagram.

    \item
        Each external point gets the exponential $\exp(-\iup \tens p \cdot
        \tens x)$. This reference to a space-time point is a problem since we
        do not want them here. In a momentum space prescription, there should
        not bet external points to begin with. There should only be incoming
        propagators.

    \item
        Since we derived that total momentum at the vertices is conserved, we
        have to apply the derived $\deltaup$-distribution and assert momentum
        conservation at each of the vertices.

    \item
        All internal momenta have to integrated over.

    \item
        And as before, we need to divide by the symmetry factor.
\end{enumerate}

\subsection{Case for Dirac and electromagnetic field}

\begin{remark}
    This problem covers the material also shown by
    \textcite[§4.7f]{Peskin/QFT/1995}.
\end{remark}

Looking at Equation~(7) from the problem set we see that the factor $\exp(-\iup
\tens k \cdot \tens x)$ comes from the application of a positive frequency
operator $\phi_\mathrm I^+$ onto a free momentum state $\ket{\vec k}$. In the
previous subproblem we looked at the Feynman rules for the scalar fields.

\paragraph{Fermions}

The fermions fulfil the Dirac equation and therefore need to be modeled with a
four-component spinor instead of a single scalar field. Also, the fermionic
fields anticommute which needs to be taken into account as well in the
contractions. A sensible way to do this is using a generalized from of Wick's
theorem where the time and normal ordering operators account for the sign
changes.

The Dirac field also has distinct creation and annihilation operators, whereas
the real scalar field only had particles which are their own antiparticles. So
far we only had the real scalar field $\phi$ which occurred in even powers in
the Hamiltonian. Now we have the fields $\bar\psi$ and $\psi$ which occur in
pairs to give bilinears with distinct transformation properties under the
Lorentz group. The only nonzero contraction we can build is from a mixed pair
of those, contractions between $\psi$s or $\bar\psi$s respectively give a zero
propagator.

The “external leg rule” then changes by the way the Dirac fields act on the
pure momentum states:
\begin{align*}
    \contraction{}{\psi}{_\mathrm I(\tens x)}{\ket{\vec p, s}}
    \psi_\mathrm I(\tens x) \ket{\vec k, s}_0
    &= \psi_\mathrm I(\tens x) \sqrt{2 E_{\vec k}} a_{\vec k}^{s\dagger} \ket
    0.
    \intertext{%
        We expand the field $\psi$ in term of positive and negative frequency
        operators. Only the positive frequency part is interesting since we
        evaluate a matrix element with single-momentum states on both sides
        eventually. \textcite[(4.114)]{Peskin/QFT/1995}
    }
    &= \int \frac{\dif^3 k'}{[2\piup]^3} \frac{1}{\sqrt{E_{\vec k'}}} \sum_{s'}
    a_{\vec k'}^{s'} u^{s'}(\tens k') \exp(-\iup \tens k' \cdot \tens x)
    \sqrt{2 E_{\vec k}} a_{\vec k}^{s\dagger} \ket 0.
    \intertext{%
        The anticommutator of the ladder operators will give us a
        $\deltaup$-distribution and we can simplify this to
    }
    &= u^{s}(\tens k) \exp(-\iup \tens k \cdot \tens x) \ket 0.
\end{align*}

Now we have the exact same exponential as there is in Equation~(8) on the
problem set. There is an additional upper spinor component $u^s(\vec k)$. From
this we would conclude that the factor for each external line would be
$u^{s}(\vec k) \exp(-\iup \tens k \cdot \tens x)$.
\textcite[118]{Peskin/QFT/1995} state the “external leg rule” for fermions as
just having $u^s(\tens k)$, though. So the exponential is moved somewhere else?

The same argument can probably be made for the $\bar \psi$ as well, and then
one would have $\bar\psi$ acting on antifermions, which will give a factor
$\bar v^s(\tens k)$.

In both cases, the action of the “hermitian conjugate and $\mat\gamma^0$
version also gets such a bar over the factor in the diagram. So reversal of the
particle number flow adds or removes the bar over the spinor part.

\paragraph{Photons}

As mentioned in front of part~6 of this exercise on the problem set, the
interaction term is given by $H_\text{int} = \iup e \bar\psi \mat\gamma^\mu
\psi A_\mu$. The “external leg rule” is given on the problem set as $- \iup e
\gamma^\mu$. We do not see how this comes from the Hamiltonian.

\subsection{Momentum flow}

In a pair creation event the total particle number is zero, as is the total
momentum. Such a process needs some energetic justification. Maybe it is a
decay of a spin-0 boson into two fermions or photons. The process we have in
mind looks like this:

\begin{fmffile}{pair}
    \begin{fmfgraph}(100, 40)
        \fmfleft{a}
        \fmfright{b}

        \fmfdot{v}

        \fmf{fermion}{v,a}
        \fmf{fermion}{v,b}
    \end{fmfgraph}
\end{fmffile}

We have momentum conservation enforced at the one vertex: $\tens p_1 + \tens
p_2 = \tens 0$. If we just take the absolute values using the directions
defined by the diagram we have $p_1 = p_2$. Since we have no particles in
total, one needs to be the anti-particle of the other one. Both particles move
away from the vertex, so for anti-particles the particle number flow goes
against the direction of the momentum arrow.

In the case of electron-positron-scattering one often has seen a diagram as
such, where the positrons travel backward in time, so to speak:

\begin{fmffile}{electron-positron}
    \begin{fmfgraph}(100,50)
        \fmfleft{e1,p1}
        \fmfright{e2,p2}

        \fmfdot{v1,v2}

        \fmf{fermion}{e1,v1}
        \fmf{fermion}{v2,e2}
        \fmf{fermion}{p2,v2}
        \fmf{fermion}{v1,p1}
        \fmf{photon}{v1,v2}
    \end{fmfgraph}
\end{fmffile}

It was asked for internal lines in the question. We can reformulate our initial
process as a vacuum bubble:

\begin{fmffile}{bubble}
    \begin{fmfgraph}(100,40)
        \fmfleft{a1}
        \fmfright{a2}

        \fmfdot{v1,v2}

        \fmf{phantom}{a1,v1}
        \fmf{phantom}{a2,v2}

        \fmf{fermion,right,tension=0.7}{v1,v2,v1}
    \end{fmfgraph}
\end{fmffile}

The particle number must be conserved at each vertex. We can look at each
vertex and interpret it like one particle is arriving, the other is leaving.
The total number is conserved. Or we say that from each vertex propagates a
particle and an anti-particle.

\subsection{Momentum space Feynman rules for Dirac- and EM-field}

The propagator for the Dirac field is by given by
\textcite[118]{Peskin/QFT/1995} as 
\[
    \frac{\iup [\slashed p + m]}{p^2 - m^2 + \iup \epsilon}.
\]
This differs from the scalar field by the factor $\slashed p + m$. The states
this propagator acts are four-component spinors, the propagator therefore needs
to be a $4 \times 4$ matrix. Something contracted with the Dirac matrices seems
to be a good idea here. Looking at the propagator of the Dirac field in
position space, we see that it exactly has the form we want here.
\parencite[(3.121)]{Peskin/QFT/1995}

The propagator for the electromagnetic field is by given by
\textcite[123]{Peskin/QFT/1995} as 
\[
    \frac{\iup g_{\mu\nu}}{q^2 + \iup \epsilon}.
\]
They say it is difficult to prove but easy to guess. So let us try whether we
could have guessed it. The denominator makes sense since photons are massless.
The imaginary unit in the numerator is also present in every other case. We
have to explain the presence of the metric tensor $\tens g$. The
four components of the electromagnetic field obey the Klein-Gordon equation
individually. So the Klein-Gordon propagator is a good start. Since photons
have a vector polarization and simple propagation cannot change this
polarization, only states with the same polarization can propagate. The metric
tensor $\tens g$ then creates the scalar product between the polarization
vectors of the incoming and outgoing states and gives the propagation overlap.
Seems legit.

\subsection{Leading-order scattering}

There are two possible diagrams, with the convention that time is upwards in
our diagrams. The first one has a repulsion of the two fermions.

\begin{fmffile}{scatter-1}
    \begin{fmfgraph}(100, 100)
        \fmfbottom{i1,i2}
        \fmftop{o1,o2}

        \fmf{fermion}{i1,v1,o1}
        \fmf{fermion}{i2,v2,o2}
        \fmf{photon}{v1,v2}
    \end{fmfgraph}
\end{fmffile}

The other one has some sort of attraction.

\begin{fmffile}{scatter-2}
    \begin{fmfgraph}(100, 100)
        \fmfbottom{i1,i2}
        \fmftop{o1,o2}

        \fmf{fermion}{i1,v1}
        \fmf{fermion}{i2,v2}
        \fmf{phantom}{v1,o1}
        \fmf{phantom}{v2,o2}
        \fmf{fermion,tension=0}{v1,o2}
        \fmf{fermion,tension=0}{v2,o1}
        \fmf{photon}{v1,v2}
    \end{fmfgraph}
\end{fmffile}

This second diagram exchanges the two fermions. Since they are distinguishable,
this would be excluded and only the first one kept.

We apply the rule and we get the following factors:

\begin{itemize}
    \item
        The left incoming particle gives us a factor $u^s(\tens k)$.

    \item
        The right one gives us $u^s(\tens k')$.

    \item
        The two vertices introduce a factor $- \iup e \mat \gamma^\mu$ each, so
        we get $[-\iup e]^2 \mat \gamma^\mu \mat \gamma^\nu$.

    \item
        The photon propagator gives us the term
        \[
            \frac{- \iup g_{\mu\nu}}{\tens q^2 + \iup \epsilon}
        \]
        where $\tens q := \tens p' - \tens p$ is the momentum exchanged.

    \item
        The left outgoing particle gives $\bar u^s(\tens p)$.

    \item
        The right outgoing particle gives $\bar u^s(\tens p')$.
\end{itemize}

Then we have to put everything together and we arrive at the matrix element:
\[
    \iup \mathscr M
    = [-\iup e]^2
    \bar u^s(\tens p')
    \mat \gamma^\nu 
    u^s(\tens k')
    \frac{- \iup g_{\mu\nu}}{\tens q^2 + \iup \epsilon}
    \bar u^s(\tens p)
    \mat \gamma^\mu
    u^s(\tens k)
\]

Perhaps the assigning of the momenta went wrong, but the structure of the
matrix element is the same except for the spin indices that we have. They need
to be summed upon, probably.

\subsection{Momentum space potential}

\begin{remark}
    This problem covers the material also shown by
    \textcite[125]{Peskin/QFT/1995}.
\end{remark}

We continue to use the matrix element we have just computed.
\begin{align*}
    \iup \mathscr M
    &= [-\iup e]^2
    \bar u^s(\tens p')
    \mat \gamma^\nu 
    u^s(\tens p)
    \frac{- \iup g_{\mu\nu}}{\tens q^2 + \iup \epsilon}
    \bar u^s(\tens k')
    \mat \gamma^\mu
    u^s(\tens k)
    \intertext{%
        We move the numbers up to the front.
    }
    &= 
    \frac{\iup e^2 g_{\mu\nu}}{\tens q^2 + \iup \epsilon}
    \bar u^s(\tens p')
    \mat \gamma^\nu 
    u^s(\tens p)
    \bar u^s(\tens k')
    \mat \gamma^\mu
    u^s(\tens k)
\end{align*}

We look at the spinor scalar products. We look at $\nu = 0$ first.
\begin{align*}
    \bar u(\tens p') \mat \gamma^0 u(\tens p)
    &= u^\dagger(\tens p') u(\tens p)
    \intertext{%
        The boosted spinors were given by a factor $\sqrt{\tens p \cdot \tens
        \sigma}$ \parencite[(3.50)]{Peskin/QFT/1995}. So we have
    }
    &=
    \begin{pmatrix}
        \sqrt{\tens p' \cdot \tens \sigma} \vec \xi' \\
        \sqrt{\tens p' \cdot \bar{\tens \sigma}} \vec \xi'
    \end{pmatrix}^\dagger
    \begin{pmatrix}
        \sqrt{\tens p \cdot \tens \sigma} \vec \xi \\
        \sqrt{\tens p \cdot \bar{\tens \sigma}} \vec \xi
    \end{pmatrix}.
    \intertext{%
        Since we assume $\vec p \to 0$, only the zeroth component in the scalar
        products will contribute a sizeable term. The $p^0$ component is the
        energy of the particle. However, we can directly use it as the mass in
        the non-relativistic limit. And although the particles are
        indistinguishable, they have the same mass $m \approx p^0$.
    }
    &=
    \begin{pmatrix}
        \sqrt{p^0 \mat\sigma^0} \vec \xi' \\
        \sqrt{p^0 \mat\sigma^0} \vec \xi'
    \end{pmatrix}^\dagger
    \begin{pmatrix}
        \sqrt{p^0 \mat\sigma^0} \vec \xi \\
        \sqrt{p^0 \mat\sigma^0} \vec \xi
    \end{pmatrix}.
    \intertext{%
        The $\mat\sigma^0$ is just the identity matrix, so this scalar product
        leaves us with
    }
    &= m \vec\xi^{\prime\dagger} \vec\xi.
\end{align*}

One can also show that the case $\nu = 1, 2, 3$ give a vanishing result.

Now we can use that to simplify the matrix element.
\begin{align*}
    \iup \mathscr M
    &= 
    \frac{\iup e^2 g_{\mu\nu}}{\tens q^2 + \iup \epsilon}
    \bar u^s(\tens p')
    \mat \gamma^\nu 
    u^s(\tens p)
    \bar u^s(\tens k')
    \mat \gamma^\mu
    u^s(\tens k) \\
    &= \frac{\iup e^2 g_{00}}{\tens q^2 + \iup \epsilon} 4 m^2 \, [\vec\xi'
    \vec\xi](\tens p) \, [\vec\xi'\vec\xi](\tens k) \\
    \intertext{%
        The zeroth-zeroth element of the metric tensor is just one.
    }
    &= \frac{\iup e^2}{\tens q^2 + \iup \epsilon} 4 m^2 \, [\vec\xi'
    \vec\xi](\tens p) \, [\vec\xi'\vec\xi](\tens k)
\end{align*}

The only interesting term is the fraction, so we have
\[
    \tilde V(\vec q) = \frac{\iup e^2}{\vec q^2 + \iup \epsilon}.
\]
One can Fourier transform this back and then one would get
\[
    V(\vec x) = \frac{e^2}{4 \piup} \frac 1r.
\]

This is a repulsive potential, since in the limit $r \to 0$, the potential
energy goes up.

\subsection{Antifermion}

\begin{remark}
    This problem covers the material also shown by
    \textcite[125]{Peskin/QFT/1995}.
\end{remark}

For the antifermion, we need to use the spinor $v$ instead of $u$. This one is
made up from $(\vec\xi, -\vec\xi)$. This minus sign does not show up in the
spinor scalar product because the $\mat\gamma^0$ exchanges the terms in a way
that they both cancel. In the contractions that we did in the derivation, there
needs to be an additional anticommutation because we want to act with a
$\bar\psi$ to the right. This will give us a minus sign which makes the
potential attractive.

\end{document}

% vim: spell spelllang=en tw=79
