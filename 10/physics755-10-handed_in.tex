\documentclass[11pt, english, fleqn, DIV=15, headinclude, BCOR=1cm]{scrartcl}

\usepackage[bibatend]{../header}
\usepackage{../my-boxes}

\usepackage{booktabs}
\usepackage{slashed}
\usepackage{simplewick}

\usepackage{feynmp-auto}
\usepackage{adjustbox}
\newenvironment{fmwrapper}{\begin{adjustbox}{margin=5mm}}{\end{adjustbox}}

\usepackage{multicol}
\usepackage{lastpage}

\newcommand\timeorder{\mathscr T}
\newcommand\normorder{\mathscr N}

\hypersetup{
    pdftitle=
}

\newcounter{totalpoints}
\newcommand\punkte[1]{#1\addtocounter{totalpoints}{#1}}

\newcounter{problemset}
\setcounter{problemset}{10}

\subject{physics755 -- Quantum Field Theory}
\ihead{physics755 -- Problem Set \arabic{problemset}}

\title{Problem Set \arabic{problemset}}

\newcommand\thegroup{Group Tuesday -- Ripunjay Acharya}

\publishers{\thegroup}
\ofoot{\thegroup}

\author{
    Martin Ueding \\ \small{\href{mailto:mu@martin-ueding.de}{mu@martin-ueding.de}}
    \and
    Oleg Hamm
}
\ifoot{Martin Ueding, Oleg Hamm}

\ohead{\rightmark}

\begin{document}

\maketitle

\vspace{3ex}

\begin{center}
    \begin{tabular}{rrr}
        problem & achieved points & possible points \\
        \midrule
        \nameref{homework:1} & & \punkte{15} \\
        \midrule
        total & & \arabic{totalpoints}
    \end{tabular}
\end{center}

\vspace{3ex}

\begin{center}
    \begin{small}
        This document consists of \pageref{LastPage} pages.
    \end{small}
\end{center}

In the tutorial on 2015-06-23 we were a bit irritated that the momentum space
Feynman rules given by \textcite[95]{Peskin/QFT/1995} since the external legs
had $\exp(-\iup \tens p \cdot \tens x)$ as an additional factor. We quickly
checked another book and found that \textcite[63]{Tong/QFT} does not
explicitly list the external legs, which means that he uses a factor of 1
there.

I think I can shed some light on this issue. The exact wording used is “For
each external \emph{point}” (emphasis mine). When we have such a point, we need
a vector $\tens x$ in position space to get this point into the computation.
The connection between momentum and space is given by $\exp(-\iup \tens p \cdot
\tens x)$ then. Looking at \parencite[115]{Peskin/QFT/1995} where the rules for
computing $\iup \mathscr M$ are listed the wording is “For each external
\emph{line}” (emphasis mine) and a factor of 1. Three pages later,
\textcite[118]{Peskin/QFT/1995} gives the Feynman rules for fermions, and there
is no exponential factor either.

So I would suggest that the rules that we got from
\textcite[95]{Peskin/QFT/1995} are not wrong in any sense. They are the wrong
rules to use if we want to look at diagrams purely in the momentum state and
not in the transition from position to momentum space like we did on the last
homework problem set where we derived, or rather guessed, the rules for the
momentum space. Taking away the external \emph{points} will take away the
exponentials.

\section{Bhabha scattering}
\label{homework:1}

\subsection{Contributing diagrams}

Since we have two different kind of particles, they are distinguishable. This
limits the number of diagrams that contribute to this process. We take the
convention that the time goes to the right.

The first diagram has the annihilation of the particles which creates a real
(because it is time-like) photon. Then this one creates a pair of electron of
positron again:

\begin{fmffile}{feyn-scatter-1}
    \begin{fmfgraph}(100, 50)
        \fmfleft{ei,pi}
        \fmfright{eo,po}
        \fmfdot{v1,v2}

        \fmf{fermion}{ei,v1}
        \fmf{fermion}{v1,pi}

        \fmf{fermion}{v2,eo}
        \fmf{fermion}{po,v2}

        \fmf{photon}{v1,v2}
    \end{fmfgraph}
\end{fmffile}

The factors that we get for the parts of each diagram will be the same for the
second diagram. However, the momentum conservation at the vertices will bind
different momenta together. A different ordering of the terms might give a
different overall sign.

The factors that we get for the first diagram are:
\begin{itemize}
    \item The incoming fermion gives $u^s(\tens p_1)$.
    \item The incoming antifermion gives $\bar v^s(\tens p_2)$.
    \item The outgoing fermion gives $\bar u^r(\tens k_1)$.
    \item The outgoing antifermion gives $v^r(\tens k_2)$.
    \item The left vertex gives $- \iup e \mat\gamma^\mu$.
    \item The right vertex gives $- \iup e \mat\gamma^\nu$.
    \item The propagator gives $\iup \eta_{\mu\nu}/\tens q^2$.
\end{itemize}

We need to impose momentum conservation at each vertex, this gives us $\tens
p_1 + \tens p_2 = \tens q$ and $\tens q = \tens k_1 + \tens k_2$.
There is no undetermined momentum left since we have a Feynman graph which is a
tree (graph theory). Is it a Feynman tree, then?

Taking all those terms together we obtain
\[
    [-\iup e]^2 \bar v^s(\tens p_2) \mat\gamma^\mu u^s(\tens p_1) \frac{\iup
    \eta_{\mu\nu}}{\tens q^2} \bar u^r(\tens k_1) \mat\gamma^\nu v^r(\tens
    k_2).
\]

To figure out the sign we look at the second order of the perturbation
expansion. For QED, the interacting Hamiltonian density is $e \bar\psi
\mat\gamma^\mu \psi A_\mu$. In second order we have this twice. So the core of
our expression would be something along the lines of
\[
    \bra 0 b_{\tens k_2} a_{\tens k_1} \; \bar\psi(\tens x) \mat\gamma^\mu
    \psi(\tens x) A_\mu(\tens x) \; \bar\psi(\tens y) \mat\gamma^\nu A_\mu(\tens
    y) \; a_{\tens p_1}^\dagger b_{\tens p_2}^\dagger \ket 0.
\]
Our first diagram corresponds to the following contraction:
\[
    \contraction{\bra 0}{b_{\tens k_2}}{a_{\tens k_1} \; \bar\psi(\tens x)
    \mat\gamma^\mu}{\psi}
    %
    \contraction{\bra 0 b_{\tens k_2}}{a_{\tens k_1}}{\;}{\bar\psi}
    %
    \contraction{\bra 0 b_{\tens k_2} a_{\tens k_1} \; \bar\psi(\tens x)
    \mat\gamma^\mu \psi(\tens x) A_\mu(\tens x) \;}{\bar\psi}{(\tens y)
    \mat\gamma^\nu \psi(\tens y) A_\mu(\tens y) \; a_{\tens
    p_1}^\dagger}{b_{\tens p_2}^\dagger}
    %
    \contraction{\bra 0 b_{\tens k_2} a_{\tens k_1} \; \bar\psi(\tens x)
    \mat\gamma^\mu \psi(\tens x) A_\mu(\tens x) \; \bar\psi(\tens y)
    \mat\gamma^\nu}{\psi}{(\tens y) A_\mu(\tens y) \;}{a_{\tens p_1}^\dagger}
    %
    \bra 0 b_{\tens k_2} a_{\tens k_1} \; \bar\psi(\tens x) \mat\gamma^\mu
    \psi(\tens x) A_\mu(\tens x) \; \bar\psi(\tens y) \mat\gamma^\nu \psi(\tens
    y) A_\mu(\tens y) \; a_{\tens p_1}^\dagger b_{\tens p_2}^\dagger \ket 0.
\]
None of the contractions intersect, there are no further modifications needed
to untangle this. This diagram has a no additional sign changes.

There is no symmetry factor to account for in this diagram.

The second diagram is the more natural way of thinking about scattering. The
two particles scatter by the exchange of a virtual photon:

\begin{fmffile}{feyn-scatter-2}
    \begin{fmfgraph}(100, 50)
        \fmfleft{ei,pi}
        \fmfright{eo,po}
        \fmfdot{v1,v2}

        \fmf{fermion}{ei,v1,eo}
        \fmf{fermion}{po,v2,pi}

        \fmf{photon}{v1,v2}
    \end{fmfgraph}
\end{fmffile}

We can take the same exact factors and build the invariant matrix element from
this. The only difference is the position of the vertices. This only changes
the internal momentum $\tens q$. Momentum conservation now gives us at the
lower vertex
\[
    \tens p_2 = \tens q + \tens k_2.
\]
The upper vertex will give us
\[
    \tens p_1 = - \tens q + \tens k_1.
\]
So we have $\tens q = \tens p_2 - \tens k_2 = \tens k_1 - \tens p_1$.

We can directly go and look at the contractions.
\[
    \acontraction{\bra 0 b_{\tens k_2}}{a_{\tens k_1}}{\; \bar\psi(\tens x)
    \mat\gamma^\mu \psi(\tens x) A_\mu(\tens x) \;}{\bar\psi}
    %
    \bcontraction{\bra 0}{b_{\tens k_2}}{a_{\tens k_1} \; \bar\psi(\tens x)
    \mat\gamma^\mu}{\psi}
    %
    \bcontraction[2ex]{\bra 0 b_{\tens k_2} a_{\tens k_1} \;}{\bar\psi}{(\tens x)
    \mat\gamma^\mu \psi(\tens x) A_\mu(\tens x) \; \bar\psi(\tens y)
    \mat\gamma^\nu \psi(\tens y) A_\mu(\tens y) \; a_{\tens
    p_1}^\dagger}{b_{\tens p_2}^\dagger}
    %
    \acontraction{\bra 0 b_{\tens k_2} a_{\tens k_1} \; \bar\psi(\tens x)
    \mat\gamma^\mu \psi(\tens x) A_\mu(\tens x) \; \bar\psi(\tens y)
    \mat\gamma^\nu}{\psi}{(\tens y) A_\mu(\tens y) \;}{a_{\tens p_1}^\dagger}
    %
    \bra 0 b_{\tens k_2} a_{\tens k_1} \; \bar\psi(\tens x) \mat\gamma^\mu
    \psi(\tens x) A_\mu(\tens x) \; \bar\psi(\tens y) \mat\gamma^\nu \psi(\tens
    y) A_\mu(\tens y) \; a_{\tens p_1}^\dagger b_{\tens p_2}^\dagger \ket 0.
\]
The $\bar\psi(\tens y)$ in the middle of the expression needs to be moved in
front of the $\psi(\tens x)$ to untangle the lower contraction. This is one
anticommutation, although we will get an additional term with
$\deltaup^{(4)}(\tens x - \tens y)$. After this exchange, we need to move the
$\bar\psi(\tens x)$ behind the $\psi(\tens x)$. This now needs two
anticommutations since we already moved the $\bar\psi(\tens y)$ into there.
This will give us no sign change but an additional term with
$\deltaup^{(4)}(\tens 0)$ which looks a bit problematic.

Ignoring those extra commutators, we have the matrix element twice with
different values for $\tens q$ and the second one with one minus sign with
respect to the first one.

\subsection{Identities}

\paragraph{First batch}

All the identities follow the same pattern. The first one is for different $u$
and $v$, setting one to the other will give the second and third identity. We
start by expanding the “bar”.
\begin{align*}
    \sbr{\bar v(\tens p) \mat\gamma^{\mu} u(\tens k)}^*
    &= \sbr{v^\dagger(\tens p) \mat\gamma^0 \mat\gamma^{\mu} u(\tens k)}^*
    \intertext{%
        From the anticommutation of the two Dirac matrices, we get a factor
        $\eta_{\mu\mu}$.
    }
    &= \eta_{\mu\mu} \sbr{v^\dagger(\tens p) \mat\gamma^{\mu} \mat\gamma^0 u(\tens k)}^*
    \intertext{%
        Then we can absorb the $\mat\gamma^0$ into the $u$ and give it a “bar”.
    }
    &= \eta_{\mu\mu} \sbr{v^\dagger(\tens p) \mat\gamma^{\mu} \bar
    u^\dagger(\tens k)}^*
    \intertext{%
        The square bracket contains a scalar. For such, we can always add a
        transpose operation which does not change the value of the scalar.
        Together with the complex conjugate we will have a hermitian
        conjugation.
    }
    &= \eta_{\mu\mu} \sbr{v^\dagger(\tens p) \mat\gamma^{\mu} \bar
    u^\dagger(\tens k)}^\dagger \\
    &= \eta_{\mu\mu} u^\dagger(\tens k) \mat\gamma^{\mu} \bar
    v(\tens p)
    \intertext{%
        This is almost the expression that we want. The Pauli matrices
        $\mat\sigma^\mu$ are
        hermitian. The Dirac matrices $\mat\gamma^\mu$ are build from those,
        except with an additional minus sign for $\mu \in \{1, 2, 3\}$. Those
        last three matrices are antihermitian, which will introduce a factor
        $\eta_{\mu\mu}$ again.
    }
    &= [\eta_{\mu\mu}]^2 u^\dagger(\tens k) \mat\gamma^{\mu} \bar
    v(\tens p)
    \intertext{%
        And since that factor can only be $\pm 1$, it will be squared away. We
        arrive at the desired term.
    }
    &= u^\dagger(\tens k) \mat\gamma^{\mu} \bar v(\tens p)
\end{align*}

\paragraph{Second batch}

The wording sounds like we are allowed to just use the completeness relations
without proof. So they are given by \textcite[49]{Peskin/QFT/1995}:
\[
    \sum_s u^s(\tens p) \bar u^s(\tens p) = \slashed{\tens p} + m
    \eqnsep
    \sum_s v^s(\tens p) \bar v^s(\tens p) = \slashed{\tens p} - m.
\]
In case there is only a blob of ink to see on the inkjet printout, that is a
slashed four-vector $p$ right after the equality sign.

We start with the first equation. We can move the summation over $s$ in the
middle and group the term for the next step.
\begin{align*}
    \sum_{s, s'} \bar u^{s'}(\tens p') \mat\gamma^\mu u^s(\tens p)
    \bar u^s(\tens p) \mat\gamma^\nu u^{s'} (\tens p')
    &= \sum_{s'} \bar u^{s'}(\tens p') \mat\gamma^\mu
    \sbr{\sum_s u^s(\tens p) \bar u^s(\tens p)}
    \mat\gamma^\nu u^{s'} (\tens p')
    \intertext{%
        Here we can use the completeness relation and replace the contents of
        the bracket. We will explicitly write $m \mat 1_4$ since adding a
        scalar to a matrix feels a bit imprecise.
    }
    &= \sum_{s'} \bar u^{s'}(\tens p') \mat\gamma^\mu
    \sbr{\slashed{\tens p} + m \mat 1_4}
    \mat\gamma^\nu u^{s'} (\tens p')
    \intertext{%
        We add spinor indices (is that correct?) to the spinors such that we
        can move them around freely.
    }
    &= \sum_{s'} \bar u^{s'}{}_a (\tens p') \sbr{\mat\gamma^\mu
    \sbr{\slashed{\tens p} + m \mat 1_4}
    \mat\gamma^\nu}^a{}_b [u^{s'}]^b (\tens p')
    \intertext{%
        The $u$ can be moved to the very front of the equation. Everything that
        depends on $s'$ is in the first terms, we add another bracket to
        isolate it for the next step.
    }
    &= \sbr{\sum_{s'} [u^{s'}]^b (\tens p') \bar u^{s'}{}_a (\tens p')}
    \sbr{\mat\gamma^\mu
    \sbr{\slashed{\tens p} + m \mat 1_4}
    \mat\gamma^\nu}^a{}_b
    \intertext{%
        We think that we can still apply the completeness relation since we sum
        over $s'$ first and then over the spinor indices $a$ and $b$.
    }
    &= \sbr{\slashed{\tens p}' + m \mat 1_4}^b{}_a
    \sbr{\mat\gamma^\mu
    \sbr{\slashed{\tens p} + m \mat 1_4}
    \mat\gamma^\nu}^a{}_b
    \intertext{%
        Then the summation over both indices will result in a matrix
        multiplication ($b$) and a trace ($a$).
    }
    &= \tr\del{\sbr{\slashed{\tens p}' + m \mat 1_4} \mat\gamma^\mu
    \sbr{\slashed{\tens p} + m \mat 1_4} \mat\gamma^\nu}
\end{align*}

The second identity has a minus sign in the $\slashed{\tens p}'$ term since
there are $v$ spinors outside. The third one has the minus sign with the
$\slashed{\tens p}$ term since the inner spinors are $v$ instead of $u$. And in
the last one there are only $v$, so there are minus signs in both spots.

\paragraph{Third batch}

This batch contains of assorted identities, we will just iterate over them.

\begin{enumerate}
    \item
        The identity $\tr(\mat 1) = 4$ is a bit weird. Either we assume that
        $\mat 1$ is a $4 \times 4$ matrix and this is rather trivial, or there
        is something very deep to it. The identity matrix is the one which only
        has ones on the diagonal. Therefore the trace is the dimension of that
        matrix.

    \item
        The Dirac matrices can be written in the chiral representation like
        this:
        \[
            \mat\gamma^\mu =
            \begin{pmatrix}
                0 & \mat\sigma^\mu \\
                \bar{\mat\sigma}^\nu & 0
            \end{pmatrix}.
        \]
        Forming the product of two such Dirac matrices will yield a block
        diagonal matrix:
        \[
            \mat\gamma^\mu \mat\gamma^\nu =
            \begin{pmatrix}
                \mat\sigma^\mu \bar{\mat\sigma}^\nu & 0 \\
                0 & \bar{\mat\sigma}^\mu \mat\sigma^\nu \\
            \end{pmatrix}.
        \]

        Any even number of Dirac matrices will give a block diagonal matrix.
        Adding one more Dirac matrix will make it an off-diagonal matrix which
        has no trace. Therefore
        \[
            \tr\del{\prod_{i = 1}^{2n+1} \mat\gamma^{\mu_i}} = 0 \qquad
            \text{with $n \in \N$}.
        \]

    \item
        We start with the anticommutation relation of the Dirac matrices. This
        gives us
        \[
            \tr(\mat\gamma^\mu \mat\gamma^\nu)
            = \tr(- \mat\gamma^\nu \mat\gamma^\mu + 2 \eta^{\mu\nu} \mat 1_4)
            = - \tr(\mat\gamma^\nu \mat\gamma^\mu) + 8 \eta^{\mu\nu},
        \]
        where we have used the linearity and the first identity in the last
        step.
        The trace is cyclic, so we also know that
        \[
            \tr(\mat\gamma^\mu \mat\gamma^\nu)
            = \tr(\mat\gamma^\nu \mat\gamma^\mu).
        \]
        We can now rewrite the first equation as
        \[
            2 \tr(\mat\gamma^\mu \mat\gamma^\nu)
            = 8 \eta^{\mu\nu},
        \]
        which will give the desired relation after dividing by a factor of two:
        \[
            \tr(\mat\gamma^\mu \mat\gamma^\nu)
            = 4 \eta^{\mu\nu}.
        \]

    \item
        We will group the matrices again in pairs and use the relation derived
        for the second identity.
        \begin{align*}
            \tr(\mat\gamma^\mu \mat\gamma^\nu \mat\gamma^\rho
            \mat\gamma^\sigma)
            &= \tr\del{
                \diag\del{
                    \mat\sigma^\mu \bar{\mat\sigma}^\nu,
                    \bar{\mat\sigma}^\mu \mat\sigma^\nu
                }
                \diag\del{
                    \mat\sigma^\rho \bar{\mat\sigma}^\sigma,
                    \bar{\mat\sigma}^\rho \mat\sigma^\sigma
                }
            }
            \intertext{%
                The product of the two block-diagonal matrices gives another
                block-diagonal matrix.
            }
            &= \tr\del{
                \diag\del{
                    \mat\sigma^\mu \bar{\mat\sigma}^\nu
                    \mat\sigma^\rho \bar{\mat\sigma}^\sigma,
                    \bar{\mat\sigma}^\mu \mat\sigma^\nu
                    \bar{\mat\sigma}^\rho \mat\sigma^\sigma
                }
            }
            \intertext{%
                The trace is the sum of the traces of this block-diagonal
                matrix.
            }
            &= \tr\del{
                \mat\sigma^\mu \bar{\mat\sigma}^\nu
                \mat\sigma^\rho \bar{\mat\sigma}^\sigma
            }
            + \tr\del{
                \bar{\mat\sigma}^\mu \mat\sigma^\nu
                \bar{\mat\sigma}^\rho \mat\sigma^\sigma
            }
            \intertext{%
                The only Pauli matrix that has a trace is the $\mat\sigma^0$
                matrix. All other ones do not have a trace. This also means
                that they can be used as the generators of a special matrix
                group. If we square the Pauli matrices, we obtain a matrix that
                is proportional to the identity matrix. The “bar” will give a
                minus sign for the spatial indices, so we need to take that
                into account. If we have $\mu$, $\nu$, $\rho$ and $\sigma$ form
                two pairs, we only have matrices which are proportional to the
                identity. The trace will be $\pm 4$ then. A metric tensor will
                allow us to cope with the minus signs by the “bar”. So we have
                three possibilities to “contract” the four matrices. Looking at
                the signs that we get, we can write the result as
            }
            &= 4 \sbr{\eta^{\mu\nu} \eta^{\rho\sigma} + \eta^{\mu\sigma}
            \eta^{\nu\rho} - \eta^{\mu\rho} \eta^{\nu\sigma}}.
        \end{align*}

        There are no other possibilities to get a non-zero trace. The product
        of two non-equal Pauli matrices is a third one which will have no
        trace. So if only two indices match up, we have a diagonal matrix times
        a traceless matrix which will be traceless. If no index matches up, the
        product of the four factors will be a single spatial Pauli matrix which
        has no trace either.

    \item
        The $\tr(\mat \gamma^5)$ is zero because the matrix is given by
        \[
            \mat\gamma^5 = \diag(1, 1, -1, -1)
        \]
        in the representation that we usually use.

    \item
        If $\mu = \nu$ then the product $[\mat\gamma^\mu]^2$ will be
        $\eta_{\mu\mu} \mat 1_4$. Together with $\mat\gamma^5$ this will be a
        traceless matrix. In the other case that $\mu \neq \nu$ the product
        will be an off-diagonal matrix, as shown for the second identity. The
        multiplication with $\mat\gamma^5$ will change the signs, but the main
        diagonal will not be populated. The matrix does not change a trace
        either.
\end{enumerate}

\end{document}

% vim: spell spelllang=en tw=79
