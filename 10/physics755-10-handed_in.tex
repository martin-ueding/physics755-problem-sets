\documentclass[11pt, english, fleqn, DIV=15, headinclude, BCOR=1cm]{scrartcl}

\usepackage[bibatend]{../header}
\usepackage{../my-boxes}

\usepackage{booktabs}
\usepackage{slashed}
\usepackage{simplewick}

\usepackage{feynmp-auto}
\usepackage{adjustbox}
\newenvironment{fmwrapper}{\begin{adjustbox}{margin=5mm}}{\end{adjustbox}}

\usepackage{multicol}
\usepackage{lastpage}

\newcommand\timeorder{\mathscr T}
\newcommand\normorder{\mathscr N}

\hypersetup{
    pdftitle=
}

\newcounter{totalpoints}
\newcommand\punkte[1]{#1\addtocounter{totalpoints}{#1}}

\newcounter{problemset}
\setcounter{problemset}{10}

\subject{physics755 -- Quantum Field Theory}
\ihead{physics755 -- Problem Set \arabic{problemset}}

\title{Problem Set \arabic{problemset}}

\newcommand\thegroup{Group Tuesday -- Ripunjay Acharya}

\publishers{\thegroup}
\ofoot{\thegroup}

\author{
    Martin Ueding \\ \small{\href{mailto:mu@martin-ueding.de}{mu@martin-ueding.de}}
    \and
    Oleg Hamm
}
\ifoot{Martin Ueding, Oleg Hamm}

\ohead{\rightmark}

\begin{document}

\maketitle

\vspace{3ex}

\begin{center}
    \begin{tabular}{rrr}
        problem & achieved points & possible points \\
        \midrule
        \nameref{homework:1} & & \punkte{15} \\
        \midrule
        total & & \arabic{totalpoints}
    \end{tabular}
\end{center}

\vspace{3ex}

\begin{center}
    \begin{small}
        This document consists of \pageref{LastPage} pages.
    \end{small}
\end{center}

In the tutorial on 2015-06-23 we were a bit irritated that the momentum space
Feynman rules given by \textcite[95]{Peskin/QFT/1995} since the external legs
had $\exp(-\iup \tens p \cdot \tens x)$ as an additional factor. We quickly
checked another book and found that \textcite[63]{Tong/QFT} does not
explicitly list the external legs, which means that he uses a factor of 1
there.

I think I can shed some light on this issue. The exact wording used is “For
each external \emph{point}” (emphasis mine). When we have such a point, we need
a vector $\tens x$ in position space to get this point into the computation.
The connection between momentum and space is given by $\exp(-\iup \tens p \cdot
\tens x)$ then. Looking at \parencite[115]{Peskin/QFT/1995} where the rules for
computing $\iup \mathscr M$ are listed the wording is “For each external
\emph{line}” (emphasis mine) and a factor of 1. Three pages later,
\textcite[118]{Peskin/QFT/1995} gives the Feynman rules for fermions, and there
is no exponential factor either.

So I would suggest that the rules that we got from
\textcite[95]{Peskin/QFT/1995} are not wrong in any sense. They are the wrong
rules to use if we want to look at diagrams purely in the momentum state and
not in the transition from position to momentum space like we did on the last
homework problem set where we derived, or rather guessed, the rules for the
momentum space. Taking away the external \emph{points} will take away the
exponentials.

\section{Bhabha scattering}
\label{homework:1}

\subsection{Contributing diagrams}

Since we have two different kind of particles, they are distinguishable. This
limits the number of diagrams that contribute to this process. We take the
convention that the time goes to the right.

The first diagram has the annihilation of the particles which creates a real
(because it is time-like) photon. Then this one creates a pair of electron of
positron again:

\begin{fmffile}{feyn-scatter-1}
    \begin{fmfgraph}(100, 50)
        \fmfleft{ei,pi}
        \fmfright{eo,po}
        \fmfdot{v1,v2}

        \fmf{fermion}{ei,v1}
        \fmf{fermion}{v1,pi}

        \fmf{fermion}{v2,eo}
        \fmf{fermion}{po,v2}

        \fmf{photon}{v1,v2}
    \end{fmfgraph}
\end{fmffile}

The factors that we get for the parts of each diagram will be the same for the
second diagram. However, the momentum conservation at the vertices will bind
different momenta together. A different ordering of the terms might give a
different overall sign.

The factors that we get for the first diagram are:
\begin{itemize}
    \item The incoming fermion gives $u^s(\tens p_1)$.
    \item The incoming antifermion gives $\bar v^s(\tens p_2)$.
    \item The outgoing fermion gives $\bar u^r(\tens k_1)$.
    \item The outgoing antifermion gives $v^r(\tens k_2)$.
    \item The left vertex gives $- \iup e \mat\gamma^\mu$.
    \item The right vertex gives $- \iup e \mat\gamma^\nu$.
    \item The propagator gives $\iup \eta_{\mu\nu}/\tens q^2$.
\end{itemize}

We need to impose momentum conservation at each vertex, this gives us $\tens
p_1 + \tens p_2 = \tens q$ and $\tens q = \tens k_1 + \tens k_2$.
There is no undetermined momentum left since we have a Feynman graph which is a
tree (graph theory). Is it a Feynman tree, then?

Taking all those terms together we obtain
\[
    [-\iup e]^2 \bar v^s(\tens p_2) \mat\gamma^\mu u^s(\tens p_1) \frac{\iup
    \eta_{\mu\nu}}{\tens q^2} \bar u^r(\tens k_1) \mat\gamma^\nu v^r(\tens
    k_2).
\]

To figure out the sign we look at the second order of the perturbation
expansion. For QED, the interacting Hamiltonian density is $e \bar\psi
\mat\gamma^\mu \psi A_\mu$. In second order we have this twice. So the core of
our expression would be something along the lines of
\[
    \bra 0 b_{\tens k_2} a_{\tens k_1} \; \bar\psi(\tens x) \mat\gamma^\mu
    \psi(\tens x) A_\mu(\tens x) \; \bar\psi(\tens y) \mat\gamma^\nu A_\mu(\tens
    y) \; a_{\tens p_1}^\dagger b_{\tens p_2}^\dagger \ket 0.
\]
Our first diagram corresponds to the following contraction:
\[
    \contraction{\bra 0}{b_{\tens k_2}}{a_{\tens k_1} \; \bar\psi(\tens x)
    \mat\gamma^\mu}{\psi}
    %
    \contraction{\bra 0 b_{\tens k_2}}{a_{\tens k_1}}{\;}{\bar\psi}
    %
    \contraction{\bra 0 b_{\tens k_2} a_{\tens k_1} \; \bar\psi(\tens x)
    \mat\gamma^\mu \psi(\tens x) A_\mu(\tens x) \;}{\bar\psi}{(\tens y)
    \mat\gamma^\nu \psi(\tens y) A_\mu(\tens y) \; a_{\tens
    p_1}^\dagger}{b_{\tens p_2}^\dagger}
    %
    \contraction{\bra 0 b_{\tens k_2} a_{\tens k_1} \; \bar\psi(\tens x)
    \mat\gamma^\mu \psi(\tens x) A_\mu(\tens x) \; \bar\psi(\tens y)
    \mat\gamma^\nu}{\psi}{(\tens y) A_\mu(\tens y) \;}{a_{\tens p_1}^\dagger}
    %
    \bra 0 b_{\tens k_2} a_{\tens k_1} \; \bar\psi(\tens x) \mat\gamma^\mu
    \psi(\tens x) A_\mu(\tens x) \; \bar\psi(\tens y) \mat\gamma^\nu \psi(\tens
    y) A_\mu(\tens y) \; a_{\tens p_1}^\dagger b_{\tens p_2}^\dagger \ket 0.
\]
None of the contractions intersect, there are no further modifications needed
to untangle this. This diagram has a no additional sign changes.

There is no symmetry factor to account for in this diagram.

The second diagram is the more natural way of thinking about scattering. The
two particles scatter by the exchange of a virtual photon:

\begin{fmffile}{feyn-scatter-2}
    \begin{fmfgraph}(100, 50)
        \fmfleft{ei,pi}
        \fmfright{eo,po}
        \fmfdot{v1,v2}

        \fmf{fermion}{ei,v1,eo}
        \fmf{fermion}{po,v2,pi}

        \fmf{photon}{v1,v2}
    \end{fmfgraph}
\end{fmffile}

We can take the same exact factors and build the invariant matrix element from
this. The only difference is the position of the vertices. This only changes
the internal momentum $\tens q$. Momentum conservation now gives us at the
lower vertex
\[
    \tens p_2 = \tens q + \tens k_2.
\]
The upper vertex will give us
\[
    \tens p_1 = - \tens q + \tens k_1.
\]
So we have $\tens q = \tens p_2 - \tens k_2 = \tens k_1 - \tens p_1$.

We can directly go and look at the contractions.
\[
    \acontraction{\bra 0 b_{\tens k_2}}{a_{\tens k_1}}{\; \bar\psi(\tens x)
    \mat\gamma^\mu \psi(\tens x) A_\mu(\tens x) \;}{\bar\psi}
    %
    \bcontraction{\bra 0}{b_{\tens k_2}}{a_{\tens k_1} \; \bar\psi(\tens x)
    \mat\gamma^\mu}{\psi}
    %
    \bcontraction[2ex]{\bra 0 b_{\tens k_2} a_{\tens k_1} \;}{\bar\psi}{(\tens x)
    \mat\gamma^\mu \psi(\tens x) A_\mu(\tens x) \; \bar\psi(\tens y)
    \mat\gamma^\nu \psi(\tens y) A_\mu(\tens y) \; a_{\tens
    p_1}^\dagger}{b_{\tens p_2}^\dagger}
    %
    \acontraction{\bra 0 b_{\tens k_2} a_{\tens k_1} \; \bar\psi(\tens x)
    \mat\gamma^\mu \psi(\tens x) A_\mu(\tens x) \; \bar\psi(\tens y)
    \mat\gamma^\nu}{\psi}{(\tens y) A_\mu(\tens y) \;}{a_{\tens p_1}^\dagger}
    %
    \bra 0 b_{\tens k_2} a_{\tens k_1} \; \bar\psi(\tens x) \mat\gamma^\mu
    \psi(\tens x) A_\mu(\tens x) \; \bar\psi(\tens y) \mat\gamma^\nu \psi(\tens
    y) A_\mu(\tens y) \; a_{\tens p_1}^\dagger b_{\tens p_2}^\dagger \ket 0.
\]
The $\bar\psi(\tens y)$ in the middle of the expression needs to be moved in
front of the $\psi(\tens x)$ to untangle the lower contraction. This is one
anticommutation, although we will get an additional term with
$\deltaup^{(4)}(\tens x - \tens y)$. After this exchange, we need to move the
$\bar\psi(\tens x)$ behind the $\psi(\tens x)$. This now needs two
anticommutations since we already moved the $\bar\psi(\tens y)$ into there.
This will give us no sign change but an additional term with
$\deltaup^{(4)}(\tens 0)$ which looks a bit problematic.

Ignoring those extra commutators, we have the matrix element twice with
different values for $\tens q$ and the second one with one minus sign with
respect to the first one.

\end{document}

% vim: spell spelllang=en tw=79
