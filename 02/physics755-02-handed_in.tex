\documentclass[11pt, english, fleqn, DIV=15, headinclude, BCOR=1cm]{scrartcl}

\usepackage[bibatend, color]{../header}
\usepackage{../my-boxes}

\usepackage{booktabs}

\hypersetup{
    pdftitle=
}

\newcounter{totalpoints}
\newcommand\punkte[1]{#1\addtocounter{totalpoints}{#1}}

\newcounter{problemset}
\setcounter{problemset}{2}

\subject{physics755 -- Quantum Field Theory}
\ihead{physics755 -- Problem Set \arabic{problemset}}

\title{Problem Set \arabic{problemset}}

\newcommand\thegroup{Group Tuesday -- Ripunjay Acharya}

\publishers{\thegroup}
\ofoot{\thegroup}

\author{
    Martin Ueding \\ \small{\href{mailto:mu@martin-ueding.de}{mu@martin-ueding.de}}
}
\ifoot{Martin Ueding}

\ohead{\rightmark}

\begin{document}

\maketitle

\vspace{3ex}

\begin{center}
    \begin{tabular}{rrr}
        problem & achieved points & possible points \\
        \midrule
        \nameref{homework:1} & & \punkte{10} \\
        \midrule
        total & & \arabic{totalpoints}
    \end{tabular}
\end{center}

\section{Poincaré algebra}
\label{homework:1}

\subsection{Generators of SO(1, 3) and SO(3)}

The $m^\mu{}_\nu$ and $t^i{}_j$ are the generators in the Physicist's
convention of hermitian generators since the Taylor expansion contains an
imaginary unit.

Orthogonal transformations from SO(1, 3), which are Lorentz transformations or
from SO(3), which are rotations in three dimensions, have to leave the metric
tensor $\eta$ invariant. I will solve this for the general case of
$\mathrm{SO}(p, q)$ and relate that to both special cases. The metric of
interest in those spaces has signature $(p, q)$, that means $p$ positive and
$q$ negative eigenvalues. Since we are only concerned with special relativity
here, the metric tensor always has the special form
\[
    \tens \eta = \diag(\underbrace{1, \ldots, 1}_\text{$p$ times},
    \underbrace{-1, \ldots, -1}_\text{$q$ times}).
\]

Let $\tens A$ be the transformation matrix. Then the invariance of the metric
is written in index notation as
\begin{align*}
    \eta_{ab}
    &= \eta_{ij} [A\inv]^i{}_a [A\inv]^j{}_b.
    \intertext{%
        The transformation can now be written in terms of a generator
        expression with the generators $\tens T_k$ of the group.
    }
    &= \eta_{ij} \exp\del{-\iup \epsilon^k \tens T_k}^i{}_a
    \exp\del{-\iup \epsilon^k \tens T_k}^j{}_b
    \intertext{%
        In the first order of $\vec \epsilon$, this can be linearized to give a
        form like the one given in the problem statement. For those orthogonal
        groups, there are multiple generators. Here it suffices to look at one
        of those generators since they all have the same properties. Linear
        combinations are still generators, the generators form a vector space.
        This vector space is the tangent space of the group manifold around the
        identity element. The problem statement basically has $\epsilon \tens m
        = \epsilon^k \tens T_k$. I will now just look at one of the generators
        $\tens T$ without an index.
    }
    &= \eta_{ij} \sbr{\deltaup^i_a - \iup \epsilon T^i{}_a}
    \sbr{\deltaup^j_b - \iup \epsilon T^j{}_b} + \mathrm O(\epsilon^2)
    \intertext{%
        Then I can expand those brackets. I will drop the fourth summand since
        it is of second order already.
    }
    &= \eta_{ij} \deltaup^i_a \deltaup^j_b - \iup \epsilon \eta_{ij} 
    \sbr{T^i{}_a \deltaup^j_b + \deltaup^i_a T^j{}_b}
    + \mathrm O(\epsilon^2)
    \intertext{%
        I will clean up the indices.
    }
    &= \eta_{ab} - \iup \epsilon \sbr{\eta_{ib} T^i{}_a + \eta_{aj} T^j{}_b}
\end{align*}
This equation now leads to a shorter one:
\begin{align*}
    \eta_{ib} T^i{}_a + \eta_{aj} T^j{}_b &= 0.
    \intertext{%
        I can even pull down the index and write this as:
    }
    T_{ba} + T_{ab} &= 0.
\end{align*}

Using the (anti)symmetrization notation by
\textcite{penrose-road_to_reality} in the idempotent form, this can be
written even shorter as:
\[
    T_{(ab)} = 0
    \iff
    \frac1{2!} \sbr{T_{ab} + T_{ba}} = 0.
\]
The left side can be read as “the symmetric part of $\tens T$ is zero”. That
means that $\tens T$ is completely antisymmetric, which is what the other
equations have been saying all along.

Generators are generally without a trace, which is automatically fulfilled by
an antisymmetric tensor. Together with
\[
    \det(\exp(\tens A)) = \exp(\tr(\tens A))
\]
the vanishing trace also means that the determinant is unity, which is required
by a special Lorentz transformation.

So the requirement for the tensors $\tens m$ and $\tens t$ are that they are
antisymmetric when written with all lower indices. Since they also have to be
hermitian, they must be purely imaginary.

\paragraph{Lorentz transformation}

For SO(1, 3) we have $\eta = \diag(1, -1, -1, -1)$ and a $4\times 4$ matrix
$\tens m$. Since it is antisymmetric and purely imaginary, there are 6 degrees
of freedom, which corresponds to the three rotations and the three
pseudo rotations (boosts).

\paragraph{Rotation}

In SO(3) we have $p = 3$ and $q = 0$. Therefore the metric tensor is trivial
with $\eta = \tens 1_3$. The matrices have three degrees of freedom which
correspond to say the Euler angles.

\subsection{Generators for Lorentz group}

\paragraph{Corresponding generators}

The generators $\tens m_i$, where $i$ could also be a more complex index like
$\rho\sigma$, are the generators of the matrix representation of the Lorentz
group. Those $\tens m_i$ act on $\tens x$ to produce $\tens x'$. The $L_i$ that
are asked for are the generators of the functional representation of the same
group. They are connected by the Taylor expansion given in Equation~(11) on the
problem set.

Looking at the infinitesimal transformation of the $\tens x$ I have already
given that in the first part of the problem. I amended the formula with an
index $i$ to sum over multiple generators. The transformation then is given by
\[
    x'^\mu = x^\mu + \iup \epsilon^i m_i{}^\mu{}_\nu x^\nu.
\]
Here, the $\tens m_i$ act as the generators of the Lorentz transformations. Now
I take the derivative with respect to the parameter $\epsilon^i$:
\[
    \pd{x'^\mu}{\epsilon^i} = \iup m_i{}^\mu{}_\nu x^\nu.
\]
To get the whole generator I need to contract it with $\partial_\mu$:
\[
    L_i = \pd{x'^\mu}{\epsilon^i} \partial_\mu
    = \iup m_i{}^\mu{}_\nu x^\nu \partial_\mu.
\]
Those are the generators in the representation on functions.

\paragraph{Commutation relation of the $m$}

The first step is to work out the commutation relation of the $\tens m_i$. This
commutator has two free indices because $\tens m_i$ are tensors of valence (1,
1). I can write out the matrix multiplication from the commutator:
\begin{align*}
    \sbr{\tens m_{\mu\nu}, \tens m_{\rho\sigma}}^\alpha{}_\beta
    &= m_{\mu\nu}{}^\alpha{}_\gamma m_{\rho\sigma}{}^\gamma{}_\beta
    - m_{\rho\sigma}{}^\alpha{}_\gamma m_{\mu\nu}{}^\gamma{}_\beta
    \intertext{%
        This still fits one one line, so the solution is to expand the $m$ as
        well in terms of its definition. For that, I will use my idempotent
        version of the antisymmetrization bracket. This causes the factor of 4
        in front of everything.
    }
    &= 4 \iup
    \sbr{
        \deltaup^\alpha_{[\mu} \eta_{\nu]\gamma}
        \deltaup^\gamma_{[\rho} \eta_{\sigma]\beta}
        -
        \deltaup^\alpha_{[\rho} \eta_{\sigma]\gamma}
        \deltaup^\gamma_{[\mu} \eta_{\nu]\beta}
    }
    \intertext{%
        Now I can execute the middle Kronecker symbol.
    }
    &= 4 \iup
    \sbr{
        \deltaup^\alpha_{[\mu} \eta_{\nu][\rho} \eta_{\sigma]\beta}
        -
        \deltaup^\alpha_{[\rho} \eta_{\sigma][\mu} \eta_{\nu]\beta}
    }
    \intertext{%
        Here comes the point where I have to expand the antisymmetrization
        brackets again. I start with the second bracket.
    }
    &= 2 \iup
    \sbr{
        \deltaup^\alpha_{[\mu} \eta_{\nu]\rho} \eta_{\sigma\beta}
        - \deltaup^\alpha_{[\mu} \eta_{\nu]\sigma} \eta_{\rho\beta}
        - \deltaup^\alpha_{[\rho} \eta_{\sigma]\mu} \eta_{\nu\beta}
        + \deltaup^\alpha_{[\rho} \eta_{\sigma]\nu} \eta_{\mu\beta}
    }
    \intertext{%
        And then the first bracket as well.
    }
    &= \iup \,
    \Big[
          \deltaup^\alpha_{\mu} \eta_{\nu\rho} \eta_{\sigma\beta}
        - \deltaup^\alpha_{\nu} \eta_{\mu\rho} \eta_{\sigma\beta}
        %
        - \deltaup^\alpha_{\mu} \eta_{\nu\sigma} \eta_{\rho\beta}
        + \deltaup^\alpha_{\nu} \eta_{\mu\sigma} \eta_{\rho\beta}
        %
    \\&\qquad
        - \deltaup^\alpha_{\rho} \eta_{\sigma\mu} \eta_{\nu\beta}
        + \deltaup^\alpha_{\sigma} \eta_{\rho\mu} \eta_{\nu\beta}
        %
        + \deltaup^\alpha_{\rho} \eta_{\sigma\nu} \eta_{\mu\beta}
        - \deltaup^\alpha_{\sigma} \eta_{\rho\nu} \eta_{\mu\beta}
    \Big]
    \intertext{%
        Now I switch the two $\tens \eta$ in each term.
    }
    &= \iup \,
    \Big[
          \deltaup^\alpha_{\mu} \eta_{\sigma\beta} \eta_{\nu\rho}
        - \deltaup^\alpha_{\nu} \eta_{\sigma\beta} \eta_{\mu\rho}
        %
        - \deltaup^\alpha_{\mu} \eta_{\rho\beta} \eta_{\nu\sigma}
        + \deltaup^\alpha_{\nu} \eta_{\rho\beta} \eta_{\mu\sigma}
        %
    \\&\qquad
        - \deltaup^\alpha_{\rho} \eta_{\nu\beta} \eta_{\sigma\mu}
        + \deltaup^\alpha_{\sigma} \eta_{\nu\beta} \eta_{\rho\mu}
        %
        + \deltaup^\alpha_{\rho} \eta_{\mu\beta} \eta_{\sigma\nu}
        - \deltaup^\alpha_{\sigma} \eta_{\mu\beta} \eta_{\rho\nu}
    \Big]
    \intertext{%
        Now those terms need to be regrouped by the indices in the last
        $\tens\eta$. The metric tensor is symmetric, so the order of the
        indices there does not make a difference.
    }
    &= \iup \,
    \Big[
          \deltaup^\alpha_{\mu} \eta_{\sigma\beta} \eta_{\nu\rho}
        - \deltaup^\alpha_{\sigma} \eta_{\mu\beta} \eta_{\rho\nu}
        + \deltaup^\alpha_{\sigma} \eta_{\nu\beta} \eta_{\rho\mu}
        - \deltaup^\alpha_{\nu} \eta_{\sigma\beta} \eta_{\mu\rho}
    \\&\qquad
        + \deltaup^\alpha_{\nu} \eta_{\rho\beta} \eta_{\mu\sigma}
        - \deltaup^\alpha_{\rho} \eta_{\nu\beta} \eta_{\sigma\mu}
        + \deltaup^\alpha_{\rho} \eta_{\mu\beta} \eta_{\sigma\nu}
        - \deltaup^\alpha_{\mu} \eta_{\rho\beta} \eta_{\nu\sigma}
    \Big]
    \intertext{%
        Now I can use the antisymmetrization bracket again for the first terms.
    }
    &= 2 \iup
    \sbr{
        \deltaup^\alpha_{[\mu} \eta_{\sigma]\beta} \eta_{\nu\rho}
        + \deltaup^\alpha_{[\sigma} \eta_{\nu]\beta} \eta_{\rho\mu}
        + \deltaup^\alpha_{[\nu} \eta_{\rho]\beta} \eta_{\mu\sigma}
        + \deltaup^\alpha_{[\rho} \eta_{\mu]\beta} \eta_{\sigma\nu}
    }
    \intertext{%
        Next is the recognition of the $\tens m$. The first two factors always
        have the indices $\alpha$ and $\beta$. So I can pull those indices out.
    }
    &= \iup
    \sbr{
        \tens m_{[\mu\sigma]} \eta_{\nu\rho}
        + \tens m_{[\sigma\nu]} \eta_{\rho\mu}
        + \tens m_{[\nu\rho]} \eta_{\mu\sigma}
        + \tens m_{[\rho\mu]} \eta_{\sigma\nu}
    }^\alpha{}_\beta
    \intertext{%
        I move the metric tensor in front of the generator.
    }
    &= \iup
    \sbr{
          \eta_{\nu\rho}   \tens m_{[\mu\sigma]}
        + \eta_{\rho\mu}   \tens m_{[\sigma\nu]}
        + \eta_{\mu\sigma} \tens m_{[\nu\rho]}  
        + \eta_{\sigma\nu} \tens m_{[\rho\mu]}  
    }^\alpha{}_\beta
    \intertext{%
        In order to match the exact notation on the problem set, I will switch
        indices. The metric tensor is symmetric and the generator is
        antisymmetric. I copied the right hand side as a reminder as well.
    }
    \sbr{\tens m_{\mu\nu}, \tens m_{\rho\sigma}}^\alpha{}_\beta
    &= \iup
    \sbr{
          \eta_{\nu\rho}   \tens m_{[\mu\sigma]}
        - \eta_{\mu\rho}   \tens m_{[\nu\sigma]}
        - \eta_{\nu\sigma} \tens m_{[\mu\rho]}  
        + \eta_{\mu\sigma} \tens m_{[\nu\rho]}  
    }^\alpha{}_\beta
\end{align*}
This looks very similar to the desired commutator of the $L_i$.

\paragraph{Commutation relation of the $\tens L$}

Now that I have the commutation of the matrix generators, I can compute the
commutator of the generators on functions.
\begin{align*}
    [L_i, L_j]
    &= \sbr{
        \iup m_i{}^\mu{}_\nu x^\nu \partial_\mu,
        \iup m_j{}^\alpha{}_\beta x^\beta \partial_\alpha
    }
    \intertext{%
        As a first step, I remove the two imaginary units and change the order
        in the commutator.
    }
    &= \sbr{
        m_i{}^\alpha{}_\beta x^\beta \partial_\alpha,
        m_j{}^\mu{}_\nu x^\nu \partial_\mu
    }
    \intertext{%
        Then I can write it out in full.
    }
    &= m_i{}^\alpha{}_\beta x^\beta \partial_\alpha
    m_j{}^\mu{}_\nu x^\nu \partial_\mu
    -
    m_j{}^\mu{}_\nu x^\nu \partial_\mu
    m_i{}^\alpha{}_\beta x^\beta \partial_\alpha
   \intertext{%
       The $\tens m$ can be pulled in front of the $\tens x$ and $\tens
       \partial$ since those do not depend on each other.
   }
   &= m_i{}^\alpha{}_\beta
   m_j{}^\mu{}_\nu
   x^\beta \partial_\alpha
   x^\nu \partial_\mu
   -
   m_j{}^\mu{}_\nu
   m_i{}^\alpha{}_\beta
   x^\nu \partial_\mu
   x^\beta \partial_\alpha
   \intertext{%
       Next I can rename the Greek indices since they are all summed over.
   }
   &= m_i{}^\alpha{}_\beta
   m_j{}^\mu{}_\nu
   x^\beta \partial_\alpha
   x^\nu \partial_\mu
   -
   m_j{}^\alpha{}_\beta
   m_i{}^\mu{}_\nu
   x^\beta \partial_\alpha
   x^\nu \partial_\mu
   \intertext{%
       Then I can factor out the $\tens x$ and $\tens \partial$.
   }
   &=
   \sbr{
       m_i{}^\alpha{}_\beta
       m_j{}^\mu{}_\nu
       -
       m_j{}^\alpha{}_\beta
       m_i{}^\mu{}_\nu
   }
   x^\beta \partial_\alpha
   x^\nu \partial_\mu
   \intertext{%
       The partial derivatives act on a function $f(\tens x)$. One has to keep
       that in mind then computing the commutator. So
       \[
           \partial_\alpha
           x^\nu \partial_\mu
           f(\tens x)
           =
           [\partial_\alpha
           x^\nu] \partial_\mu
           f(\tens x)
           +
           x^\nu \partial_\alpha \partial_\mu
           f(\tens x)
           =
           \deltaup_\alpha^\nu \partial_\mu
           f(\tens x)
           +
           x^\nu \partial_\alpha \partial_\mu
           f(\tens x)
           = \sbr{
               \deltaup_\alpha^\nu
               f(\tens x)
               +
               x^\nu \partial_\alpha
           } \partial_\mu
           f(\tens x).
       \]
       I use this for the last three factors now.
   }
   &= \sbr{
       m_i{}^\alpha{}_\beta
       m_j{}^\mu{}_\nu
       -
       m_j{}^\alpha{}_\beta
       m_i{}^\mu{}_\nu
   }
   x^\beta
   \sbr{\deltaup_\alpha^\nu + x^\nu \partial_\alpha} \partial_\mu
   \intertext{%
       Then I expand the second bracket and apply the Kronecker symbol.
   }
   &=
   \sbr{
       m_i{}^\nu{}_\beta
       m_j{}^\mu{}_\nu
       -
       m_j{}^\nu{}_\beta
       m_i{}^\mu{}_\nu
   }
   x^\beta \partial_\mu
   +
   \sbr{
       m_i{}^\alpha{}_\beta
       m_j{}^\mu{}_\nu
       -
       m_j{}^\alpha{}_\beta
       m_i{}^\mu{}_\nu
   }
   x^\beta
   x^\nu \partial_\alpha \partial_\mu
   \intertext{%
       The second bracket is antisymmetric in $i$ and $j$ but symmetric in
       $(\alpha, \mu)$ and $(\beta\nu)$. Therefore, the second summand is just
       zero. Only the first summand remains.
   }
   &=
   \sbr{
       m_i{}^\nu{}_\beta
       m_j{}^\mu{}_\nu
       -
       m_j{}^\nu{}_\beta
       m_i{}^\mu{}_\nu
   }
   x^\beta \partial_\mu
   \intertext{%
       I switch both of the $\tens m$ in pairs such that the commutator
       gets removed.
   }
   &=
   \sbr{
       m_j{}^\mu{}_\nu
       m_i{}^\nu{}_\beta
       -
       m_i{}^\mu{}_\nu
       m_j{}^\nu{}_\beta
   }
   x^\beta \partial_\mu
   \intertext{%
       This is the commutator that I have derived previously. I just have to
        expand the indices a bit. I set $i = \mu\nu$ and $j = \rho\sigma$ and
        rename the other indices such that there is no clash in them.
   }
   &=
   \sbr{
       m_{\rho\sigma}{}^\alpha{}_\gamma
       m_{\mu\nu}{}^\gamma{}_\beta
       -
       m_{\mu\nu}{}^\alpha{}_\gamma
       m_{\rho\sigma}{}^\gamma{}_\beta
   }
   x^\beta \partial_\alpha
   \intertext{%
       Now I can write this as the commutator.
   }
   &=
   \sbr{\tens m_{\rho\sigma}{}, \tens m_{\mu\nu}{}}^\alpha{}_\beta
   x^\beta \partial_\alpha
   \intertext{%
       The commutator was computed before, I just insert the result now. I just
       need to add the minus sign again since the order in the commutator is
       not the same here.
   }
   &= - \iup
   \sbr{
       \eta_{\nu\rho}   \tens m_{[\mu\sigma]}
       - \eta_{\mu\rho}   \tens m_{[\nu\sigma]}
       - \eta_{\nu\sigma} \tens m_{[\mu\rho]}  
       + \eta_{\mu\sigma} \tens m_{[\nu\rho]}  
   }^\alpha{}_\beta
   x^\beta \partial_\alpha
   \intertext{%
       Now I can take the contraction with the vector and the partial
       derivative into every single term and obtain the $L_{\mu\nu}$ again.
   }
   &= -
   \sbr{
         \eta_{\nu\rho}   L_{[\mu\sigma]}
       - \eta_{\mu\rho}   L_{[\nu\sigma]}
       - \eta_{\nu\sigma} L_{[\mu\rho]}  
       + \eta_{\mu\sigma} L_{[\nu\rho]}  
   }
\end{align*}
I am now missing an imaginary unit, though.

\subsection{Translations}

\paragraph{Translation generators}

Using Equation~(11), this is a direct application. The transformation is
\[
    x'^\mu = x^\mu + a^\mu
\]
where $\{a^\mu\}$ are the four parameters of this transformation. The
differential operators for the transformation are then simply:
\[
    P_\lambda
    = \iup \pd{x'^\mu}{a^\lambda} \partial_\mu
    = \iup \deltaup^\mu_\lambda \partial_\mu
    = \iup \partial_\lambda.
\]

\paragraph{Momentum commutators}

The commutators $[P_\mu, P_\nu]$ are zero because the partial derivatives of
differentiable functions commute.

\paragraph{Lorentz and momentum commutator}

\begin{align*}
    [L_{\mu\nu}, P_\lambda]
    &= \sbr{
    - m_{\mu\nu}{}^\alpha{}_\beta x^\beta \partial_\alpha,
    \iup \partial_\lambda
    }
    \intertext{%
        The generators $\tens m_i$ do not depend on $\tens x$ an can therefore
        be moved out of the commutator together with all constants.
    }
    &= - \iup m_{\mu\nu}{}^\alpha{}_\beta \sbr{
        x^\beta \partial_\alpha, \partial_\lambda
    }
    \intertext{%
        The partial derivatives commute anyway, so that can be moved out of the
        commutator as well.
    }
    &= - \iup m_{\mu\nu}{}^\alpha{}_\beta \sbr{
        x^\beta, \partial_\lambda
    } \partial_\alpha
    \intertext{%
        That commutator is well known by now, it is $\iup
        \deltaup^\beta_\lambda$.
    }
    &= m_{\mu\nu}{}^\alpha{}_\beta \deltaup^\beta_\lambda \partial_\alpha
    \intertext{%
        Then the indices can be contracted easily.
    }
    &= m_{\mu\nu}{}^\alpha{}_\lambda \partial_\alpha
\end{align*}
Now that looks like
\[
    \pd{L_{\mu\nu}}{x^\lambda}
\]
or $- \iup m_{\mu\nu}{}^\alpha{}_\lambda P_\alpha$. I am not sure which makes
more sense here.

\paragraph{Closing of algebra}

% TODO
\emph{(Missing)}

\subsection{Commutators with Pauli-Lubanski vector}

\paragraph{First one}

\begin{align*}
    [W^\lambda, L^{\mu\nu}]
    &= \frac12 \epsilon^{\lambda\sigma\alpha\beta} [L_{\alpha\beta}
    P_\sigma, L^{\mu\nu}]
    \intertext{%
        I now use the commutator identity $[AB, C] = A[B, C] + [A, C]B$.
    }
    &= \frac12 \epsilon^{\lambda\sigma\alpha\beta}
    \sbr{
        L_{\alpha\beta} [P_\sigma, L^{\mu\nu}]
        +
        [L_{\alpha\beta}, L^{\mu\nu}] P_\sigma
    }
\end{align*}

% TODO

\paragraph{Second one}

I first expand the Pauli-Lubanski vector by its definition.
\begin{align*}
    [W^\lambda, P^\sigma]
    &= \frac12 \epsilon^{\lambda\rho\mu\nu} [L_{\mu\nu} P_\rho, P_\sigma]
    \intertext{%
        The components of $\tens P$ commute with each other, as shown before.
        Therefore I can move that out to the back of the expression.
    }
    &= \frac12 \epsilon^{\lambda\rho\mu\nu} [L_{\mu\nu}, P_\sigma] P_\rho
    \intertext{%
        Now I can recognize the commutator from the previous problem.
    }
    &= - \iup \frac12 \epsilon^{\lambda\rho\mu\nu} m_{\mu\nu}{}^\alpha{}_\sigma P_\rho
    \intertext{%
        Then I expand the $\tens m$.
    }
    &= - \iup \epsilon^{\lambda\rho\mu\nu} \deltaup^\alpha_{[\mu}
    \eta_{\nu]\sigma} P_\alpha P_\rho
    \intertext{%
        The Levi-Civita symbol antisymmetrizes in the indices $\mu$ and $\nu$
        already, so I can drop the antisymmetrization bracket.
    }
    &= - \iup \epsilon^{\lambda\rho\mu\nu} \deltaup^\alpha_{\mu}
    \eta_{\nu\sigma} P_\alpha P_\rho
    \intertext{%
        I execute the Kronecker symbol.
    }
    &= - \iup \epsilon^{\lambda\rho\mu\nu}
    \eta_{\nu\sigma} P_{\mu} P_\rho
    \intertext{%
        The expression is antisymmetrized by the Levi-Civita symbol in $\mu$
        and $\rho$, but the $\tens P$ at the end are symmetric in those
        indices. The complete expression is therefore zero.
    }
    &= 0
\end{align*}

\paragraph{Third one}

\begin{align*}
    [W^\lambda, W^\sigma]
    &= [\epsilon^{\lambda\rho\mu\nu} L_{\mu\nu} P_\rho, W^\sigma]
    \intertext{%
        I pull out the scalar.
    }
    &= \epsilon^{\lambda\rho\mu\nu} [L_{\mu\nu} P_\rho, W^\sigma]
    \intertext{%
        As shown just before this commutator, the $\tens P$ commutes with the
        $\tens W$. This can be moved to the end then.
    }
    &= \epsilon^{\lambda\rho\mu\nu} [L_{\mu\nu}, W^\sigma] P_\rho
    \intertext{%
        Now I can apply the first commutation relation.
    }
    &= 2 \iup \epsilon^{\lambda\rho\mu\nu} W^{[\mu} \eta^{\nu] \sigma} P_\rho
    \intertext{%
        The Levi-Civita symbol already antisymmetrizes $\mu$ and $\nu$, so I
        drop the extra bracket.
    }
    &= \iup \epsilon^{\lambda\rho\mu\nu} W^{\mu} \eta^{\nu \sigma} P_\rho
    \intertext{%
        Now I can contract over $\nu$.
    }
    &= \iup \epsilon^{\lambda\rho\mu\sigma} W^{\mu} P_\rho
\end{align*}
The dummy indices have to be renamed, but this is the same expression as given
on the problem set.

\subsection{Casimir operators}

In order to show that $P^2$ and $W^2$ are Casimir operators, I show that the
commutator with every operator is zero.

\paragraph{$P^2$ with $P$}

Since the components of $P$ commute with each other, this is trivially the
case:
\[
    [P^\mu P_\mu, P_\alpha] = 0.
\]

\paragraph{$P^2$ with $W$}

The components of $P$ commute with the components of $W$, so this equally
trivially is
\[
    [P^\mu P_\mu, W^\alpha] = 0.
\]

\paragraph{$P^2$ with $L$}

\begin{align*}
    [P^\mu P_\mu, L_{\lambda\rho}]
    &= P^\mu [P_\mu, L_{\lambda\rho}] + [P^\mu, L_{\lambda\rho}] P_\mu
    \intertext{%
        I move indices up and down within the contraction.
    }
    &= P^\mu [P_\mu, L_{\lambda\rho}] + [P_\mu, L_{\lambda\rho}] P^\mu
    \intertext{%
        Now I expand the $L$. The $m$ commute with the $P$.
    }
    &= \iup m_{\lambda\rho}{}^\alpha{}_\mu [P^\mu P_\alpha + P_\alpha P^\mu]
    \intertext{%
        The $P$ commute with each other.
    }
    &= 2 \iup m_{\lambda\rho}{}^\alpha{}_\mu P^\mu P_\alpha
    \intertext{%
        Then I expand the $m$.
    }
    &= 4 \iup \deltaup^\alpha_{[\lambda} \eta_{\rho]\mu} P^\mu P_\alpha
    \intertext{%
        Now I execute the Kronecker symbol. I also use the metric tensor to
        lower the index on the first $P$.
    }
    &= 4 \iup P_{[\lambda} P_{\rho]}
    \intertext{%
        It can be seen clearly that the antisymmetric part of this symmetric
        tensor product $\tens P \otimes \tens P$.
    }
    &= 0
\end{align*}

\paragraph{$W^2$ with $P$}

The components of $W$ of $P$ commute with each other, so we have
\[
    [W^\mu W_\mu, P_\lambda] = 0.
\]

\paragraph{$W^2$ with $W$}

\begin{align*}
    [W^\mu W_\mu, W^\lambda]
    &= W_\mu [W^\mu, W^\lambda] + [W^\mu, W^\lambda] W_\mu
    \intertext{%
        Then I expand the $W^\mu$ inside the commutator.
    }
    &= \iup \epsilon^{\mu\lambda\alpha\beta}
    [W_\mu W_\alpha P_\beta + W_\alpha P_\beta W_\mu]
    \intertext{%
        This is symmetric in $\alpha$ and $\beta$ since the $W$ and $P$
        commute. It is antisymmetric in those indices because of the
        Levi-Civita symbol. Therefore, the whole expression is zero.
    }
    &= 0
\end{align*}

\paragraph{$W^2$ with $L$}

\begin{align*}
    [W^\mu W_\mu, L_{\lambda\sigma}]
    &= W^\mu [W_\mu, L_{\lambda\sigma}] + [W_\mu, L_{\lambda\sigma}] W^\mu
    \intertext{%
        Now I can insert the already computed commutator of a single $W$ with
        $L$.
    }
    &= 2 \iup
    \sbr{
        W^\mu W^{[\sigma} \eta^{\lambda]\mu}
        + W^{[\sigma} \eta^{\lambda]\mu} W^\mu
    }
    \intertext{%
        Then I contract over $\mu$.
    }
    &= 2 \iup \sbr{ W^{[\lambda} W^{\sigma]} + W^{[\sigma} W^{\lambda]} }
    \intertext{%
        I switch the indices in the second summand.
    }
    &= 2 \iup \sbr{ W^{[\lambda} W^{\sigma]} - W^{[\lambda} W^{\sigma]} } \\
    &= 0
\end{align*}

\end{document}

% vim: spell spelllang=en tw=79
