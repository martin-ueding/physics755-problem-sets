\documentclass[11pt, english, fleqn, DIV=15, headinclude, BCOR=1cm]{scrartcl}

\usepackage[bibatend, color]{../header}
\usepackage{../my-boxes}

\usepackage{booktabs}

\hypersetup{
    pdftitle=
}

\newcounter{totalpoints}
\newcommand\punkte[1]{#1\addtocounter{totalpoints}{#1}}

\newcounter{problemset}
\setcounter{problemset}{2}

\subject{physics755 -- Quantum Field Theory}
\ihead{physics755 -- Problem Set \arabic{problemset}}

\title{Problem Set \arabic{problemset}}

\publishers{Group Friday -- Robert Lauktien}
\ofoot{Group Friday -- Robert Lauktien}

\author{
    Martin Ueding \\ \small{\href{mailto:mu@martin-ueding.de}{mu@martin-ueding.de}}
}
\ifoot{Martin Ueding}

\ohead{\rightmark}

\begin{document}

\maketitle

\vspace{3ex}

\begin{center}
    \begin{tabular}{rrr}
        problem & achieved points & possible points \\
        \midrule
        \nameref{homework:1} & & \punkte{10} \\
        \midrule
        total & & \arabic{totalpoints}
    \end{tabular}
\end{center}

\section{Poincaré algebra}
\label{homework:1}

\subsection{Generators of SO(1, 3) and SO(3)}

The $m^\mu{}_\nu$ and $t^i{}_j$ are the generators in the Physicist's
convention of hermitian generators since the Taylor expansion contains an
imaginary unit.

Orthogonal transformations from SO(1, 3), which are Lorentz transformations or
from SO(3), which are rotations in three dimensions, have to leave the metric
tensor $\eta$ invariant. I will solve this for the general case of
$\mathrm{SO}(p, q)$ and relate that to both special cases. The metric of
interest in those spaces has signature $(p, q)$, that means $p$ positive and
$q$ negative eigenvalues. Since we are only concerned with special relativity
here, the metric tensor always has the special form
\[
    \tens \eta = \diag(\underbrace{1, \ldots, 1}_\text{$p$ times},
    \underbrace{-1, \ldots, -1}_\text{$q$ times}).
\]

Let $\tens A$ be the transformation matrix. Then the invariance of the metric
is written in index notation as
\begin{align*}
    \eta_{ab}
    &= \eta_{ij} [A\inv]^i{}_a [A\inv]^j{}_b.
    \intertext{%
        The transformation can now be written in terms of a generator
        expression with the generators $\tens T_k$ of the group.
    }
    &= \eta_{ij} \exp\del{-\iup \epsilon^k \tens T_k}^i{}_a
    \exp\del{-\iup \epsilon^k \tens T_k}^j{}_b
    \intertext{%
        In the first order of $\vec \epsilon$, this can be linearized to give a
        form like the one given in the problem statement. For those orthogonal
        groups, there are multiple generators. Here it suffices to look at one
        of those generators since they all have the same properties. Linear
        combinations are still generators, the generators form a vector space.
        This vector space is the tangent space of the group manifold around the
        identity element. The problem statement basically has $\epsilon \tens m
        = \epsilon^k \tens T_k$. I will now just look at one of the generators
        $\tens T$ without an index.
    }
    &= \eta_{ij} \sbr{\deltaup^i_a - \iup \epsilon T^i{}_a}
    \sbr{\deltaup^j_b - \iup \epsilon T^j{}_b} + \mathrm O(\epsilon^2)
    \intertext{%
        Then I can expand those brackets. I will drop the fourth summand since
        it is of second order already.
    }
    &= \eta_{ij} \deltaup^i_a \deltaup^j_b - \iup \epsilon \eta_{ij} 
    \sbr{T^i{}_a \deltaup^j_b + \deltaup^i_a T^j{}_b}
    + \mathrm O(\epsilon^2)
    \intertext{%
        I will clean up the indices.
    }
    &= \eta_{ab} - \iup \epsilon \sbr{\eta_{ib} T^i{}_a + \eta_{aj} T^j{}_b}
\end{align*}
This equation now leads to a shorter one:
\begin{align*}
    \eta_{ib} T^i{}_a + \eta_{aj} T^j{}_b &= 0.
    \intertext{%
        I can even pull down the index and write this as:
    }
    T_{ba} + T_{ab} &= 0.
\end{align*}

Using the (anti)symmetrization notation by
\textcite{penrose-road_to_reality} in the idenpotent form, this can be
written even shorter as:
\[
    T_{(ab)} = 0
    \iff
    \frac1{2!} \sbr{T_{ab} + T_{ba}} = 0.
\]
The left side can be read as “the symmetric part of $\tens T$ is zero”. That
means that $\tens T$ is completely antisymmetric, which is what the other
equations have been saying all along.

Generators are generally without a trace, which is automatically fulfilled by
an antisymmetric tensor. Together with
\[
    \det(\exp(\tens A)) = \exp(\tr(\tens A))
\]
the vanishing trace also means that the determinant is unity, which is required
by a special Lorentz transformation.

So the requirement for the tensors $\tens m$ and $\tens t$ are that they are
antisymmetric when written with all lower indices. Since they also have to be
hermitian, they must be purely imaginary.

\paragraph{Lorentz transformation}

For SO(1, 3) we have $\eta = \diag(1, -1, -1, -1)$ and a $4\times 4$ matrix
$\tens m$. Since it is antisymmetric and purely imaginary, there are 6 degrees
of freedom, which corresponds to the three rotations and the three
pseudorotations (boosts).

\paragraph{Rotation}

In SO(3) we have $p = 3$ and $q = 0$. Therefore the metric tensor is trivial
with $\eta = \tens 1_3$. The matrices have three degrees of freedom which
correspond to say the Euler angles.

\section{Generators for Lorentz group}

\paragraph{Corresponding generators}

I am a little puzzled by this problem. It introduces the $\tens m$ as well as
the $\tens L$ which seem to be the exact same things. So to me it seems $\tens
m_{[\mu\nu]} = \tens L_{[\mu\nu]}$, or with indices it is $L_{[\mu\nu]}{}^i{}_j
= m_{[\mu\nu]}{}^i{}_j$.

\paragraph{Commutation relation}

The first step is to write the $\tens L$ as $\tens m$, nothing fancy if the
above identification is correct.
\begin{align*}
    \sbr{\tens L_{\mu\nu}, \tens L_{\rho\sigma}}^\alpha{}_\beta
    &= \sbr{\tens m_{\mu\nu}, \tens m_{\rho\sigma}}^\alpha{}_\beta
    \intertext{%
        Then I can write out the matrix multiplication from the commutator.
    }
    &= m_{\mu\nu}{}^\alpha{}_\gamma m_{\rho\sigma}{}^\gamma{}_\beta
    - m_{\rho\sigma}{}^\alpha{}_\gamma m_{\mu\nu}{}^\gamma{}_\beta
    \intertext{%
        This still fits one one line, so the solution is to expand the $m$ as
        well in terms of its definition. For that, I will use my idempotent
        version of the antisymmetrization bracket. This causes the factor of 4
        in front of everything.
    }
    &= 4 \iup
    \sbr{
        \deltaup^\alpha_{[\mu} \eta_{\nu]\gamma}
        \deltaup^\gamma_{[\rho} \eta_{\sigma]\beta}
        -
        \deltaup^\alpha_{[\rho} \eta_{\sigma]\gamma}
        \deltaup^\gamma_{[\mu} \eta_{\nu]\beta}
    }
    \intertext{%
        Now I can execute the middle Kronecker symbol.
    }
    &= 4 \iup
    \sbr{
        \deltaup^\alpha_{[\mu} \eta_{\nu][\rho} \eta_{\sigma]\beta}
        -
        \deltaup^\alpha_{[\rho} \eta_{\sigma][\mu} \eta_{\nu]\beta}
    }
    \intertext{%
        Here comes the point where I have to expand the antisymmetrization
        brackets again. I start with the second bracket.
    }
    &= 2 \iup
    \sbr{
        \deltaup^\alpha_{[\mu} \eta_{\nu]\rho} \eta_{\sigma\beta}
        - \deltaup^\alpha_{[\mu} \eta_{\nu]\sigma} \eta_{\rho\beta}
        - \deltaup^\alpha_{[\rho} \eta_{\sigma]\mu} \eta_{\nu\beta}
        + \deltaup^\alpha_{[\rho} \eta_{\sigma]\nu} \eta_{\mu\beta}
    }
    \intertext{%
        And then the first bracket as well.
    }
    &= \iup \,
    \Big[
          \deltaup^\alpha_{\mu} \eta_{\nu\rho} \eta_{\sigma\beta}
        - \deltaup^\alpha_{\nu} \eta_{\mu\rho} \eta_{\sigma\beta}
        %
        - \deltaup^\alpha_{\mu} \eta_{\nu\sigma} \eta_{\rho\beta}
        + \deltaup^\alpha_{\nu} \eta_{\mu\sigma} \eta_{\rho\beta}
        %
    \\&\qquad
        - \deltaup^\alpha_{\rho} \eta_{\sigma\mu} \eta_{\nu\beta}
        + \deltaup^\alpha_{\sigma} \eta_{\rho\mu} \eta_{\nu\beta}
        %
        + \deltaup^\alpha_{\rho} \eta_{\sigma\nu} \eta_{\mu\beta}
        - \deltaup^\alpha_{\sigma} \eta_{\rho\nu} \eta_{\mu\beta}
    \Big]
    \intertext{%
        Now I switch the two $\tens \eta$ in each term.
    }
    &= \iup \,
    \Big[
          \deltaup^\alpha_{\mu} \eta_{\sigma\beta} \eta_{\nu\rho}
        - \deltaup^\alpha_{\nu} \eta_{\sigma\beta} \eta_{\mu\rho}
        %
        - \deltaup^\alpha_{\mu} \eta_{\rho\beta} \eta_{\nu\sigma}
        + \deltaup^\alpha_{\nu} \eta_{\rho\beta} \eta_{\mu\sigma}
        %
    \\&\qquad
        - \deltaup^\alpha_{\rho} \eta_{\nu\beta} \eta_{\sigma\mu}
        + \deltaup^\alpha_{\sigma} \eta_{\nu\beta} \eta_{\rho\mu}
        %
        + \deltaup^\alpha_{\rho} \eta_{\mu\beta} \eta_{\sigma\nu}
        - \deltaup^\alpha_{\sigma} \eta_{\mu\beta} \eta_{\rho\nu}
    \Big]
    \intertext{%
        Now those terms need to be regrouped by the indices in the last
        $\tens\eta$. The metric tensor is symmetric, so the order of the
        indices there does not make a difference.
    }
    &= \iup \,
    \Big[
          \deltaup^\alpha_{\mu} \eta_{\sigma\beta} \eta_{\nu\rho}
        - \deltaup^\alpha_{\sigma} \eta_{\mu\beta} \eta_{\rho\nu}
        + \deltaup^\alpha_{\sigma} \eta_{\nu\beta} \eta_{\rho\mu}
        - \deltaup^\alpha_{\nu} \eta_{\sigma\beta} \eta_{\mu\rho}
    \\&\qquad
        + \deltaup^\alpha_{\nu} \eta_{\rho\beta} \eta_{\mu\sigma}
        - \deltaup^\alpha_{\rho} \eta_{\nu\beta} \eta_{\sigma\mu}
        + \deltaup^\alpha_{\rho} \eta_{\mu\beta} \eta_{\sigma\nu}
        - \deltaup^\alpha_{\mu} \eta_{\rho\beta} \eta_{\nu\sigma}
    \Big]
    \intertext{%
        Now I can use the antisymmetrization bracket again for the first terms.
    }
    &= 2 \iup
    \sbr{
        \deltaup^\alpha_{[\mu} \eta_{\sigma]\beta} \eta_{\nu\rho}
        + \deltaup^\alpha_{[\sigma} \eta_{\nu]\beta} \eta_{\rho\mu}
        + \deltaup^\alpha_{[\nu} \eta_{\rho]\beta} \eta_{\mu\sigma}
        + \deltaup^\alpha_{[\rho} \eta_{\mu]\beta} \eta_{\sigma\nu}
    }
    \intertext{%
        Next is the recognition of the $\tens m$ and from there the $\tens L$.
        The first two factors always have the indices $\alpha$ and $\beta$. So
        I can pull those indices out.
    }
    &= \iup
    \sbr{
        \tens m_{[\mu\sigma]} \eta_{\nu\rho}
        + \tens m_{[\sigma\nu]} \eta_{\rho\mu}
        + \tens m_{[\nu\rho]} \eta_{\mu\sigma}
        + \tens m_{[\rho\mu]} \eta_{\sigma\nu}
    }^\alpha{}_\beta
    \intertext{%
        And since the $L$ and $m$ are the same, I can replace those as well. At
        the same step, I move the metric tensor in front of the generator.
    }
    &= \iup
    \sbr{
          \eta_{\nu\rho}   \tens L_{[\mu\sigma]}
        + \eta_{\rho\mu}   \tens L_{[\sigma\nu]}
        + \eta_{\mu\sigma} \tens L_{[\nu\rho]}  
        + \eta_{\sigma\nu} \tens L_{[\rho\mu]}  
    }^\alpha{}_\beta
    \intertext{%
        In order to match the exact notation on the problem set, I will switch
        indices. The metric tensor is symmetric and the generator is
        antisymmetric.
    }
    &= \iup
    \sbr{
          \eta_{\nu\rho}   \tens L_{[\mu\sigma]}
        - \eta_{\mu\rho}   \tens L_{[\nu\sigma]}
        - \eta_{\nu\sigma} \tens L_{[\mu\rho]}  
        + \eta_{\mu\sigma} \tens L_{[\nu\rho]}  
    }^\alpha{}_\beta
\end{align*}
If one removes the indices $\alpha$ and $\beta$ on both sides, this is exactly
Equation~(13) on the problem set.

\subsection{Translations}

\paragraph{Translation generators}

Using Equation~(11), this is a direct application. The transformation is
\[
    x'^\mu = x^\mu + a^\mu
\]
where $\{a^\mu\}$ are the four parameters of this transformation. The
differential operators for the transformation are then simply:
\[
    P_\lambda
    = \iup \pd{x'^\mu}{a^\lambda} \partial_\mu
    = \iup \deltaup^\mu_\lambda \partial_\mu
    = \iup \partial_\lambda.
\]

\paragraph{Momentum commutators}

The commutators $[P_\mu, P_\nu]$ are zero because the partial derivatives of
differentiable functions commute.

\paragraph{Lorentz and momentum commutator}

\begin{align*}
    [L_{\mu\nu}, P_\lambda]
    &= \sbr{
    - m_{\mu\nu}{}^\alpha{}_\beta x^\beta \partial_\alpha, \iup
    \partial_\lambda
    }
\end{align*}

\end{document}

% vim: spell spelllang=en tw=79
