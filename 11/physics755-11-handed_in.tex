\documentclass[11pt, english, fleqn, DIV=15, headinclude, BCOR=1cm]{scrartcl}

\usepackage[bibatend]{../header}
\usepackage{../my-boxes}

\usepackage{booktabs}
\usepackage{slashed}
\usepackage{simplewick}

\usepackage{feynmp-auto}
\usepackage{adjustbox}
\newenvironment{fmwrapper}{\begin{adjustbox}{margin=5mm}}{\end{adjustbox}}

\usepackage{multicol}
\usepackage{lastpage}

\usepackage{tikz}
\usepackage{pgfplots}
\pgfplotsset{compat=1.10}

\newcommand\timeorder{\mathscr T}
\newcommand\normorder{\mathscr N}

\hypersetup{
    pdftitle=
}

\newcounter{totalpoints}
\newcommand\punkte[1]{#1\addtocounter{totalpoints}{#1}}

\newcounter{problemset}
\setcounter{problemset}{11}

\subject{physics755 -- Quantum Field Theory}
\ihead{physics755 -- Problem Set \arabic{problemset}}

\title{Problem Set \arabic{problemset}}

\newcommand\thegroup{Group Tuesday -- Ripunjay Acharya}

\publishers{\thegroup}
\ofoot{\thegroup}

\author{
    Martin Ueding \\ \small{\href{mailto:mu@martin-ueding.de}{mu@martin-ueding.de}}
    \and
    Oleg Hamm
}
\ifoot{Martin Ueding, Oleg Hamm}

\ohead{\rightmark}

\begin{document}

\maketitle

\vspace{3ex}

\begin{center}
    \begin{tabular}{rrr}
        problem & achieved points & possible points \\
        \midrule
        \nameref{homework:1} & & \punkte{8} \\
        \nameref{homework:2} & & \punkte{7} \\
        \midrule
        total & & \arabic{totalpoints}
    \end{tabular}
\end{center}

\vspace{3ex}

\begin{center}
    \begin{small}
        This document consists of \pageref{LastPage} pages.
    \end{small}
\end{center}

\section{Muon pair production}
\label{homework:1}

\subsection{Invariant amplitude}

The only contributing diagram is the following (time to the left):

\begin{fmffile}{feyn-scatter-1}
    \begin{fmfgraph}(100, 50)
        \fmfleft{ei,pi}
        \fmfright{eo,po}
        \fmfdot{v1,v2}

        \fmf{fermion}{ei,v1}
        \fmf{fermion}{v1,pi}

        \fmf{fermion}{v2,eo}
        \fmf{fermion}{po,v2}

        \fmf{photon}{v1,v2}
    \end{fmfgraph}
\end{fmffile}

From this diagram we read off:
\[
    \iup \mathscr M = [- \iup e]^2
    \bar u^{r'}(\tens k_1) \, \mat\gamma^\mu v^r(\tens k_2)
    \frac{1}{[\tens p_1 + \tens p_2]^2}
    \bar v^{s'}(\tens p_1) \, \mat\gamma_\mu u^s(\tens p_2)
\]
The denominator in the photon propagator can be replaced by the Mandelstam
variable $s$. We go on to compute the appropriately averaged sum over squared
amplitudes.
\begin{align*}
    &\frac 14 \sum_{s,s',r,r'} |\iup \mathscr M|^2 \\
    &= \frac 14  \frac{e^4}{s^2} \sum_{s,s',r,r'}
    \sbr{\bar u^{r'}(\tens k_1) \, \mat\gamma^\mu v^r(\tens k_2)
    \bar v^{s'}(\tens p_1) \, \mat\gamma_\mu u^s(\tens p_2)}^*
    \bar u^{r'}(\tens k_1) \, \mat\gamma^\nu v^r(\tens k_2)
    \bar v^{s'}(\tens p_1) \, \mat\gamma_\nu u^s(\tens p_2)
    \intertext{%
        We use the first batch of relations from the previous exercise.
    }
    &= \frac 14  \frac{e^4}{s^2} \sum_{s,s',r,r'}
    \bar v^r(\tens k_2) \, \mat\gamma^\mu u^{r'}(\tens k_1)
    \bar u^s(\tens p_2) \, \mat\gamma_\mu v^{s'}(\tens p_1)
    \bar u^{r'}(\tens k_1) \, \mat\gamma^\nu v^r(\tens k_2)
    \bar v^{s'}(\tens p_1) \, \mat\gamma_\nu u^s(\tens p_2)
    \intertext{%
        We reorder the bilinears such that we can use the second batch of
        relations.
    }
    &= \frac 14  \frac{e^4}{s^2} \sum_{s,s',r,r'}
    \bar v^r(\tens k_2) \, \mat\gamma^\mu u^{r'}(\tens k_1)
    \bar u^{r'}(\tens k_1) \, \mat\gamma^\nu v^r(\tens k_2)
    \bar v^{s'}(\tens p_1) \, \mat\gamma_\nu u^s(\tens p_2)
    \bar u^s(\tens p_2) \, \mat\gamma_\mu v^{s'}(\tens p_1)
    \intertext{%
        Now we actually use the relations where $m$ is the electron mass and
        $M$ is the muon mass.
    }
    &= \frac 14  \frac{e^4}{s^2}
    \tr\del{[\slashed{\tens k}_1 + M] \, \mat\gamma^\nu [\slashed{\tens k}_2 +
    M] \, \mat\gamma^\mu}
    \tr\del{[\slashed{\tens p}_1 + m] \, \mat\gamma_\nu [\slashed{\tens p}_2 +
    m] \, \mat\gamma_\mu}
    \intertext{%
        We neglect the mass of the electron. Since traces of an odd number of
        Dirac matrices vanish, there will not be any terms in first order of
        $m$. There is are only terms in second order of $m$, which is very
        small compared to $M^2$.
    }
    &= \frac 14  \frac{e^4}{s^2}
    \tr\del{[\slashed{\tens k}_1 + M] \, \mat\gamma^\nu [\slashed{\tens k}_2 +
    M] \, \mat\gamma^\mu}
    \tr\del{\slashed{\tens p}_1 \mat\gamma_\nu \slashed{\tens p}_2 \mat\gamma_\mu}
    \intertext{%
        We start with the last trace. We can extract the components of the
        momenta out of the trace.
    }
    &= \frac 14  \frac{e^4}{s^2}
    \tr\del{[\slashed{\tens k}_1 + M] \, \mat\gamma^\nu [\slashed{\tens k}_2 +
    M] \, \mat\gamma^\mu}
    \tens p_1^\alpha \tens p_2^\beta \tr\del{\mat\gamma_\nu \mat\gamma^\alpha
    \mat\gamma_\mu \mat\gamma^\beta}
    \intertext{%
        The trace of four Dirac matrices was computed on the last sheet. We
        just use this here.
    }
    &= \frac{e^4}{s^2}
    \tr\del{[\slashed{\tens k}_1 + M] \, \mat\gamma^\nu [\slashed{\tens k}_2 +
    M] \, \mat\gamma^\mu}
    p_1^\alpha p_2^\beta \sbr{\deltaup_\nu^\alpha
        \deltaup_\mu^\beta - \eta_{\nu\mu} \eta^{\alpha\beta} +
    \deltaup^\beta_\nu \deltaup^\alpha_\mu} \\
    &= \frac{e^4}{s^2}
    \tr\del{[\slashed{\tens k}_1 + M] \, \mat\gamma^\nu [\slashed{\tens k}_2 +
    M] \, \mat\gamma^\mu}
    \sbr{p_{1\mu} p_{2\nu} + p_{1\nu} p_{2\mu} - p_1^\alpha p_{2\alpha}
    \eta_{\mu\nu}}
    \intertext{%
        Now we tend to the first trace. When we factor out the terms inside, we
        will get an order $M^0$ and an order $M^2$ term. The $M^1$ term will
        not contribute due to an odd number of Dirac matrices.
    }
    &= \frac{e^4}{s^2}
    \sbr{
        \tr\del{\slashed{\tens k}_1 \mat\gamma^\nu \slashed{\tens k}_2 \mat\gamma^\mu}
        + M^2 \tr\del{\mat\gamma^\nu\mat\gamma^\mu}
    }
    \sbr{p_{1\mu} p_{2\nu} + p_{1\nu} p_{2\mu} - p_1^\alpha p_{2\alpha}
    \eta_{\mu\nu}}
    \intertext{%
        The trace has the same structure we had with $\tens p_1$ and $\tens
        p_2$ before. The second trace is just a multiple of the metric tensor.
    }
    &= \frac{e^4}{s^2}
    \sbr{
        \sbr{k_1^\mu k_2^\nu + k_1^\nu k_2^\mu - k_1^\alpha k_{2\alpha}
        \eta^{\mu\nu}}
        + M^2 \eta^{\mu\nu}
    }
    \sbr{p_{1\mu} p_{2\nu} + p_{1\nu} p_{2\mu} - p_1^\alpha p_{2\alpha}
    \eta_{\mu\nu}}
    \intertext{%
        Then we can expand the left bracket and perform the contractions over
        $\mu$ and $\nu$ in the second term which we then get.
    }
    &= \frac{e^4}{s^2}
    \sbr{k_1^\mu k_2^\nu + k_1^\nu k_2^\mu - k_1^\alpha k_{2\alpha}
    \eta^{\mu\nu}}
    \sbr{p_{1\mu} p_{2\nu} + p_{1\nu} p_{2\mu} - p_1^\alpha p_{2\alpha}
    \eta_{\mu\nu}}
    +
    \frac{M^2 e^4}{s^2}
    \sbr{2 \tens p_1 \cdot \tens p_2 - 4 \tens p_1 \cdot \tens p_2}
    \intertext{%
        Now we look at the three terms that we will get when we expand the
        first bracket.
    }
    &= \frac{e^4}{s^2} \Bigg[
    k_1^\mu k_2^\nu
    \sbr{p_{1\mu} p_{2\nu} + p_{1\nu} p_{2\mu} - p_1^\alpha p_{2\alpha} \eta_{\mu\nu}}
    + k_1^\nu k_2^\mu
    \sbr{p_{1\mu} p_{2\nu} + p_{1\nu} p_{2\mu} - p_1^\alpha p_{2\alpha} \eta_{\mu\nu}}
    \\&\qquad
    - k_1^\alpha k_{2\alpha} \eta^{\mu\nu}
    \sbr{p_{1\mu} p_{2\nu} + p_{1\nu} p_{2\mu} - p_1^\alpha p_{2\alpha} \eta_{\mu\nu}}
    +
    M^2
    \sbr{2 \tens p_1 \cdot \tens p_2 - 4 \tens p_1 \cdot \tens p_2}
    \Bigg] \\
    \intertext{%
        The third bracket can already be contracted.
    }
    &= \frac{e^4}{s^2} \Bigg[
    k_1^\mu k_2^\nu
    \sbr{p_{1\mu} p_{2\nu} + p_{1\nu} p_{2\mu} - \tens p_1 \cdot \tens p_2 \eta_{\mu\nu}}
    + k_1^\nu k_2^\mu
    \sbr{p_{1\mu} p_{2\nu} + p_{1\nu} p_{2\mu} - \tens p_1 \cdot \tens p_2 \eta_{\mu\nu}}
    \\&\qquad
    + 2 [\tens k_1 \cdot \tens k_2] [\tens p_1 \cdot \tens p_2]
    -
    2 M^2 \tens p_1 \cdot \tens p_2
    \Bigg]
    \intertext{%
        The remaining terms which are not contracted are symmetric in their
        indices. We can combine those.
    }
    &= 2 \frac{e^4}{s^2} \sbr{
        k_1^\mu k_2^\nu
        \sbr{p_{1\mu} p_{2\nu} + p_{1\nu} p_{2\mu} - \tens p_1 \cdot \tens p_2 \eta_{\mu\nu}}
        + [\tens k_1 \cdot \tens k_2] [\tens p_1 \cdot \tens p_2]
        - M^2 \tens p_1 \cdot \tens p_2
    }
    \intertext{%
        And now we need to do the contractions.
    }
    &= 2 \frac{e^4}{s^2} \sbr{
          [\tens k_1 \cdot \tens p_2] \, [\tens k_2 \cdot \tens p_1]
        + [\tens k_1 \cdot \tens p_1] \, [\tens k_2 \cdot \tens p_2]
        - [\tens k_1 \cdot \tens k_2] \, [\tens p_1 \cdot \tens p_2]
        + [\tens k_1 \cdot \tens k_2] \, [\tens p_1 \cdot \tens p_2]
        - M^2 \tens p_1 \cdot \tens p_2
    }
    \intertext{%
        The middle terms cancel, so we only have three terms left.
    }
    &= 2 \frac{e^4}{s^2} \sbr{
        [\tens k_1 \cdot \tens p_2] [\tens k_2 \cdot \tens p_1]
        + [\tens k_1 \cdot \tens p_1] [\tens k_2 \cdot \tens p_2]
        - M^2 \tens p_1 \cdot \tens p_2
    }
\end{align*}

The actual result should be: \parencite[(5.10)]{Peskin/QFT/1995}
\[
    \frac 14 \sum_\text{spins} |\mathscr M|^2
    = \frac{8 e^4}{q^4} \sbr{
        [\tens p_1 \cdot \tens k_1] [\tens p_2 \cdot \tens k_2]
        + [\tens p_1 \cdot \tens k_2] [\tens p_2 \cdot \tens k_1]
        + M^2 \tens p_1 \cdot \tens p_2
    }.
\]
So we have a minus sign in front of the mass term and are missing a factor of
four.

\subsection{Differential cross section}

The differential cross section in the center of mass system should only depend
on the scattering angle $\theta$ and the energy $E$ each electron brings into
the collision. We again use the center of mass diagram which we reproduced in
Figure~\ref{fig:center_of_mass}.

\begin{figure}[hbp]
    \centering
    \begin{tikzpicture}
        \draw (0, 0) circle (3);

        \draw[<-, thick] (90:.1) -- (90:3) node[midway, left] {$\vec p_1$};
        \draw[<-, thick] (270:.1) -- (270:3) node[midway, right] {$\vec p_2$};
        \draw[->, thick] (210:.1) -- (210:3) node[midway, above left] {$\vec k_1$};
        \draw[->, thick] (30:.1) -- (30:3) node[midway, above left] {$\vec k_2$};

        \draw[<-] (210:.7) arc (210:270:.7);
        \node at (240:1) {$\theta$};
    \end{tikzpicture}
    \caption{%
        Scattering of two particles in the center of mass system. Shown are the
        momentum three-vectors. The gap in the center is just for presentation
        purposes.
    }
    \label{fig:center_of_mass}
\end{figure}

From the diagram one can read off the relations between the three-momenta and
compute the following scalar products of the four-momenta.

\begin{itemize}
    \item

        The spatial parts $\vec p_1$ and $\vec p_2$ are the negatives of each
        other. The electrons are assumed to be massless, so the norm of the
        three-momenta is just the energy itself.
        \[
            \tens p_1 \cdot \tens p_2 = E^2 - \vec p_1 \cdot \vec p_2 = 2E^2
        \]

    \item

        Mixed terms will include the scattering angle.
        \[
            \tens k_1 \cdot \tens p_1
            \overset 1= E^2 - |\vec k_1| \, |\vec p_1| \cos(\theta)
            \overset 2= E^2 - E |\vec k_1| \cos(\theta)
            \overset 3= E^2 \sbr{1 - \frac{|\vec k|}{E} \cos(\theta)}
        \]
        About the steps: (1) We have used the definition of the scalar product
        of four-vectors and the angle relation of the three-vector scalar
        product. (2) The norm of the incoming momentum is the energy since we
        assumed the electron to be massless. (3) We factor out $E$. The norms
        of $\vec k_1$ and $\vec k_2$ are the same, so we omit the index when we
        refer to the norm only.

    \item

        Now we look at the other mixed terms. The steps are the same, just that
        the angle is shifted by $\piup$.
        \[
            \tens k_2 \cdot \tens p_1
            = E^2 - |\vec k| \, |\vec p_1| \cos(\theta + \piup)
            = E^2 + E |\vec k| \cos(\theta)
            = E^2 \sbr{1 + \frac{|\vec k|}{E} \cos(\theta)}
        \]
\end{itemize}

We can take those terms to express the invariant matrix element in this
coordinate system. We will use the version from
\textcite[(5.10)]{Peskin/QFT/1995} to have a chance to get the correct answer.
\begin{align*}
    \frac 14 \sum_\text{spins} |\mathscr M|^2
    &= \frac{8 e^4}{q^4} \sbr{
        [\tens p_1 \cdot \tens k_1] [\tens p_2 \cdot \tens k_2]
        + [\tens p_1 \cdot \tens k_2] [\tens p_2 \cdot \tens k_1]
        + M^2 \tens p_1 \cdot \tens p_2
    } \\
    &= \frac{8 e^4}{16 E^4} \sbr{
            E^4 \sbr{1 - \frac{|\vec k|}{E} \cos(\theta)}^2
            + E^4 \sbr{1 + \frac{|\vec k|}{E} \cos(\theta)}^2
        + 2 E^2 M^2
    }
    \intertext{%
        We expand the squares. The middle terms will just cancel due to the
        opposing signs.
    }
    &= \frac{e^4}{2 E^4} \sbr{
    2 E^4 +
    2 E^4 \sbr{\frac{|\vec k|}{E} \cos(\theta)}^2
        + 2 E^2 M^2
    } \\
    &= e^4 \sbr{1 + \frac{|\vec k|^2}{E^2} \cos(\theta)^2 + \frac{M^2}{E^2}}
    \intertext{%
        We know that $E^2 = |\vec k^2| + M^2$ and therefore we can write this
        as
    }
    &= e^4 \sbr{1 + \frac{E^2 - M^2}{E^2} \cos(\theta)^2 + \frac{M^2}{E^2}}.
    \intertext{%
        We expand the fraction.
    }
    &= e^4 \sbr{
        1
        + \cos(\theta)^2
        - \frac{M^2}{E^2} \cos(\theta)^2
        + \frac{M^2}{E^2}
    }
    \intertext{%
        Then we can factor out the cosine squared and obtain the result also
        given by \textcite[(5.11)]{Peskin/QFT/1995}.
    }
    &= e^4 \sbr{
    1 + \frac{M^2}{E^2}
    + \sbr{1 - \frac{M^2}{E^2}} \cos(\theta)^2
    }
\end{align*}

From here we can plug this into Equation~(2) given on the problem set and
obtain the differential cross section.
\begin{align*}
    \od{\sigma_\mathrm{cm}}{\cos \theta}
    &= \frac{1}{2 E_\mathrm{cm}^2} \frac{|\vec k|}{16 \piup^2 E_\mathrm{cm}}
    \frac 14 \sum_\text{spins} |\mathscr M|^2 \\
    &= \frac{1}{2 E_\mathrm{cm}^2} \frac{|\vec k|}{16 \piup^2 E_\mathrm{cm}}
    e^4 \sbr{1 + \frac{M^2}{E^2} + \sbr{1 - \frac{M^2}{E^2}} \cos(\theta)^2}
    \intertext{%
        We can simplify the first term a bit.
    }
    &= \frac{e^4}{32 \piup^2 E_\mathrm{cm}^3} \sqrt{E^2 - M^2}
    \sbr{1 + \frac{M^2}{E^2} + \sbr{1 - \frac{M^2}{E^2}} \cos(\theta)^2}
    \intertext{%
        We extract one factor of the energy $E$ out of the square root to make
        it look like the other factors in the large bracket.
    }
    &= \frac{e^4}{32 \piup^2 E_\mathrm{cm}^3} E \sqrt{1 - \frac{M^2}{E^2}}
    \sbr{1 + \frac{M^2}{E^2} + \sbr{1 - \frac{M^2}{E^2}} \cos(\theta)^2}
    \intertext{%
        The relation between single particle energy $E$ and the center of mass
        energy is $E_\mathrm{cm} = 2 E$. The fine structure constant is $\alpha
        = e^2 / [4 \piup]$ We cancel the one factor of $E$, introduce the fine
        structure constant and obtain the final result.
    }
    &= \frac{\alpha^2}{4 E_\mathrm{cm}^2} \sqrt{1 - \frac{M^2}{E^2}}
    \sbr{1 + \frac{M^2}{E^2} + \sbr{1 - \frac{M^2}{E^2}} \cos(\theta)^2}
\end{align*}

This result matched the one given by \textcite[(5.12)]{Peskin/QFT/1995}.

\subsection{Electron muon scattering}

\section{Decay of the charged pion}
\label{homework:1}



\end{document}

% vim: spell spelllang=en tw=79
